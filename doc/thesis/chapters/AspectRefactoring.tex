% !TEX root = ../thesis.tex

%%%%%%%%%%%%%%%%%%%%%%%%%%%%%%%%%%%%%%%%%%%%%%%%%%%%%%%%%%%%%%%%%%%%%%%%%%%%%%%
% Chapter: Aspect refactoring 
%%%%%%%%%%%%%%%%%%%%%%%%%%%%%%%%%%%%%%%%%%%%%%%%%%%%%%%%%%%%%%%%%%%%%%%%%%%%%%%
\chapter{Aspect refactoring of JHotDraw in managed data}\label{AspectRefactoring}

\section{Introduction}

\section{Aspect Refactoring}

\subsection{Aspect Solution Templates}

\section{Aspect Refactoring of JHotDraw}

\subsection{Observer Pattern in JHotDraw}\label{Observer Pattern in JHotDraw}

\subsubsection{FigureSelectionListener}

\subsection{Undo Concern of JHotDraw}\label{Undo JHotDraw}

\subsubsection{Current Undo Implementation}
% A number of activities in JHOTDRAW, such as handling font sizes and colors, image rotation, or inserting the clipboard's content into a drawing, support the undo functionality. 

\section{Threads to Validity}
% all study designs have flaws. By acknowledging them explicitly, the researchers show that they are aware of the flaws and have taken reasonable steps to minimize their effects.

% Construct Validity: Efficient intepretation of the results 
% Internal validity: Study design
% External validity: Justified results (is it the right case?)
% Reliability validity: same results on replication?