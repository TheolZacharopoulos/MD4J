% !TEX root = ../thesis.tex

%%%%%%%%%%%%%%%%%%%%%%%%%%%%%%%%%%%%%%%%%%%%%%%%%%%%%%%%%%%%%%%
% Chapter: Acknowledgments
%%%%%%%%%%%%%%%%%%%%%%%%%%%%%%%%%%%%%%%%%%%%%%%%%%%%%%%%%%%%%%%

\chapter*{Acknowledgments}
It all started when my supervisor, Tijs van der Storm, presented us the classic Aspect Oriented Programming paper during our software construction course of this Master's program.
I was always thinking that scattering was annoying while tangling looked very bad, but I did not know the name. 
The Aspect Oriented Programming paper explained it to me explicitly.
First thing was to go to Tijs and ask him if AOP really works, since I was not really convinced.
How can no one use this technique if it really works?
Tijs then explained to me that he and William R. Cook worked together on something different, that would solve that problem in a different way, the Managed Data.
That day I left from his office with a list of papers discussing AOP and Managed Data.
I started implementing it in Java the same day in order to see if it actually works.
Well, it worked, and I got excited.
This was the time I knew I wanted to do my research on this.

For this thesis, I would first like to express my sincere gratitude to Tijs van der Storm, my supervisor, who introduced me to this idea.
He helped me with the problems I faced during this process.
I will never forget the self-describing schema process; Tijs had the patience to explain it to me a number of times.
I also want to thank Tijs because he was a real inspiration to me during my master's program, in particular his way of thinking, his knowledge and of course his perfectionism. 
He changed my way of programming and thinking about programming entirely.

Moreover, I would like to thank my colleagues and friends from CWI for the interesting and inspiring conversations.
However, my biggest support was from Ifigeneia.
The writing of this thesis would never be complete if she was not on my side.
I am very grateful to Ifigeneia for her precious constant help, irreplaceable support and her interest on the subject during its completion.

