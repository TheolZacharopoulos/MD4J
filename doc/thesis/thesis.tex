
\documentclass{uvamscse}

\input{program-listings}
\newcommand{\cmd}[1]{\texttt{$\backslash$#1}}

\usepackage[toc,acronym,nogroupskip]{glossaries}
\usepackage{url}

\usepackage{graphicx}
\graphicspath{ {figures/} }

\usepackage{booktabs}
\usepackage{multirow}
\usepackage{scrextend}

% \usepackage[subtle]{savetrees}

% ====== Listing Setup =======
\usepackage{color}
\usepackage{listings}
\usepackage{parcolumns}
\definecolor{javared}{rgb}{0.6,0,0} % for strings
\definecolor{javagreen}{rgb}{0.25,0.5,0.35} % comments
\definecolor{javapurple}{rgb}{0.5,0,0.35} % keywords
\definecolor{javadocblue}{rgb}{0.25,0.35,0.75} % javadoc
\definecolor{darkgray}{rgb}{.4,.4,.4}
 
\lstset{language=Java,
	basicstyle=\ttfamily,
	keywordstyle=\color{javapurple}\bfseries,
	stringstyle=\color{javared},
	commentstyle=\color{javagreen},
	morecomment=[s][\color{javadocblue}]{/**}{*/},
	keywordstyle={[2]\color{red}\bf},
	numbers=left,
	numberstyle=\tiny\color{black},
	stepnumber=1,
	numbersep=10pt,
	tabsize=2,
	showspaces=false,
	showstringspaces=false
}

\lstdefinelanguage{AspectJ}[]{Java}{
    morekeywords={declare, pointcut, aspect, before, around, after, returning, throwing, call, execution, this, target, args, within, withincode, get, set, initialization, preinitialization, staticinitialization, handler, adviceexecution, cflow, cflowbelow, if, proceed},
    moredelim=[is][\textcolor{darkgray}]{\%\%}{\%\%},
    moredelim=[il][\textcolor{darkgray}]{§§}
}

% ====== Glossaries setup =======
\makeglossaries
\newcommand{\ac}[1]{\gls{#1}}
\newcommand{\acs}[1]{\acrshort{#1}}
\newcommand{\acp}[1]{\glspl{#1}}

\newacronym{ccc}{CCC}{Cross Cutting Concerns}
\newacronym{oop}{OOP}{Object Oriented Programming}
\newacronym{aop}{AOP}{Aspect Oriented Programming}
\newacronym{pp}{PP}{Procedural Programming}
\newacronym{dsl}{DSL}{Domain Specific Language}
\newacronym{aom}{AOM}{Adaptive Object Model}
\newacronym{mop}{MPO}{Meta-Object Protocol}
\newacronym{mdsd}{MDSD}{Model Driven Software Development}
\newacronym{jvm}{JVM}{Java Virtual Machine}

%%%%%%%%%%%%%%%%%%%%%%%%%%%%%%%%%%%%%%%%%%%%%%%%%%%%%%%%%%%%%%
% Title
%%%%%%%%%%%%%%%%%%%%%%%%%%%%%%%%%%%%%%%%%%%%%%%%%%%%%%%%%%%%%%
\title{Taming Aspects with Managed Data }
% \coverpic[100pt]{figures/terminal.png}
\subtitle{Master's Project in Software Engineering}
\date{Summer 2016}

\author{Theologos A. Zacharopoulos}
\authemail{theol.zacharopoulos@cwi.nl}
\supervisor{Tijs van der Storm}
\host{Centrum Wiskunde \& Informatica, \url{http://www.cwi.nl}}

\begin{document}

\abstract{
	The problem of crosscutting concerns results to code scattering and tangling.
	This in turn significantly affects the modularity of a system, since the separation of concerns principle is violated.
	Although Aspect Oriented Programming is created in order to solve this problem, several issues arise, the most important being the coupling between aspects and components.
	This leads to an evolution paradox in Aspect Oriented Software Development.

	In this thesis we implement managed data and use them in order to solve the problem of crosscutting concerns.
	Managed data is a data abstraction mechanism that allows the programmer to define data in addition to their manipulation mechanisms, something that is hard-coded in the programming languages.
	Defining crosscutting concerns in the manipulation mechanisms, namely data managers, results to a modular way of controlling aspects of data.
	Therefore, managed data can be used to solve the crosscutting concerns problem and avoid the coupling problem of Aspect Oriented Programming.

	Our work focuses on the implementation of managed data in Java, proving that managed data can be implemented in a static language.
	Moreover, we demonstrate that managed data can tame aspects by refactoring an existing program's crosscutting concerns.
	Finally, we provide an evaluation of our managed data refactoring in relation to an Aspect Oriented refactoring of the same use case application, in order to show how managed data can overcome Aspect Oriented Programming problems.
}

\maketitle

\printglossary[type=\acronymtype, title=Abbreviations]

%%%%%%%%%%%%%%%%%%%%%%%%%%%%%%%%%%%%%%%%%%%%%%%%%%%%%%%%%%%%%%
% Chapters
%%%%%%%%%%%%%%%%%%%%%%%%%%%%%%%%%%%%%%%%%%%%%%%%%%%%%%%%%%%%%%

%%%%%%%%%%%%%%%%%%%%%%%%%%%%%%%%%%%%%%%%%%%%%%%%%%%%%%%%%%%%%%%%%%%%%%%%%%%%%%%
% Chapter: Introduction
%%%%%%%%%%%%%%%%%%%%%%%%%%%%%%%%%%%%%%%%%%%%%%%%%%%%%%%%%%%%%%%%%%%%%%%%%%%%%%%
\chapter{Introduction}\label{Introduction}
\ac{ccc} is a problem for which the classic programming techniques can not tackle with sufficiently. 
This results in scattered and tangled code, which affects the system's modularity and it's ease of maintenance and evolution. 
Since \ac{oop} and \ac{pp} techniques can not solve this problem, \ac{aop} presented \cite{kiczales1997aspect} in order to
provide a solution by introducing the notion of \textit{aspects}.

\ac{aop} results in a modular and \textit{single-responsibility} design whose properties must be implemented as \textit{components} (cleanly encapsulated procedure) and \textit{aspects} (not clearly encapsulated procedure), both separate concepts that are combined for the result through a process called \textit{weaving}. 
However, relying on \ac{aop}, paradoxically, does not improve the evolution of a project even with the modularity that it provides 
since it introduces tight coupling between the aspects and the application. 
As a result the way to tackle with this problem we need a more sophisticated and expressing crosscut language.
Consequently, \ac{ccc} could be handled in a higher level of the language such as the data structuring and management mechanisms.

Managed data \cite{loh2012managed} allows programmers to take control of important aspects of data as reusable modules. 
Using managed data a developer can build \textit{data managers} that handle the fundamental data manipulation primitives 
that are usually hard-coded in the programming language, by introducing custom data manipulation mechanisms. 
Managed data have been researched and implemented under the Enso project\footnote{\label{enso}\url{http://enso-lang.org/}}, which is developed in Ruby\footnote{\label{ruby}\url{https://www.ruby-lang.org/en/}} (a dynamic programming language) using Ruby’s reflection capabilities. 
Furthermore, managed data are considered less able to be supported in static languages directly which makes it more challenging for 
this thesis since it is going to be implemented in Java.
In this thesis I am going to use the Java reflection capabilities to implement managed data and focus on specific aspects and design patterns implementations using the data managers concept of managed data. 

%%%%%%%%%%%%%%%%%%%%%%%%%%%%%%%%%%%%%%%%%%%%%%%%%%%%%%%%%%%%%%%%%%%%%%%%%%%%%%%
\section{Initial Study}\label{Initial Study}
In their study on managed data, Loh et al. \cite{loh2012managed} present the main idea of managed data, while using a show case of it in an implementation in Ruby. As a use case they present the Enso project in order to reuse database management and  access control mechanisms across different data definitions.

This thesis is an extension of their work; we implement managed data in Java (a static programming language) using the Java reflection API\footnote{\url{https://docs.oracle.com/javase/tutorial/reflect/}} and dynamic proxies\footnote{\url{https://docs.oracle.com/javase/8/docs/api/java/lang/reflect/Proxy.html}}. 
Although proxies in static programming languages can not implement the full range of managed data \cite{loh2012managed}. 
Java provides a strong implementation of the \ac{mop} \cite{kiczales1991art}, which can be used though the Java Reflection API \cite{forman2004java}. 
Additionally, our work focuses on the aspects perspective and it provides a solution to the \ac{ccc} problem by using managed data
and their data managers.

%%%%%%%%%%%%%%%%%%%%%%%%%%%%%%%%%%%%%%%%%%%%%%%%%%%%%%%%%%%%%%%%%%%%%%%%%%%%%%%
\section{Problem statement}\label{Problem statement}

%%%%%%%%%%%%%%%%%%%%%%%%%%%%%%%%%%%%%%%%%%%%%%%%%%%%%%%%%%%%%%%%%%%%%%%%%%%%%%%
\subsection{Problem Analysis}\label{Problem Analysis}
% TODO: FOCUS HERE 
% An analysis of the problem: where does it occur and how, how often, and what are the consequences? An important part is also to scope the research: what aspects are included and what aspects are deliberately left out, and why?

The problem we study regards the \ac{ccc} that are scattered around the application, resulting to a hard to maintain system 
by tangling implementation logic and concerns code together.
Even though, \ac{aop} provides new modularization mechanisms, which should result in easier evolving software, 
it delivers solutions that are as hard and sometimes even harder to evolve than before \cite{tourwe2003existence}. 
The problem lays on the aspects, which have to include a crosscut description of all places in the application where this code yields an influence. 
Thus, the aspects are tightly coupled to the application and this greatly affects the evolvability of the overall system. 

Additionally, Friedrich Steimann \cite{steimann2005domain} argues that modeling languages are not aspect ready. 
The problem that arises is located at the level of software modeling. 
More specifically, in \textit{roles modeling}, whereas in \ac{oop} roles are tied to the collaborations,
collaborations rely on interactions of objects, and aspects on the other hand are typically defined independently of one another. 

Furthermore, in terms of order, it has been observed that aspects are not elements of the domain, they describe the order rather than the domain. 
Finally, aspects invariably express non-functional requirements, but if the non-functional requirements are not elements of domain models then neither are aspects.

In order to solve the aforementioned problems, we implement manage data, lifting the data management up to the application.

%%%%%%%%%%%%%%%%%%%%%%%%%%%%%%%%%%%%%%%%%%%%%%%%%%%%%%%%%%%%%%%%%%%%%%%%%%%%%%%
\subsection{Research Questions}\label{Research Questions}
Managed data has not been practically implemented in a static language before, therefore my first research questions states 
``Can managed data be implemented in a static language like Java?''. 
Based in the previous argumentation about the relevance of \ac{aop} and the solutions that managed data can provide in \ac{ccc}, my second research question is ``Can managed data solve \ac{ccc} and to what extend does it improve the software evolution problems that \ac{aop} introduces in a modular solution?''. 
Finally by using a software showcase, the JHotDraw framework, as well as its \ac{aop} implementation AJHotDraw \cite{marinajhotdraw}, 
I am going to evaluate the implementation of managed data on an inventory of aspects and design patterns. 
As a result the third research question states ``To what extent can managed data tame an inventory of aspects and design patterns in the JHotDraw framework, in contrast with the original and the AOP implementation.''

%%%%%%%%%%%%%%%%%%%%%%%%%%%%%%%%%%%%%%%%%%%%%%%%%%%%%%%%%%%%%%%%%%%%%%%%%%%%%%%
\subsection{Solution Outline}\label{Solution Outline}
Our solution is consisted of an implementation of managed data in Java. 
This framework can be used by applications in order to deal with \ac{ccc}.

To validate our hypotheses we are going to implement managed data in Java using the Java Reflection API and Dynamic Proxies. 
More specifically we are going to use Java interfaces for \textit{schemas} and dynamic proxies for \textit{data managers}. 
Furthermore, we are going to provide as a proof of concept the examples given in \cite{loh2012managed} but this time developed in Java. As mentioned in \cite{loh2012managed} to stack data managers I am going to use the Decorator Pattern \cite{gamma1995design}. 

In order to prove that managed data solves the problems that \ac{aop} introduces, we are going to implement an inventory of the following aspects and design patterns from JHotDraw using data managers:
\begin{description}
  \item[The Observer Pattern,] which as presented in literature \cite{tourwe2003existence} \cite{hannemann2005role} \cite{marin2005approach}, is by nature not modularized and the scatters pattern code through the classes. 
  This pattern is considered as a difficult case because it is used a lot in the original JHotDraw source code but with multiple variations, thus it is difficult to extract an abstract version.

  \item[The Singleton Pattern,] which as presented \cite{hannemann2005role} \cite{hannemann2002design}, it can easily be abstracted as an aspect and replace the \ac{oop} usage in JHotDraw. 

  \item[The Template Method,] which as presented \cite{hannemann2005role} \cite{hannemann2002design}, it scatters code by introducing roles such as those of \textit{AbstractClass} and \textit{ConcreteClass}.

  \item[The Undo aspect,] which is analyzed extensively \cite{marin2004refactoring} and a solution is provided by AJHotDraw. 
  More specifically, this aspect consists of aspect-oriented refactoring of the \textit{Command} pattern with \textit{Undo} actions.
\end{description}

This inventory is implemented using data managers that have modularity as a main characteristic and is been evaluated in a new JHotDraw implementation. 
We compare those aspects with the original version of JHotDraw, and the aspect version, AJHotDraw. 
Since our solution is a refactoring of the JHotDraw framework we need a way to ensure the behavioral equivalence between the original and the refactored solution \cite{fowler2009refactoring}. 
However, JHotDraw comes with no tests. 
Thus, we use the TestJHotDraw, which is a subproject of the AJHotDraw development team, and it is developed in order to contribute to a gradual and safe adoption of aspect-oriented techniques in existing applications and allow for a better assessment of aspect orientation.
Since we use our JHotDraw implementation for the functional evaluation of our solution, we can use the presented criteria \cite{hannemann2002design}, which are \textit{Locality},\textit{Re-usability}, \textit{Composition Transparency}, and \textit{(Un)pluggability}, in order to present metrics of our solution. 

%%%%%%%%%%%%%%%%%%%%%%%%%%%%%%%%%%%%%%%%%%%%%%%%%%%%%%%%%%%%%%%%%%%%%%%%%%%%%%%
\subsection{Research Method}\label{Research Method}
The answers for the research questions have been extracted from the background material, in our Java managed data implementation and an evaluation of our implementation in an existing use case system the JHotDraw.

\begin{description}

  \item[Managed data implementation in a static language.]
  In order to answer the question if managed data could be implemented in a static language, we've implemented managed data in Java 
  using Java's reflection capabilities\footnote{\url{https://docs.oracle.com/javase/tutorial/reflect/}}, using Java interfaces 
  for schemas definition and dynamic proxies\footnote{\url{https://docs.oracle.com/javase/8/docs/api/java/lang/reflect/Proxy.html}}
  for the data managers. An extensive presentation of the implementation is given in Chapter \ref{Implementation}.

  \item[Use case implementation and evaluation.] In order to argue about the contribution of our implementation and managed data for aspects handling in general, we've used an use case application (JHotDraw) which is considered as a good design use case for \ac{oop}, along with it's \ac{aop} implementation (AJHotDraw). Using these application as references we've implemented our own version of JHotDraw using managed data to tame some of it's aspects, the results are presented extensively in Chapter \ref{Evaluation}.

\end{description}	

%%%%%%%%%%%%%%%%%%%%%%%%%%%%%%%%%%%%%%%%%%%%%%%%%%%%%%%%%%%%%%%%%%%%%%%%%%%%%%%
\section{Contributions}\label{Contributions}

\begin{description}
  \item[Contribution 1: Managed data implementation in Java.]
  Our first contribution with this thesis is the implementation of managed data in a static language like Java.
  Managed data implemented as an internal \ac{dsl} in Java, using interfaces for schema definitions and dynamic proxies
  for the data managers.

  \item[Contribution 2: Managed data Java framework.]
  The final deliverable is a Java library with which the developer can define and implement aspects as reusable modules
and integrate them with an application without mixing the business logic with concern logic.
More specifically, the schemas and the data managers have to be defined by the developer, as well as any additional
functionality that may needed to be integrated to the patterns or roles of the application.

  \item[Contribution 3: Managed data Evaluation in JHotDraw.]
  We implemented a new version of the JHotDraw application using our framework in order to evaluated our \ac{ccc} solution.
  More specifically, we focused on the \textit{Undo} concern, which is scattered around the implementation of the JHotDraw.

  \item[Contribution 4: JHotDraw implementation results assessment and comparison with AJHotDraw.]
  Finally, we present the results of our evaluation and we compare them we another implementation of the JHotDraw which 
  implements aspects using \ac{aop} and the AspectJ language, the AJHotDraw.

\end{description}

%%%%%%%%%%%%%%%%%%%%%%%%%%%%%%%%%%%%%%%%%%%%%%%%%%%%%%%%%%%%%%%%%%%%%%%%%%%%%%%
\section{Related Work}\label{Related Work}
In this section we discuss the related work of research that inspired this thesis. 
More specifically, we discuss points that we followed and points that we've tried to improve as well as the reason of doing it.

\begin{description}

  \item[Meta-Object Protocol]~\\
  Managed data can be implemented using reflection and the \ac{mop}. 
  The authors of Enso \cite{loh2012managed} implemented it in Ruby using reflection and the \textbf{method\_missing} mechanism. 
  In other languages (such as Java, JavaScript or C\#) that support dynamic proxies, they can be used for the managed data implementation, which is the way we've implemented it.
  The \ac{mop} \cite{kiczales1991art} was first implemented for simple \ac{oop} capabilities of the Lisp language in order to satisfy some developer demands including compatibility, extensibility and developers experimentation. 
  The idea was that the languages have been designed to be viewed as black box abstractions without giving the programmers the control over semantics or the implementation of those abstractions. 
  \ac{mop} opens up those abstractions to the programmer so he can adjust aspects of the implementation strategy. 
  Providing an open implementation can be advantageous in wide range of high-level languages and that \ac{mop} technology is a powerful tool for providing that power to the programmer \cite{kiczales1991art}. 
  Furthermore, \ac{mop} provides flexibility to the programmer because as a language becomes more and more high level and it's expressive power becomes more and more focused, the ability to cleanly integrate something outside the language's scope becomes more and more difficult. 
  Thus, both \ac{mop} and managed data allow the programmer to be able to control the interpretation of structure and behavior in a program.
  However, \ac{mop} focuses on behavior of the objects and classes, while in managed data focus on the data management only.
  One could conclude that managed data is a subset of the \ac{mop} approach since managed data have a more narrow scope.


  \item[Adaptive Object Model]~\\
  Managed data \cite{loh2012managed} is closely related to the \ac{aom}. \ac{aom} \cite{yoder2002adaptive} is an architectural style that emphasizes flexibility and runtime dynamic configuration. 
  Architectures that are designed to adapt at runtime to new user requirements by retrieving descriptive information that can be interpreted at runtime, are sometimes called a ``reflective architecture'' or a ``meta architecture''. 
  An \ac{aom} system, is a system that represents classes, attributes, relationships, and behavior as metadata, something that is closely related to the managed data.
  However, on one hand \ac{aom} style is more general than the managed data since it is described at a very high level as a pattern language. Additionally, it covers business rules and user interfaces, in addition to data management. 
  On the other hand, \ac{aom} does not discuss issues of integration with programming languages, the representation of data schemas, or of bootstrapping, which are central characteristics of managed data. \ac{aom} is also presented as a technique for implementing business systems, not as a general programming or data abstraction technique \cite{loh2012managed}.

  \item[Model Driven Software Development]~\\
  \ac{mdsd} refers to a software development method which generates code from defined models. 
  The model represent abstract data that consisted of the structure and properties definition of an entity.
  The idea of the model in \ac{mdsd} is closely related to the \textit{schemas} in managed data.
  Similarly to the model definition, schemas define the structure, the properties and any meta-data that describe an entity
  following by a code generation that adds functionality to that entity for it's manipulation.

  \item[The Enso Language]~\\
  Enso project\footnote{\url{http://enso-lang.org/}} is the first implementation of managed data, 
  it is open source\footnote{\url{https://github.com/enso-lang/enso}} and is used for EnsoWeb, a web framework written with managed data.
  Although Enso is implemented in Ruby, which is a dynamic language, the source code was a informative place to start for our static implementation in Java. 
  The structure of Enso was an inspiration of our implementation even though some parts have changed completely in order to follow our needs and support of Java's static system.
  Additionally, examples presented in Enso, are also implemented in our case and are presented in the Appendix \ref{Example Application}.

  \item[Aspect Oriented Programming]~\\
  Although \ac{aop} is not directly connected to managed data, it allows a mechanism that is relative easy to be supported in managed data.
  This mechanism consist of the \textit{weaving} of aspect code in specific join points. 
  The way to support this mechanism in managed data it is through data managers, which it is the main topic of this thesis and 
  is going to presented deeply in the following chapters.

\end{description}

%%%%%%%%%%%%%%%%%%%%%%%%%%%%%%%%%%%%%%%%%%%%%%%%%%%%%%%%%%%%%%%%%%%%%%%%%%%%%%%
\section{Document Outline}\label{Document Outline}
In this section we outline the structure of this thesis. 
In Chapter \ref{Background} we introduce the background, focusing on the concepts, which the reader must be familiarizes with 
in order to follow the next chapters.
In Chapter \ref{Theory} a basic theory is presented, which is going to be used a a reference for our implementation.
In Chapter \ref{Implementation} the implementation is demonstrated and analyzed, providing explanation of our issues 
and implementation details.
Next, in Chapter \ref{Evaluation} an evaluation of our implementation is presented, by applying it in JHotDraw. 
Additionally, some metrics, claims and results are presented.
Finally, a conclusion is given in Chapter \ref{Conclusion} along with some further work in \ref{Further Work}.


%%%%%%%%%%%%%%%%%%%%%%%%%%%%%%%%%%%%%%%%%%%%%%%%%%%%%%%%%%%%%%%%%%%%%%%%%%%%%%%
% Chapter: Background
%%%%%%%%%%%%%%%%%%%%%%%%%%%%%%%%%%%%%%%%%%%%%%%%%%%%%%%%%%%%%%%%%%%%%%%%%%%%%%%
\chapter{Background}\label{Background}

%%%%%%%%%%%%%%%%%%%%%%%%%%%%%%%%%%%%%%%%%%%%%%%%%%%%%%%%%%%%%%%%%%%%%%%%%%%%%%%
% Section: CCC
%%%%%%%%%%%%%%%%%%%%%%%%%%%%%%%%%%%%%%%%%%%%%%%%%%%%%%%%%%%%%%%%%%%%%%%%%%%%%%%
\section{Cross Cutting Concerns}\label{Cross Cutting Concerns}
A lot of research in software engineering focuses on the importance of software modularity. 
The number of advantages of modular systems are outrageous, with one of the most significant, the extensibility and evolution of a system \cite{parnas1972criteria}.

During the development of a system though there are many concerns that they have to be considered and implemented into the system. 
In order to follow the modularity principles, those concerns have to be implemented in a separate modules, this way the program will be extensible and it's evolution easier.
However, many of those concerns can not fit into the existing modular mechanisms of any of the programming paradigms including both \ac{oop} and \ac{pp}. 
In those cases, the concerns are scattered through the modules of the system, resulting to scattered and tangled code, those concerns called \acrlong{ccc} \cite{hannemann2005role}.
\ac{ccc} considered a significant issue for the evolution of a system and the affects of tangled and scattered code disastrous for a system's extensibility.

The reason is that, the code that it is related to one concern now is scattered in multiple modules, while the concern code is now tangled with the module's logic. 
This results to a system that does not follow the \textit{Single Responsibility Principle} principle and consequently the system will be hard to maintain.

Some examples of those \ac{ccc} are the following: persistence, caching, logging, error handling \cite{lippert2000study}, access control and many more, as well as some design patterns that scatter ``design pattern code'' thought the application, such as the \textit{Observer Pattern}, \textit{template Pattern}, \textit{command Pattern} etc. \cite{hannemann2002design} \cite{marin2004refactoring}.

In order to solve this problem we need a way to refactor to transform the non-modularized CCC into a modular aspect.
Refactorings of \ac{ccc} should replace all the scattered and tangled code of a concern with an equivalent module \cite{hannemann2005role}, which in \ac{aop} they call it \textit{aspect} \cite{kiczales1997aspect}.

%%%%%%%%%%%%%%%%%%%%%%%%%%%%%%%%%%%%%%%%%%%%%%%%%%%%%%%%%%%%%%%%%%%%%%%%%%%%%%%
% Section: AOP
%%%%%%%%%%%%%%%%%%%%%%%%%%%%%%%%%%%%%%%%%%%%%%%%%%%%%%%%%%%%%%%%%%%%%%%%%%%%%%%
\section{Aspect Oriented Programming}\label{Aspect Oriented Programming}

Kiczales et al. present \cite{kiczales1997aspect} using an example of a simple image processing application, that in general, whenever two properties being programmed must compose differently and yet be coordinated (in the example filters and loop-fusion), they \textbf{crosscut} each other. 
Because general purposes languages provide only one composition mechanism, and those mechanisms lead to complexity and tangling, the programmer must do the co-composition manually. 
According to their theory, a property that must be implemented can be either a \textit{component}, if it can be cleanly encapsulated in a generalized procedure, or an \textit{aspect}, if it can not be cleanly encapsulated in a generalized procedure. \ac{aop} supports the programmer in cleanly separating components and aspects from each other, by providing mechanisms that make it possible to abstract and compose them when producing the overall system. 
This is in contrast to general purpose programming, \ac{oop} or \ac{pp}, which support programmers in only separating components from each.

However, implementing \ac{aop} programs is not that easy and several tools are needed. 
More specifically, while a general purpose language needs a language, a \textit{compiler} and a \textit{program} written in the language that implements the application, the \ac{aop} based implementation of an application consists of a \textit{component language} in which the components are programmed, one or more \textit{aspect languages} in which the aspects are programmed, an aspect \textit{weaver} for the combined languages, a \textit{component program} that implements the components using the component language, and one or more \textit{aspect programs} that implement the aspects using the aspect languages. 
Essential to the function of the aspect weaver is the concept of join points, which are those elements of the component language semantics that the aspect programs coordinate with. 

%%%%%%%%%%%%%%%%%%%%%%%%%%%%%%%%%%%%%%%%%%%%%%%%%%%%%%%%%%%%%%%%%%%%%%%%%%%%%%%
\subsection{AspectJ}\label{AspectJ}
There is a lot of work in the area of aspect oriented languages but one the most important contribution is the AspectJ\footnote{\url{https://eclipse.org/aspectj/}} project.
Kiczales et al. introduce AspectJ \cite{kiczales2001overview}. AspectJ is a simple and practical aspect-oriented extension to Java. 
The authors of AspectJ provide examples of programs developed in AspectJ and show that by using it \ac{ccc} can be implemented in clear form, which otherwise would lead to tangled code. 
AspectJ was developed as a compatible extension to Java so that it would facilitate adoption by current Java programmers. 
The compatibility lays on upward compatibility, platform compatibility, tool compatibility, and programmer compatibility. One of the most important characteristic of AspectJ is that it is not a \ac{dsl} but a general purpose language that uses Java's static type system.
Something that it holds for our managed data implementation as well.

%%%%%%%%%%%%%%%%%%%%%%%%%%%%%%%%%%%%%%%%%%%%%%%%%%%%%%%%%%%%%%%%%%%%%%%%%%%%%%%
\subsection{Design Patterns in \acrlong{aop}}\label{Design Patterns in Aspect Oriented Programming}

Hannemann et al. present a showcase of \ac{aop} \cite{hannemann2002design} in which they conduct an aspect-oriented implementation of the GoF design patterns \cite{gamma1995design} in AspectJ, where 17 out of 23 cases show modularity improvements. 
Even though design patterns offer flexible solutions to common software problems, those patterns involve crosscutting structures between roles and classes and objects. 
There are several problems that the \ac{oop} design patterns introduce in respect of \ac{ccc}, and specifically in cases when one object plays multiple roles, many objects play one role, or an object play roles in multiple patterns \cite{sullivan2002advanced} (design pattern composition).

The problem lays on the way a design pattern influences the structure of the system and its implementation, pattern implementations are often tailored to the instance of use, and this often leads them to disappear into the code and lose their modularity \cite{hannemann2002design}. Even worst, in case of multiple patterns used in a system (pattern overlay and pattern composition), it can become difficult to trace particular instances of a design pattern. 
Composition creates large clusters of mutually dependent classes\cite{sullivan2002advanced}, and some design patterns explicitly use other patterns in their solution.

\subsubsection{Observer pattern in \acrlong{aop}}\label{Observer pattern in Aspect Oriented Programming}
Hannemann et al. \cite{hannemann2002design} provide some example implementations of several design patterns, including the \textit{Observer Pattern} in which they focus on with a detailed implementation. As they mention, in a observer pattern implementation, both the \textit{Subject} and the \textit{Observer} have to know about their roles in the pattern and consequently have ``pattern related code'' in them, so that adding and removing code from a class requires changes in that class. 

In their implementation[6] of the observer pattern in AspectJ, they separate abstract aspects for: 
\begin{itemize}
	\item The Subject and Observer roles in classes.

	\item Maintenance of a mapping from Subjects to Observers.

	\item The trigger of Subjects that update Observers.
\end{itemize}

And concrete aspects for each instance of the pattern fills in the specific parts:
\begin{itemize}
	\item Which classes can be Subjects and which Observers.

	\item A set of changes on the Subjects that triggers the Observers.

	\item The specific means of updating each kind of Observer when the update logic requires it.
\end{itemize}


The modularity properties which this implementation relates are:

\begin{description}
	\item [Locality:] All the code that implements the Observer pattern is in the abstract and concrete observer aspects, none of it is in the participant classes. The participant classes are entirely free of the pattern context, and as a consequence there is no coupling between the participants. 

	\item [Reusability:] The core pattern code is abstracted and reusable. The abstract aspect can be reused and shared across multiple Observer pattern instances.

	\item [Composition transparency:] Because a pattern participant’s implementation is not coupled to the pattern, if a Subject or Observer takes part in multiple observing relationships their code does not become more complicated and the pattern instances are not confused. 

	\item [(Un)pluggability:] Because Subjects and Observers don’t need to be aware of their role in any pattern instance, it is possible to switch between using a pattern and not using it in the system. 
\end{description}

% TODO: Too much detail???
\subsubsection{More design pattens in \acrlong{aop}}\label{More design pattens in Aspect Oriented Programming}
% Other patterns that are presented in the paper \cite{hannemann2002design} are clustered based on common features. 
% In case of the \textit{Composite}, \textit{Command}, \textit{Mediator}, \textit{Chain of responsibility}, the roles only used within pattern aspect. 
% More specifically, these patterns introduce roles that need no client-accessible interface and are only used within the pattern, e.g. in Composite pattern, to allow walking the tree structure inherent to the parents, the authors define facilities to have visitor traverse and/or change the structure. 

% There visitors are defined in the concrete aspect.
% The patterns \textit{Singleton}, \textit{Prototype}, \textit{Memento}, \textit{Iterator}, \textit{Flyweight} have aspects as object factories. 
% They administrate access to specific object instances. 
% All of them offer factory methods to clients and share a create-on-demand strategy. 
% The patterns are abstracted with code for the factory in the aspect. 
% Participants no longer need to have pattern code in them, the otherwise close coupling between an original object and its representation or accessors is removed from the participants.

% In case of the patterns \textit{Adapter}, \textit{Decorator}, \textit{Strategy}, \textit{Visitor} and \textit{Proxy}, using AspectJ the implementation of some patterns completely disappear because AspectJ language constructs implement them directly. This applies to these patterns in varying degrees. Next, the patterns \textit{Abstract Factory}, \textit{Factory Method}, \textit{Template Method}, \textit{Builder} and \textit{Bridge} can be implemented using multiple inheritance, AspectJ provides a solution to implement multiple inheritance in Java. The patterns \textit{State} and \textit{Interpreter} introduce high coupling between their participants. In the AspectJ implementations, parts of the scattered code can be modularized. Finally, the Facade pattern shows no benefit from AspectJ because there is no significant difference between the AspectJ and the Java implementation of this pattern. 

% As it has been presented above, Hannemann et al. \cite{hannemann2002design} evaluate their solutions by using the following criteria: \textit{Locality}, \textit{Reusability}, \textit{Composition Transparency}, and \textit{(Un)pluggability}.
In general an object or class that is not coupled to its role in a pattern can be used in different contexts without modifications, therefore the reusability of participants can be increased. 
The locality also means that existing classes can be incorporated into a pattern instance without the need to adapt them with extra effort, all the changes are made in the pattern instance. 
This makes the pattern implementations themselves relatively (un)pluggable. 
If we can reuse generalized pattern code and localize the code for a particular pattern instance, this can result in multiple instances of the same pattern in one application  not being easily confused (composition transparency). This solves a common problem with having multiple instances of a design pattern in one application.

%%%%%%%%%%%%%%%%%%%%%%%%%%%%%%%%%%%%%%%%%%%%%%%%%%%%%%%%%%%%%%%%%%%%%%%%%%%%%%%
\subsection{Evolvability issues}\label{Aspect Oriented Programming Evolvability}
Since modularization and separation of concerns makes the evolution of an application a lot easier and \ac{aop} provides mechanisms for modularization and system decomposition, aspect-oriented programs should be easier to be evolved and maintained, but paradoxically they are not \cite{tourwe2003existence}. \ac{aop} technologies deliver applications that are as hard, and sometimes even harder to than was the case before.

According to Tourwe et al. \cite{tourwe2003existence} the problem is that aspects have to include a crosscut description of all places in the application. 
Consequently it is much harder to make such crosscuts unaware to the application and most importantly to the rest of the modules. 
Additionally, current means for specifying concerns rely heavily in the existing structure of the application, therefore the aspects are tightly coupled to the application and of course this affects negatively the evolvability of a system.
As Tourwe et al. \cite{tourwe2003existence} propose a solution for the problem would be the creation of a new more sophisticated crosscut language. 
A language that enables the developer to discriminate between methods based on what they actually do instead of what they look like, in a more intentional way.
This is something that we tried to show in our thesis, a new language that implements \ac{ccc} in a modular way.

%%%%%%%%%%%%%%%%%%%%%%%%%%%%%%%%%%%%%%%%%%%%%%%%%%%%%%%%%%%%%%%%%%%%%%%%%%%%%%%
% Section: Managed Data 
%%%%%%%%%%%%%%%%%%%%%%%%%%%%%%%%%%%%%%%%%%%%%%%%%%%%%%%%%%%%%%%%%%%%%%%%%%%%%%%
\section{Managed Data}\label{Managed Data}
Managed data \cite{loh2012managed} is a data abstraction mechanism that allows the programmer to define the data and their manipulation mechanisms. 
It provides a modular way to control aspects of data.
Managed data is an approach to data abstraction that helps the programmer by giving them control over the structuring mechanism. 
Until now those data structuring mechanisms were predefined form the programming languages and the developers could not take control on the data structuring and management mechanisms, but only to create data of those types.

Managed data provide flexibility and lifts data management up to the application level, by allowing the programmer to build data managers that handle the fundamental data manipulation primitives that are normally hard-coded into the programming language. 

The input to a data manager is a schema, which describes the structure and behavior of the data to be managed. Managed data has three essential components:

\begin{description}
	\item [Data description language,] that describe the desired structure and properties of data.
	\item [Data managers,] that enable creation and manipulation of instances of data.
	\item [Integration,] with a programming language to allow data created and manipulated.
\end{description}

In the traditional approach, the programming language includes a process and data definition mechanism, which are both predefined. 
However, with managed data, the data structuring mechanisms are defined by the programmer by interpretation of data definitions. 
Of course, since a data definition model is also data, it requires a meta-definition mechanism.

%%%%%%%%%%%%%%%%%%%%%%%%%%%%%%%%%%%%%%%%%%%%%%%%%%%%%%%%%%%%%%%%%%%%%%%%%%%%%%%
\subsection{Schemas}\label{Schemas}
The schemas in managed data are the way to describe the structure and the behavior of the data to be managed. 
Schemas can be just a simple data description language which programmers can use describe simple kind of data. 
For example Loh et al. \cite{loh2012managed} used Ruby hash for the data description on a simple example where the hash was an object that represents a mapping from values to values. 
However, a simple schema format like this can not be used to describe itself, because a simple schema is not a record. 
We need a self-describing schema that can be used to describe schemas. 
Self-describing allows schemas to be managed data themselves.
A Schema-Schema is also managed data with its own data manager. This process is called \textit{Bootstrapping} and it is needed in order to jumpstart this process, this extends the benefits of programmable data structuring to their own implementation.
Schemas can be interpreted in many different ways to create different kinds of records based on the manipulation provided by the data managers.

%%%%%%%%%%%%%%%%%%%%%%%%%%%%%%%%%%%%%%%%%%%%%%%%%%%%%%%%%%%%%%%%%%%%%%%%%%%%%%%
\subsection{Data Managers}\label{Data Managers}
Data managers are the mechanisms that interpret \textit{schemas} with defined manipulation strategies set by the programmers. 
Since the schema is only known dynamically, the data managers must be able to determine the fields and methods of the managed data object dynamically as well.
In order to implement such operation we need a meta-programming mechanism that dynamically analyses the structure of a schema and applies the functionality of the data managers to it.
In their implementation Loh et al. \cite{loh2012managed} used the ``missing\_method'' implementation in order to succeed that.
In case of Java we can use reflection and dynamic proxies.

%%%%%%%%%%%%%%%%%%%%%%%%%%%%%%%%%%%%%%%%%%%%%%%%%%%%%%%%%%%%%%%%%%%%%%%%%%%%%%%
% Section: Reflection
%%%%%%%%%%%%%%%%%%%%%%%%%%%%%%%%%%%%%%%%%%%%%%%%%%%%%%%%%%%%%%%%%%%%%%%%%%%%%%%
\section{Java Reflection and Dynamic Proxies}\label{Java Reflection and Dynamic Proxies}
The Java programming language provides the programmer with a Reflection API\footnote{\url{https://docs.oracle.com/javase/tutorial/reflect/}} and with it offers the ability to examine or modify the runtime behavior of applications running in the \ac{jvm}. 
Also, Java comes with an implementation of Dynamic Proxies\footnote{\url{https://docs.oracle.com/javase/8/docs/api/java/lang/reflect/Proxy.html}} which is a class that implements a list of interfaces specified at runtime.

%%%%%%%%%%%%%%%%%%%%%%%%%%%%%%%%%%%%%%%%%%%%%%%%%%%%%%%%%%%%%%%%%%%%%%%%%%%%%%%
\subsection{Reflection}\label{Reflection}

Reflection is the ability of a running program to examine itself and its software environment, and to change what it does depending on what it finds \cite{forman2004java}.

In order for this self-examination to be done, the program needs to have a representation of itself which is called \textit{metadata}. 
In a Object Oriented programming language this metadata is organized into objects, called \textit{metaobjects}. 
Finally, the runtime self-examination of these metaobjects is called \textit{introspection}.

Java supports reflection with it's reflection API since the version 1.1.
Java provides a operations for using metaobjects performing intercession.
Using Java reflection a running program can learn a lot about it self, this information may include: classes (the \texttt{Class} metaobject), class name, class methods, a class super and sub classes, methods (the \texttt{Method} metaobject), method name, method parameters, method type, variables, variables handlers and many more. Querying information from these metaobjects is called introspection.
Additionally to the examining of the these metaobjects, a developer has the ability to dynamically call a method that is discovered at the runtime. Using dynamic invocation, a \texttt{Method} metaobject can be commanded to invoke the method that it represents. 

Although reflection is considered helpful for developing flexible software, it has some pitfalls:

\begin{description}
	
	\item[Security] Since metaobjects give a developer the ability to invoke and change underline data of the program, that gives access to places that are supposed to be secure.

	\item[Code complexity] Consider a program that uses both normal object and metaobjects, that introduces an extra level of complexity since now a developer has to deal with different kind of objects with one on the meta level.

	\item[Runtime performance] Of course the runtime dynamic examination and introspection introduce significant overhead on most language implementations in case of Java's dynamic proxies it has been observed 6.5x overhead \cite{marr2015zero} However, this is not something to consider in this thesis.

\end{description}

%%%%%%%%%%%%%%%%%%%%%%%%%%%%%%%%%%%%%%%%%%%%%%%%%%%%%%%%%%%%%%%%%%%%%%%%%%%%%%%
\subsection{Dynamic Proxies}\label{Dynamic Proxies}
Since the version 1.3 Java supports the concept of \textit{Dynamic Proxies}.
A \textit{proxy} is an object that supports the interface of another object \textit{target}, so that the proxy can substitute for the target for all practical purposes\cite{forman2004java}.
A proxy \textit{proxy} the same interface as the \textit{target} so that it can be used in exactly the same way. 
The proxy \textit{delegates} some or all of the calls that it receives to its target and thus acts as either an intermediary or a substitute.
As a result, a programmer has the capability to add behavior to objects reflectively and dynamically. The Java reflection API contains a dynamic proxy-creation facility, \texttt{java.lang.reflect.Proxy}.

There are several examples of dynamic proxies implementation in Java including \textit{implicit conformance}, \textit{future invocations} \cite{pratikakis2004transparent}, \textit{dynamic multi dispatch}, \textit{design by contract} or \textit{\ac{aop}} \cite{eugster2006uniform}.

\subsubsection{Proxy Objects}
A proxy is an object which conforms to a of interfaces for which that proxy was created. 
The corresponding proxy class extends class \texttt{Proxy} and implements all interfaces.
Thus, conforming to all those interfaces, a proxy can be casted to any of them, and any method defined in those interface can be invoked on the proxy object \cite{eugster2006uniform}.

\subsubsection{Invocation Handlers}
All the proxy objects have an associated object of type \texttt{InvocationHandler}, which handles the method invocations performed on the proxy.

\begin{sourcecode}
	\begin{lstlisting}[language=Java]
public interface InvocationHandler {
	public Object invoke(Object proxy, Method method,Object[] args) throws Throwable;
}		
	\end{lstlisting}
	\caption{The Invocation Handler Interface}
\end{sourcecode}

The arguments of the \texttt{invoke} method include the object on which the method was originally invoked (i.e., the proxy), a the method itself that was invoked on the proxy, and the arguments of that method.
Therefore, the \texttt{invoke} method is capable of handling any method invocation.

\subsubsection{Issues}
A proxy instance is an object, and so it responds to the methods declared by \texttt{java.lang.Object}. Thus an issue is when these methods should be invoked \cite{forman2004java}.

The methods \texttt{equals}, \texttt{hashCode}, and \texttt{toString} inherited by all classes from the \texttt{Object} class and they are handled just like custom methods.
If they are proxied then they are also overridden by proxy classes, and invocations to them are forwarded to the invocation handler of the proxy. 
Other methods defined in Object are not overridden by proxy classes, as they are \texttt{final} \cite{eugster2006uniform}.

\subsubsection{Logging Example}
Previously we mentioned the \textit{logging} \ac{ccc}.
In case of every method invocation of an object has to be logged into the console we need to write logging code on each of the methods of that class.
Hence, this would lead to scattered logging code and tangled code with the method's logic.
This is a problem that can be solved with dynamic proxies, and it's can be seen into the following source code.

\begin{sourcecode}
	\begin{lstlisting}[language=Java]

import java.lang.reflect.*;
import java.io.PrintWriter;

public class TracingIH implements InvocationHandler {
    public static Object createProxy( Object obj, PrintWriter out) {
        return Proxy.newProxyInstance(
            obj.getClass().getClassLoader(),
            obj.getClass().getInterfaces(),
            new TracingIH( obj, out));
    }

    private Object target;
    private PrintWriter out;

    private TracingIH(Object obj, PrintWriter out) {
        target = obj;
        this.out = out;
    }

    public Object invoke(Object proxy, Method method, Object[] args) throws Throwable {
        Object result = null;

        try {
            out.println( method.getName() + "(...) called" );
            result = method.invoke( target, args );
        } catch (InvocationTargetException e) {
            out.println( method.getName() + " throws " + e.getCause() );
            throw e.getCause();
        }
        out.println(method.getName() + " returns" );
        return result;
    }
}	

	\end{lstlisting}
	\caption{An invocation handler for a proxy that traces calls \cite{forman2004java}}
\end{sourcecode}

% TODO: Reread - Recheck

%%%%%%%%%%%%%%%%%%%%%%%%%%%%%%%%%%%%%%%%%%%%%%%%%%%%%%%%%%%%%%%%%%%%%%%%%%%%%%%
% Section: JHotDraw
%%%%%%%%%%%%%%%%%%%%%%%%%%%%%%%%%%%%%%%%%%%%%%%%%%%%%%%%%%%%%%%%%%%%%%%%%%%%%%%
\section{JHotDraw And AJHotDraw}\label{JHotDraw And AJHotDraw}
JHotDraw\footnote{\url{http://www.jhotdraw.org/}} is a Java GUI framework for technical and structured graphics. 
It is an open-source, well-designed and flexible drawing framework of around 18,000 non-comment lines of  Java code. 
JHotDraw's  design relies heavily on some well-known design patterns\cite{gamma1995design} and is a showcase for software quality techniques provided to the \ac{oop} community. 

The fact that JHotDraw is appraised as a so well-designed application makes it an ideal candidate as a showcase for an aspect oriented migration. 
Marin and Moonen \cite{marinajhotdraw} use this showcase for adoption of aspect-oriented techniques in existing systems. 
More specifically, they present AJHotDraw\footnote{\url{http://swerl.tudelft.nl/bin/view/AMR/AJHotDraw}}, which is an aspect-oriented version of JHotDraw developed in AspectJ \ref{AspectJ}. 
The goal of AJHotDraw is to take the existing JHotDraw and migrate it to a functionally equivalent aspect-oriented version. 
The contributions of AJHotDraw include: 

\begin{enumerate}
	\item Identification and documentation of crosscutting concerns in JHotDraw, which helps a lot to aware the developers of the existence of \ac{ccc} in JHotDraw.

	\item Building a common benchmark for testing aspect mining techniques and tools. 

	\item Associating aspect solutions to the identified crosscutting concerns and extra general solutions including design patterns, good practices etc. 

	\item Assessing reliability of proposed aspect solutions to known crosscutting concerns in the context of JHotDraw. 

	\item Building a model application based on an aspect-oriented solution.
	
	\item Providing support for refactoring from an object-oriented solution to an aspect oriented solution.
	
	\item Addressing the challenges risen by testing aspect oriented systems.
\end{enumerate}

The first thing that AJHotDraw developers needed to do was to create a test subproject for the existing JHotDraw (called TestJHotDraw\footnote{\url{http://swerl.tudelft.nl/bin/view/AMR/TestJHotDraw}}) to ensure behavioral equivalence between the original and the refactored solution, since refactoring implies preserving the observable behavior of an application\cite{fowler2009refactoring} and the original JHotDraw comes without tests. 
There were several benefits taken from the aspect-oriented implementation approach\cite{marinajhotdraw}. 
The authors believe that the project will contribute to a gradual and safe adoption of aspect-oriented techniques in existing applications and allow for a better assessment of aspect orientation.

In this thesis we have used JHotDraw and AJHotDraw as well as the test project, TestJHotDraw, in order to evaluate our \ac{ccc} refactoring in managed data. 
The detailed evaluation is described in Chapter \ref{Evaluation}.

%%%%%%%%%%%%%%%%%%%%%%%%%%%%%%%%%%%%%%%%%%%%%%%%%%%%%%%%%%%%%%%%%%%%%%%%%%%%%%%
\subsection{Refactoring of \acrlong{ccc}}\label{Refactoring of ccc}
The refactoring of legacy code to aspect oriented code is known as \textit{Aspect Refactoring}. 
During this process is important to identify which elements are going to be refactored and which \textit{aspect} solutions will replace them. 
To evaluate the refactored elements \cite{fowler2009refactoring}, a testing component for a type is needed in order to ensure behavior conservation, hence some coherent criteria to organize \ac{ccc} are needed. 
Marin, Moonen and Deursen \cite{marin2005approach} organize the \ac{ccc} into types, which are descriptions of similar concerns that share a number of properties: 

\begin{itemize}
	\item A generic behavioral, design or policy requirement to describe the concern within a formalized, consistent context (e.g., role superimposition to modular units (classes), enforced consistent behavior, etc.),

	\item An associated legacy implementation idiom in a given (non-aspect oriented) language (e.g., interface implementations, method calls, etc.)

	\item An associated (desired) aspect language mechanism to support the modularization of the type's concerns (e.g., \textit{pointcut} and \textit{advice}, introduction, composition models).
\end{itemize}

During AJHotDraw implementation\cite{marin2005approach} \cite{hannemann2005role}, the authors propose a type-based refactoring on the same \textit{Observer} instance, \texttt{SelectionListener}, in JHotDraw. 

The legacy code architecture of JHotDraw is displayed in figure \ref{fig:Selection_Listener}.

\begin{figure}[h]
	\centering
  	\fbox{\includegraphics[width=.6\textwidth]{figures/BG_Observer_pattern_Selection_Listener.png}}
  	\caption{Observer pattern: Selection Listener \cite{marin2005approach}}
  	\label{fig:Selection_Listener}
\end{figure}

The \texttt{FigureSelectionListener} interface defines the \textit{Observer} role as its primary concern. 
The interface is implemented by all classes interested in changes of the selection of figures in a drawing view. 
The \texttt{DrawingView} interface partially defines the \textit{Subject} role. 
The \textit{Subject} role is a secondary concern for the \texttt{DrawingView} interface. 
Two classes implement this interface and only one, \texttt{StandardDrawingView}, contains a non-empty implementation of the \textit{Subject} role.

The aspect refactoring would be described as the aspect construct comprises both the \textit{Subject} and \textit{Observer} roles definition, and maintains a list of associations between each \textit{Subject} and its \textit{Observer} objects.
The type-based refactoring\cite{marin2005approach} distinguishes several crosscutting elements that occur in an implementation of the \textit{Observer} pattern: role superimposition, applied twice, for each of the two roles and consistent behavior to notify the observers of the changes in the subject object. 
The \texttt{GenericRole} (empty) interface documents the crosscutting type of role superimposition. 
Specific roles, like \textit{Observer} and \textit{Subject} (\texttt{SelectionSubject}) extend the interface.
These elements are shown in the figure \ref{fig:Concerns_Selection_Listener}.

\begin{figure}[h]
	\centering
  	\fbox{\includegraphics[width=.6\textwidth]{figures/BG_The_concern_types_in_Selection_Listener.png}}
  	\caption{The concern types in Selection Listener \cite{marin2005approach}}
  	\label{fig:Concerns_Selection_Listener}
\end{figure}

%%%%%%%%%%%%%%%%%%%%%%%%%%%%%%%%%%%%%%%%%%%%%%%%%%%%%%%%%%%%%%%%%%%%%%%%%%%%%%%
% \subsubsection{Role-based Refactoring of \acrlong{ccc}}
% \paragraph{Evaluation}
% TODO: Keep it?

%%%%%%%%%%%%%%%%%%%%%%%%%%%%%%%%%%%%%%%%%%%%%%%%%%%%%%%%%%%%%%%%%%%%%%%%%%%%%%%
\subsection{The ``Undo'' Concern of JHotDraw}\label{The Undo Concern of JHotDraw}
After a fan-in analysis of JHotDraw \cite{marin2004identifying}, Marin identified the  ``Undo'' \ac{ccc} in JHotDraw and he presents an approach to the aspect-oriented refactoring of this concern \cite{marin2004refactoring}. 
Marin uses the \textit{(un)pluggability} property of a concern as an estimate of its refactoring cost. 
The author propose the refactoring of the ``Undo'' \ac{ccc} in JHotDraw using AspectJ. 
During the fan-in analysis \cite{marin2004identifying}, the results have shown about 30 undo activities defined for various elements of JHotDraw. 
A representation of the elements in the JHotDraw implementation of Undo concern in figure \ref{fig:Participants_for_undo_in_JHotDraw}.

\begin{figure}[h]
	\centering
  	\fbox{\includegraphics[width=.6\textwidth]{figures/BG_Participants_for_undo_in_JHOTDRAW.png}}
  	\caption{Participants for undo in JHotDraw \cite{marin2004refactoring}}
  	\label{fig:Participants_for_undo_in_JHotDraw}
\end{figure}

The \texttt{Activity} component participate in the implementation of the \textit{Command} design pattern\cite{gamma1995design}. 
Many of these activities have support for undo functionality, which in JHotDraw is implemented by means of nested (undo) classes. 
The nested class knows how to undo the given activity and maintains a list of affected figures whose state is also affected if the activity must be undone. 
Whenever the activity modifies its state, it also updates fields in its associated undo activity to actually perform the undo. 
The \texttt{Undoable Command} object serves three roles: 
\begin{itemize}
	\item it assumes the request to execute the command
	\item then, it delegates the command's execution to the wrapped command 
	\item and last, acquires a reference to the undo activity associated with the wrapped command and it pushes it into a stack managed by an \texttt{UndoManager}. 
	When executing an \texttt{Undo Command}, the top undo activity in the stack is extracted and, after the execution of its \texttt{undo()} method, is pushed into a redo stack managed  by the same \texttt{UndoManager}.
\end{itemize}

Given this implementation, it is obvious that the primary decomposition of \textit{Command} is crosscut by a number of elements as follows: 

\begin{itemize}
	\item the field by \texttt{AbstractCommand} for storing the reference to the associated \texttt{Undo Activity}.

	\item the accessors for this field implemented by the same class.
	\item the \texttt{UndoActivity} nested classes implemented by most of the concrete commands that support undo.

	\item the factory methods for the undo activities declared by each concrete command that can be undone.
	
	\item the references to the before enumerated elements from non-undo related members.
\end{itemize}

In order to refactor this concern in AspectJ, the implementation of AJHotDraw succeeded though the following steps \cite{marin2004refactoring}:

\begin{enumerate}

	\item An undo-dedicated aspect is associated to each of undo-able command. 
	The aspect will implement the entire undo functionality for the given command, while the undo code is removed from the command class.

 	\item Each aspect will consistently be named by appending \texttt{UndoActivity} to the name of its associated command class to enforce the relation between the two.

	\item Next, the command's nested \texttt{UndoActivity} class moves to the aspect. 
	The factory methods for the undo activities also move to the the aspect, from where are introduced back, into the associated command classes, using inter-type declarations.

	\item Finally, the undo setup is attached to those methods from which was previously removed, namely execute() method, by means of an AspectJ \texttt{advice}.

\end{enumerate}

This proposition \cite{marin2004refactoring} provides an easy migration to an aspect-based solution. 
The \ac{ccc} has been identified, then removed from the system, and finally re-added in an aspect-specific manner.

%%%%%%%%%%%%%%%%%%%%%%%%%%%%%%%%%%%%%%%%%%%%%%%%%%%%%%%%%%%%%%%%%%%%%%%%%%%%%%%
\subsubsection{(Un)pluggability of the Undo concern}
In order to evaluate the \ac{ccc} refactoring of ``Undo'' Marin \cite{marin2004refactoring} used the (Un)pluggability property.

The author groups the commands as complexity assessment based on first the degree of \textit{tangling} of the undo setup in the command's logic, particularly the activity's \texttt{execute()}method and second, on the impact of removing the undo-related part from its original site, which can be estimated by the number of references to the factory method and to the methods of the nested undo activity.

Thus, the (un)pluggability property gives a measure of how clear the concern is distinguished in the original code and is a good estimate of the refactoring costs.

%%%%%%%%%%%%%%%%%%%%%%%%%%%%%%%%%%%%%%%%%%%%%%%%%%%%%%%%%%%%%%%%%%%%%%%%%%%%%%%
\subsubsection{AspectJ Drawbacks in the Undo Solution}
By aspect-refactoring, now the two concerns are separated, modularized and the secondary concern of undo is no longer tangled into the implementation of the primary one. 
However, a number of drawbacks that could overcome by a better aspect language support, which can be discussed in relation to this experiment. 
For instance, since AspectJ's mechanisms do not allow introduction of nested classes, the post-refactoring association will only be an indirect one, based on naming conventions (``\texttt{UndoActivity}''). 
This is a weaker connection than the one provided by the original solution. 
Another drawback that is observed is the change of the visibility of the methods introduced from the aspects, for example the inter-type declarations. 
The visibility declared in the aspect refers to the aspect and not to the target class. 

%%%%%%%%%%%%%%%%%%%%%%%%%%%%%%%%%%%%%%%%%%%%%%%%%%%%%%%%%%%%%%%%%%%%%%%%%%%%%%%
% \subsection{The ``Persistence'' Concern of JHotDraw}


% !TEX root = ../thesis.tex

%%%%%%%%%%%%%%%%%%%%%%%%%%%%%%%%%%%%%%%%%%%%%%%%%%%%%%%%%%%
% Chapter: Example Application
%%%%%%%%%%%%%%%%%%%%%%%%%%%%%%%%%%%%%%%%%%%%%%%%%%%%%%%%%%%
\chapter{Example Application: State Machine}\label{Example Application}
In this chapter in order to show how our managed data implementation works in practice, and in particular in terms of aspect refactoring, we present an example showcase.
A detailed demonstration of managed data is presented in Chapter \ref{Implementation}; however, in this chapter we focus on its usage.
The example showcase consists of a very simple state machine application.
A similar example is presented in Enso paper as a showcase for its Object Grammar capabilities \cite{storm2012object}.

Consider the requirements of the state machine as the following: 
\begin{itemize}
	\item A state \texttt{Machine} consists of a number of named \texttt{State} declarations.

	\item Each \texttt{State} contains \texttt{Transitions} to other states, which are identified by a \texttt{name}, when a certain event happens.

	\item A \texttt{Transition} is identified by a certain \texttt{event}.
\end{itemize}

For reasons of simplicity, this example will be a very basic \textit{door} state machine, which includes three states \textbf{Open}, \textbf{Close} and \textbf{Locked}, accompanied by their transitions: \textbf{open\_door}, \textbf{close\_door}, \textbf{lock\_door} and \textbf{unlock\_door} respectively.
Figure \ref{fig:State_machine} illustrates the door state machine.

\begin{figure}[H]
	\centering
  	\fbox{\includegraphics[width=.40\textwidth]{figures/State_machine.png}}
  	\caption{Basic door state machine}
  	\label{fig:State_machine}
\end{figure}

To implement this we need to define the models, interpret the definition given from a list of events and finally add any additional functionality (\textit{concern}) needed, in our case we will implement logging of door's current state.

%%%%%%%%%%%%%%%%%%%%%%%%%%%%%%%%%%%%%%%%%%%%%%%%%%%%%%%%%%%%%
\section{Schemas definition}
As a first step, all the models of the state machine program need to be defined. 
An object diagram is illustrated in Figure \ref{fig:State_machine_object}.

\begin{figure}[H]
	\centering
  	\fbox{\includegraphics[width=.50\textwidth]{figures/State_machine_object_diagram.png}}
  	\caption{Basic door state machine object diagram}
  	\label{fig:State_machine_object}
\end{figure}

In our implementation we define schemas using Java interfaces with a set of meta-data described with Java annotations.
Therefore, as extracted from the requirements we need \texttt{Machine} (Listing \ref{lst:Machine_Schema}), \texttt{State} (Listing \ref{lst:State_Schema}) and \texttt{Transition} (Listing \ref{lst:Transition_Schema}) schemas.

%%%%%%%%%%%%%%%%%%%%%%%%%%%%%%%%%%%%%%%%%%%%%%%%%%%%%%%%%%%
\begin{sourcecode}[H]
	\begin{lstlisting}[language=Java,escapechar=|]
public interface Machine extends M {
	State start(State... startingState);

	State current(State... currentState);

	@Contain
	Set<State> states(State... states);
}
	\end{lstlisting}
	\caption{The Machine Schema}
	\label{lst:Machine_Schema}
\end{sourcecode}

As it can be seen in Listing \ref{lst:Machine_Schema}, the \texttt{Machine} schema definition requires a \texttt{start}ing state, the \texttt{current} state of the machine and a set of \texttt{states} that the machine can be into at each time.
Note that the \texttt{@Contain} annotation suggests that the \texttt{states} field is part of the spine tree and it is not a cross-reference.
This will be further explained in Chapter \ref{Implementation}.

%%%%%%%%%%%%%%%%%%%%%%%%%%%%%%%%%%%%%%%%%%%%%%%%%%%%%%%%%%%%%%%%%%%%%%%%%%%%%%%
\begin{sourcecode}[H]
	\begin{lstlisting}[language=Java,escapechar=|]
public interface State extends M {
	@Key
	String name(String... name);

	@Inverse(other = Machine.class, field = "states")
	Machine machine(Machine... machine);

	@Contain
	Set<Transition> out(Transition... transition);

	@Contain
	Set<Transition> in(Transition... transition);
}
	\end{lstlisting}
	\caption{The State Schema}
	\label{lst:State_Schema}
\end{sourcecode}

For the \texttt{State} definition, Listing \ref{lst:State_Schema}, we need a \texttt{name} field, which represents the name of the state. 
This \texttt{name} field has been annotated with the \texttt{@Key} annotation, which indicates uniqueness. 
The states field of Machine can be indexed by name.
Moreover, the schema includes a set of \texttt{in} and \texttt{out} \texttt{Transition}s.
Since those two fields are of type \texttt{Set}, one field of the \texttt{Transition} schema has to be marked as \textit{key}.
In this case, it is the \texttt{name} field (Line \ref{line:transition_key} Listing \ref{lst:Transition_Schema}).
Finally, the field \texttt{machine} represents the state machine that the state is part of. 
As it can be seen in the schema definition, Listing \ref{lst:State_Schema}, the \texttt{machine} field has been annotated with \texttt{@Inverse}, which indicates that this field is a reference to a field of another schema.
In this case, the \texttt{machine} field of \texttt{State} schema is a reference to \texttt{states} field of \texttt{Machine} schema.

%%%%%%%%%%%%%%%%%%%%%%%%%%%%%%%%%%%%%%%%%%%%%%%%%%%%%%%
\begin{sourcecode}[H]
	\begin{lstlisting}[language=Java,escapechar=|]
public interface Transition extends M {
	@Key 					|\label{line:transition_key}| 
	String event(String... event);

	@Inverse(other = State.class, field = "out")
	State from(State... from);

	@Inverse(other = State.class, field = "in")
	State to(State... to);
}
	\end{lstlisting}
	\caption{The Transition Schema}
	\label{lst:Transition_Schema}
\end{sourcecode}

Finally, in the \texttt{Transition} schema definition, Listing \ref{lst:Transition_Schema}, we need an \texttt{event} that corresponds to the event of the transition and is the \textbf{key} of that schema.
The \texttt{from} and \texttt{to} fields represent the state that the machine changes from and to respectively.
However, these are just references to the \texttt{State} schema (Listing \ref{lst:State_Schema}).
The \textit{parsing} of schemas is performed through a schema loading process presented in Section \ref{sec:Schema Loading}.

%%%%%%%%%%%%%%%%%%%%%%%%%%%%%%%%%%%%%%%%%%%%%%%%%%%%%%%
\section{Factory definition}
Now that we have our schemas, we need a way to build instances of managed objects that these schemas describe. 
In Java to create instances of these schemas as managed data we need to define a factory, which creates managed data instances (managed objects) for each of these schemas \ref{lst:StateMachineFactory}.
Note that the method definitions work as \texttt{Constructors} of managed objects.

\begin{sourcecode}[H]
	\begin{lstlisting}[language=Java,escapechar=|]
public interface StateMachineFactory extends IFactory {
	Machine Machine();  		// constructor for Machine managed objects
	State State(); 				// constructor for State managed objects
	Transition Transition(); 	// constructor for Transition managed objects
}
	\end{lstlisting}
	\caption{The StateMachine Factory}
	\label{lst:StateMachineFactory}
\end{sourcecode}

%%%%%%%%%%%%%%%%%%%%%%%%%%%%%%%%%%%%%%%%%%%%%%%%%%%%%%%
\section{Basic Data Manager}
As mentioned above, in order to interpret and manage the defined schemas of data we need data managers. 
Our framework includes the definition of a \texttt{Basic data manager} that is responsible of interpreting a schema definition to instances of \textit{managed object}s.
Conclusively, in order to make a \textit{managed object}, the data manager needs its schema definition (the interfaces that define the schemas) and the factory (the interface that defines the constructors of the schemas).

%%%%%%%%%%%%%%%%%%%%%%%%%%%%%%%%%%%%%%%%%%%%%%%%%%%%%%%
\subsection{A simple program}
In the case of a simple program without any concerns, we have to use our schemas to define the state machine, parse it, interpret it and finally use it.
The definition of the door state machine is given in Listing \ref{lst:Door_state_machine} in Java.

In practice, after defining our schemas the schema loader parse them and provides us with schema instance. 
Furthermore, the basic data manager provides us with mechanisms that interpret the managed object based on \texttt{stateMachineSchema}, shown in Line \ref{line:state_meaning_full_code}.
The basic data manager also supports the field accessors of those data, namely, the setters and getters of their values.
An basic interpreter for the state machine is shown in Line \ref{line:state_machine_interpreter}.
As it can be seen, the factory is used to create managed objects.
The \textit{setup} of the fields is done automatically by the data manager who is responsible for the managed object interpretation.

\begin{sourcecode}
	\begin{lstlisting}[language=Java, escapechar=|]
public class StateMachineExample {
	public static void main(String[] args) {
		SchemaFactory schemaFactory = ...; // get schemaFactory from framework
		Schema stateMachineSchema = SchemaLoader // load state machine schema
			.load(schemaFactory, Machine.class, State.class, Transition.class); |\label{line:state_schemaMachineSchema}|

		BasicDataManager basicDataManager = new BasicDataManager(); |\label{line:state_meaning_full_code}|
		StateMachineFactory stateMachineFactory = // create factory with data manager
			basicDataManager.factory(StateMachineFactory.class, stateMachineSchema);

		Machine doorStateMachine = stateMachineFactory.Machine(); |\label{line:state_machine_creation_basic}|

		State openState = stateMachineFactory.State(OPEN_STATE);
		openState.machine(doorStateMachine);

		State closedState = stateMachineFactory.State(CLOSED_STATE);
		closedState.machine(doorStateMachine);

		State lockedState = stateMachineFactory.State(LOCKED_STATE);
		lockedState.machine(doorStateMachine);

		Transition closeTransition = stateMachineFactory.Transition(CLOSE_TRANSITION);
		closeTransition.from(openState); closeTransition.to(closedState);

		Transition openTransition = stateMachineFactory.Transition(OPEN_TRANSITION);
		openTransition.from(closedState); openTransition.to(openState);

		Transition lockTransition = stateMachineFactory.Transition(LOCK_TRANSITION);
		lockTransition.from(closedState); lockTransition.to(lockedState);

		Transition unlockTransition = stateMachineFactory.Transition(UNLOCK_TRANSITION);
		unlockTransition.from(lockedState); unlockTransition.to(closedState);

		doorStateMachine.start(closedState);
		interpretStateMachine(doorStateMachine, new LinkedList<>(Arrays.asList(
				LOCK_TRANSITION,
				UNLOCK_TRANSITION,
				OPEN_TRANSITION)));
		}	
	}

	private static void interpretStateMachine(
			Machine stateMachine, List<String> commands) 
	{ |\label{line:state_machine_interpreter}|
	    stateMachine.current(stateMachine.start());
		for (String event : commands) {
			for (Transition trans : stateMachine.current().out()) {
				if (trans.event().equals(event)) {
					stateMachine.current(trans.to());
					break;
				}
			}
	}
}
	\end{lstlisting}
	\caption{Door state machine}
	\label{lst:Door_state_machine}
\end{sourcecode}

%%%%%%%%%%%%%%%%%%%%%%%%%%%%%%%%%%%%%%%%%%%%%%%%%%%%%%%
%%%%%%%%%%%%%%%%%%%%%%%%%%%%%%%%%%%%%%%%%%%%%%%%%%%%%%%
%%%%%%%%%%%%%%%%%%%%%%%%%%%%%%%%%%%%%%%%%%%%%%%%%%%%%%%
\section{Logging crosscutting concern}
Now consider a case where we want to add a crosscutting concern at the previous door state machine implementation.
A simple concern could be \textit{logging}, which would log every change in the ``current'' state of the door state machine.

In order to implement this concern we need a mechanism that continuously observes the changes (transitions) of the machine's \texttt{current} state and reacts accordingly.
Usually, this would lead to scattered logging code in the interpretation method or the models themselves (the machine model).
This is where data managers come to the rescue.
A data manager can implement concerns as modular aspects without scattering code to the components.
The programmer can define a manipulation mechanism of his/her data that includes an aspect of preference.
Therefore, by implementing our concern with a data manager we can keep the component and aspect code separate.

%%%%%%%%%%%%%%%%%%%%%%%%%%%%%%%%%%%%%%%%%%%%%%%%%%%%%%%
\subsection{Observable Data Manager}
Regarding the continuous \textit{observation} of our state machine's ``current'' state changes, we need a data manager that observes these changes in a managed object and executes actions defined by the programmer.
Particularly, it has to observe the \texttt{Machine}'s current \texttt{State} field and perform logging in case this field's value changes.
This data manager creates concrete managed objects as subjects, where observers can be attached in order to be notified of changes and execute an action.
It is important to mention that this new data manager has to inherit the basic one in order to include the basic functionality of schema interpretation and field access.
This leads to a \textbf{stack} of two data managers, each one adding a new aspect of data in a modular way.

In order to define the specifications of our new data manager we first need to define how it is going to be used (its API).
First, we need to attach the \textit{logging} concern to our \texttt{Machine} object. 
This is going to be executed in case the \texttt{current} state changes.
The client code can be seen in Listing \ref{lst:StateMachineMonitoringConcerns}.

\begin{sourcecode} [H]
	\begin{lstlisting}[language=Java, escapechar=|]
ObservableDataManager observableDataManager = new 	ObservableDataManager();
StateMachineFactory stateChangesMachineFactory =
	observableDataManager.factory(StateMachineFactory.class, stateMachineSchema);

Machine doorStateMachine = stateChangesMachineFactory.Machine();

((Observable) doorStateMachine) // Add logging concern on current field changes
	.observe((obj, fieldName, newState) -> { |\label{line:state_machine_monitor}|
		if (fieldName.equals("current")) {
			System.out.println(" > State changed to " + ((State)newState).name());
		}
	});
	\end{lstlisting}
	\caption{Door state machine with logging concern}
	\label{lst:StateMachineMonitoringConcerns}
\end{sourcecode}

%%%%%%%%%%%%%%%%%%%%%%%%%%%%%%%%%%%%%%%%%%%%%%%%%%%%%%%
\subsection{Data Manager Implementation}
Now that we have specified the API of our data manager, we need to implement it.
First, we need to define our specifications.
As it can be seen from Listing \ref{lst:StateMachineMonitoringConcerns}, the \texttt{ObservableDataManager} provides a managed object with the method \texttt{observe}, which adds observers on field changes.

More specifically, this data manager allows to add observers in managed data in the form of a functional interface.
The observe action is shown in Listing \ref{lst:Observe functional interface}.

\begin{sourcecode} [H]
	\begin{lstlisting}[language=Java, escapechar=|]
@FunctionalInterface
public interface Observe {
	void observe(Object obj, String fieldName, Object newValue);
}
	\end{lstlisting}
	\caption{Observe Functional Interface}
	\label{lst:Observe functional interface}
\end{sourcecode}

This functional interface represents an action that is performed when a field changes.
In order to be executed, an action needs to be added in an list of observers.
This specification can be defined in an interface, shown in Listing \ref{lst:Observable interface}.

\begin{sourcecode} [H]
	\begin{lstlisting}[language=Java, escapechar=|]
public interface Observable {
	void observe(Observe _observer);
}
	\end{lstlisting}
	\caption{Observable Interface}
	\label{lst:Observable interface}
\end{sourcecode}

This interface describes the specification for our data manager.
The proxy factory code is shown in Listing \ref{lst:ObservableDataManager} and its MObject implementation in Listing \ref{lst:ObservableMObject}.

\begin{sourcecode} [H]
	\begin{lstlisting}[language=Java, escapechar=|]
public class ObservableDataManager extends BasicDataManager {

    @Override
    public <T extends IFactory> T factory(
    	Class<T> moSchemaFactoryClass, Schema schema, Class<?>... proxiedInterfaces) 
    {
        return super.factory(moSchemaFactoryClass, schema, Observable.class);
    }

    @Override
    protected MObject createManagedObject(Klass klass, Object... _inits) {
        return new ObservableMObject(klass, _inits);
    }
}
	\end{lstlisting}
	\caption{ObservableDataManager - Proxy factory}
	\label{lst:ObservableDataManager}
\end{sourcecode}

Note that the \texttt{ObservableDataManager} adds the \texttt{Observable.class} in its proxy interfaces.
By doing this, the managed object can be ``casted'' as an \texttt{Observable} and adopt the data manager's functionality, as shown in the client code Listing \ref{lst:StateMachineMonitoringConcerns}.

\begin{sourcecode} [H]
	\begin{lstlisting}[language=Java, escapechar=|]
public class ObservableMObject extends MObject implements Observable {
	// a list of observers for that object
	private List<Observe> observers;

	public ObservableMObject(Klass schemaKlass, Object... initializers) {
		super(schemaKlass, initializers);
		observers = new ArrayList<Observe>();
	}

	public void observe(Observe _observer) {
		observers.add(_observer);
	}

	@Override
	public void _set(String _name, Object _value) {
		super._set(_name, _value);
		// Run the observe function for each of the observers on every "set"
		observers.forEach(observer -> observer.observe(thisObject, _name, _value));
	}
}
	\end{lstlisting}
	\caption{ObservableMObject - Invocation Handler}
	\label{lst:ObservableMObject}
\end{sourcecode}

The data manager keeps a list of \texttt{Observe} actions.
The programmer can add actions by using the \texttt{observe} method.
Every time a field's value changes, calling the \texttt{\_set} method of the \texttt{MObject}, the list of the observer actions is executed.
Note that this data manager is general, it does not exclusively work for \texttt{Machine} objects, as in this case, but for all managed objects.

Concluding, it can be observed that the only part that has been changed in the original code is the data manager and the logging concern definition.
The data manager of the \texttt{Machine} managed object has been changed to the new observable data manager.
Additionally, the logging concern has been attached to the machine object very easily simply by using lambdas (Line \ref{line:state_machine_monitor} of Listing \ref{lst:StateMachineMonitoringConcerns}).

By running the program with the commands \texttt{LOCK\_TRANSITION}, \texttt{UNLOCK\_TRANSITION} and \texttt{OPEN\_TRANSITION}, the output is presented in Listing \ref{lst:StateMachineMonitoringConcernsOutput}.
\lstdefinestyle{Bash} {
    backgroundcolor=\color{white},
    basicstyle=\scriptsize\color{black}\ttfamily
}

\begin{sourcecode} [H]
	\lstset{numbers=none}
	\begin{lstlisting}[style=Bash]
> Current state changed to Closed
> Current state changed to Locked
> Current state changed to Open
	\end{lstlisting}
	\caption{Door state machine with logging concern: output}
	\label{lst:StateMachineMonitoringConcernsOutput}
\end{sourcecode}

The basic data manager allows to solely build managed objects, but the observable data manager also provides the functionality of attaching concerns in the managed objects after a specified event.

%%%%%%%%%%%%%%%%%%%%%%%%%%%%%%%%%%%%%%%%%%%%%%%%%%%%%%%
%%%%%%%%%%%%%%%%%%%%%%%%%%%%%%%%%%%%%%%%%%%%%%%%%%%%%%%
\section{Combine crosscutting concern}
One of the main characteristics of managed data is that it allows to define reusable aspects but also to combine them independently.
The concerns are not aware of each other; however, a new data manager can combine them.
An example of this can be shown in Figure \ref{fig:concerns_combination}.

\begin{figure}[H]
	\centering
  	\fbox{\includegraphics[width=1\textwidth]{figures/Example_Class_Diagram.png}}
  	\caption{Combination of logging and immutability concerns}
  	\label{fig:concerns_combination}
\end{figure}

As the figure shows there are two separate concerns implemented in \texttt{MObjects}, namely \texttt{Observable} and \texttt{Lockable}.
Consider that we want to add the \texttt{Lockable} feature (immutability) in our previous door state machine while the \texttt{Observable} functionality still exists.
The client code for this is presented in Listing \ref{lst:StateMachineMonitoringConcernsCombination}.
The \texttt{Lockable} data manager is presented in detail in Section \ref{Implementing a Data Manager}.

\begin{sourcecode} [H]
	\begin{lstlisting}[language=Java, escapechar=|]
// The new data manger that combines two aspects
LockableObservableDataManager dataManager = new 	LockableObservableDataManager();
StateMachineFactory stateChangesMachineFactory =
	dataManager.factory(StateMachineFactory.class, stateMachineSchema);

final Machine doorStateMachine = stateChangesMachineFactory.Machine();

((Observable) doorStateMachine) // Add logging concern on current field changes
	.observe((obj, fieldName, newState) -> {
		if (fieldName.equals("current")) {
			System.out.println(" > State changed to " + ((State)newState).name());
		}
	});
// ...
// It was mutable until now, will not allow changes from now on.
((Lockable) doorStateMachine).lock(); // Add immutability concern 
// ...
	\end{lstlisting}
	\caption{Door state machine with logging and immutability concerns}
	\label{lst:StateMachineMonitoringConcernsCombination}
\end{sourcecode}

The new \texttt{MObject} simply combines the two aspects by including \texttt{LockableMObject} and \texttt{ObservableMObject} instances.
By overriding the methods defined by the \texttt{Lockable} and \texttt{Observable} interfaces and using the \texttt{LockableMObject} and \texttt{ObservableMObject} instances respectively, the new data manager combines the two concerns in a modular way.
Listings \ref{lst:LockableObservableMObject} and \ref{lst:LockableObservableDataManager} show this implementation.

\begin{sourcecode} [H]
	\begin{lstlisting}[language=Java, escapechar=|]
public class LockableObservableMObject 
	extends MObject implements Lockable, Observable 
	{

	private LockableMObject lockableMObject;
	private ObservableMObject observableMObject;

	public LockableObservableMObject(Klass schemaKlass, Object... initializers) {
		super(schemaKlass, initializers);
		lockableMObject = new LockableMObject(schemaKlass, initializers);
		observableMObject = new ObservableMObject(schemaKlass, initializers);
	}

	@Override
	public void observe(Observe _observer) {
		observableMObject.observe(_observer);
	}

	@Override
	public void lock() {
		lockableMObject.lock();
	}

	@Override
	public void _set(String name, Object value) {
		lockableMObject._set(name, value);
		observableMObject._set(name, value);
		super._set(name, value);
	}
}
	\end{lstlisting}
	\caption{LockableObservableMObject}
	\label{lst:LockableObservableMObject}
\end{sourcecode}

\begin{sourcecode} [H]
	\begin{lstlisting}[language=Java, escapechar=|]
public class LockableObservableDataManager extends BasicDataManager {

	@Override
	public <T extends IFactory> T factory(
		Class<T> factoryClass, Schema schema, Class<?>... additionalInterfaces) 
	{
		return super.factory(factoryClass, schema, Lockable.class, Observable.class);
	}

	@Override
	protected MObject createManagedObject(Klass klass, Object... _inits) {
		return new LockableObservableMObject(klass, _inits);
	}
}
	\end{lstlisting}
	\caption{LockableObservableDataManager}
	\label{lst:LockableObservableDataManager}
\end{sourcecode}

Concluding, the example presented a reusable solution of \ac{ccc} without scattering and tangling code in the components.
Additionally, it demonstrates a simple way of combining \ac{ccc} implementation in a modular way.
% !TEX root = ../thesis.tex

%%%%%%%%%%%%%%%%%%%%%%%%%%%%%%%%%%%%%%%%%%%%%%%%%%%%%%%%%%%
% Chapter: Implementation
%%%%%%%%%%%%%%%%%%%%%%%%%%%%%%%%%%%%%%%%%%%%%%%%%%%%%%%%%%%
\chapter{Managed data in Java}\label{Implementation}

As it has already been mentioned, the programming languages include data definition mechanisms that are predefined. 
This makes them unable to define \ac{ccc} without repeating and scattering code through the components \cite{loh2012managed}.
Notably, the problem is that \ac{ccc} are not considered features of the data types, but instead features of data management.
As a result, we implement managed data to allow the developer to define the mechanisms of data manipulation.
This chapter describes our managed data implementation in Java, testing our first research question, which states \textit{``How to implemented managed data in a static language?''}.
It is important to mention that our implementation is inspired by Enso\footnote{\url{https://github.com/enso-lang/enso}}, which is written in Ruby.
Although Ruby is a dynamic language, Enso significantly contributed to our implementation's design.
In this chapter we preset the implementation of managed data in Java, which is available also online as an open-source project called JavaMD (Java Managed Data)\footnote{\url{https://github.com/TheolZacharopoulos/JavaMD}}.

%%%%%%%%%%%%%%%%%%%%%%%%%%%%%%%%%%%%%%%%%%%%%%%%%%%%%%%%%%%
\section{Managed Data Implementation}\label{sec:Managed Data Implementation}
Managed data allows the programmer to handle the fundamental data manipulation mechanisms using \textit{Data Managers}, one of its distinguishing features being modularity.
Using a data description language the programmer defines \textit{Schemas}. 
\textit{Schemas} are the input of \textit{Data Managers}. 
A \textit{Data Manager} in turn interprets the data description language that is used to define the structure and the behavior of the data to be managed.
\textit{Schemas} and \textit{Data Managers} are essential components of managed data, along with \textit{Integration} in the programming language, in our case being Java.

\subsection{Data description with Schemas}\label{Schema Definition}
To create instances of data, we first need to define their structure.
\textit{Schemas} describe the outline structure of our data.
In order to define \textit{Schemas} in managed data we need a data description language that allows to define records as collections of fields.
This language can be anything, e.g. XML, JSON or a different formalism like the one used in Enso.
For our implementation we chose to use \textbf{Java Interfaces} as a data description language to define records of managed data.
By using Java interfaces we use Java's syntax for our definitions.
Moreover, Java interfaces use several conventions to encode semantics, for instance Java annotations, which are very useful for meta data definition on \textit{Schema}s.

As a result, to define a \textit{Schema} we first need to define a set of classes that describe that schema.
A schema \texttt{Klass} \footnote{
	We use the ``Klass'' instead of ``Class'' convention in order to avoid any kind of ambiguities between Java's Class type and our type system. Klass is used to describe our own class type while Class describes Java's native class type.} 
is described by a name and a set of \texttt{Field}s, each of which has a name and a \texttt{Type}.
Since Java interfaces are used to define a \texttt{schemaKlass} we need a way to define \texttt{Field}s for that \texttt{schemaKlass}.
A \texttt{Field} in our data description language can be defined by using \textbf{Java's Method} definition.

Additionally, there are several attributes, considered meta data, that help define the structure of a \texttt{Schema}.
In order to define the meta data in our data description language (interfaces), we use \textit{Java Annotations}.
Annotations are very declarative in the way they express meta data in interfaces and they are consistent with the system (Java).

Thus, to provide a field with meta data, we define annotations in a \textit{Method} target level since a \texttt{Field} is defined by a \textit{Method} declaration Java interfaces.

Note that by using Java interfaces and annotations for our schemas definition, we gain a first level of type checking from \ac{jvm}. 
The reason is that before we run our runtime interpretation of schemas, \ac{jvm} performs type checking in the definitions and in case of wrong types it notifies the programmer.
Additionally, this is beneficial when a programmer uses IDE's that perform real time type inspection\footnote{\url{https://www.jetbrains.com/help/idea/15.0/code-analysis.html}}. 
In those cases errors on the definitions will be spotted immediately. 

The list of the available structure concepts that are supported in our language is presented below \cite{loh2012managed}:
\begin{description}
	\item [@Key] When a method (field definition) is annotated with the \texttt{@Key} annotation that forces its value to be unique within collections of this field's Klass.
	The key should be used on a single field of a Type and its value represents the uniqueness of its Klass's instance.
	Another way to look at this is as a counterpart of the \texttt{hashCode} in traditional Java programs.
	This way when many values of a Klass are in a Set, the key field ensures uniqueness in its context.

	\item [@Inverse] This annotation includes two \textit{annotation element definitions} \footnote{
		\url{https://docs.oracle.com/javase/tutorial/java/annotations/declaring.html}}.
	When a method is annotated with the \texttt{@Inverse(Class other, String field)} annotation, then the inverse \texttt{field} element must be a \texttt{Field}'s name in the \texttt{Class} interface, given by the \texttt{type} element.
	This meta data is used as a reference declaration in schemas, meaning that when a programmer updates the value of a field that is annotated with inverse, then the value of the field that refers to will be also updated.
	This mechanism is interpreted by the managed object and is used for automated \textit{wiring} of the field across a schema.

	\item [@Contain] When a field is annotated with the \texttt{@Contain} annotation, then this field is considered as \textit{traversal}. 
	In general, traversals describe a minimum spanning tree that is called \textit{spine} and ensures reachability of values.
	The spine is used in implementations that need a depth-first search by distinguishing between the actual information and the cross-references of the spanning tree.
	If a spanning tree is defined, then all nodes in a model must be uniquely reachable by following just the spine fields \cite{storm2012object}.
	An example of such functionality is the equivalence between managed objects that is presented in Section \ref{Managed Object equivalence}.
	Sometimes traversal fields describe composition, or ``is a part of'', relationships \cite{loh2012managed}.

	\item [@Optional] When the \texttt{@Optional} annotation is on a field's definition this field can include \texttt{null} values.
	\texttt{Inverse} fields are \texttt{Optional}. 

	\item [Java Inheritance] In addition to the Java annotations, our language uses more Java mechanisms for schemas definition. 
	Java inheritance is one of them. 
	A \texttt{schemaKlass} can extend another Klass (super), which works as the traditional Java inheritance, supporting sub typing mechanisms.
	Implementing this we introduce a \textit{Type Hierarchy} model that includes super and sub classes on managed objects.
	Note that since we use interfaces for \texttt{schemaKlass}, we implicitly support multiple inheritance because a Java interface can extend more than one interfaces.

	\item [Java Collections] Finally, another Java mechanism that we use is the definition of a field that includes many values.
	To define such a field, a programmer has to declare a field's \texttt{Type} as a \texttt{java.util.List} or a \texttt{java.util.Set} of this \texttt{Type}.

\end{description}

Using all the aforementioned constructs of our data definition language, a programmer can define any kind of schema, even itself (see Section \ref{Self-Describing Schemas}).
Schema definition examples are presented in Chapter \ref{Example Application} Listings \ref{lst:Machine_Schema}, \ref{lst:State_Schema} and \ref{lst:Transition_Schema}.
In those definitions the above concepts can be recognized and their meaning can be revealed in context.

%%%%%%%%%%%%%%%%%%%%%%%%%%%%%%%%%%%%%%%%%%%%%%%%%%%%%%%%%%%
\subsection{IFactories}\label{IFactories}
However, even if we have the definitions of schemas, we still need a way to create instances of managed data described by them.
We can not use Java's mechanisms\footnote{\texttt{new} keyword} for this functionality since we need them to be managed data and not ordinary objects.
Thus, we use Java interfaces to define instance factories.
An \texttt{IFactory} is a list of \textit{constructor definitions} for specific schemas.

The methods in this interface are used similarly to the constructors in a Java class, while their implementation is handled by the data managers.
Since those methods are constructors, we can define a constructor with or without initial values.
Unfortunately, we have encountered a limitation regarding constructors with initialization values, making them inappropriate to use in complicated schemas.

%%%%%%%%%%%%%%%%%%%%%%%%%%%%%%%%%%%%%%%%%%%%%%%%%%%%%%%%%%%
\subsubsection{Methods Ordering Issue}\label{Methods ordering}
The problem lays on Java's reflection mechanisms in terms of methods ordering.
More specifically, when the methods of a \texttt{java.lang.Class} are requested by using the \texttt{public Method[] getMethods()} method\footnote{
	As it is mentioned in \url{https://docs.oracle.com/javase/8/docs/api/java/lang/Class.html\#getMethods--}, the elements in the returned array are not sorted and are not in any particular order.}, 
the returned values are not ordered the way as defined in the source code.
Consequently, since the schema definition is reflectively analyzed in the data managers and is dependent on that order, those methods can not be used in the initialization of values.

However, we overcame this difficulty and were able to support this feature in an alternative manner.
In our implementation both the defined methods and the fields are \textbf{alphabetically ordered} by name before being initialized.

That feature can be used by the programmer although it can be confusing.
Therefore, as an advice, we suggest to either provide constructors without initialization values or to write constructors with only \textbf{primitive} initialization values in \textbf{alphabetical order}.
Otherwise we risk getting values in a random order leading to an error or a wrong value assignment.

%%%%%%%%%%%%%%%%%%%%%%%%%%%%%%%%%%%%%%%%%%%%%%%%%%%%%%%%%%%
\subsection{Data Managers Implementation}\label{Data Managers Implementation}
However, the schemas are not a complete managed data specification without a corresponding \texttt{Data Manager}.
A data manager is responsible for interpreting the schema and building virtual objects (managed objects). 
The managed object's fields are defined by the given schema and acts according to the specifications given by the data manager.
Additionally, the data manager ensures that the data given are valid with respect to the schema.
More specifically, the data managers describe how a schema definition is handled from the outside world and what its specifications are.
These properties may include \ac{ccc} that can be described separately by special data managers, separating schema and concern definitions.
Thus, a managed object can have multiple interpretations based on the data manager that is used to interpret it.

A data manager is initialized with a \texttt{Schema} and provides a new \texttt{Managed Object} instance whose properties are defined by that data manager.
Additional to the \texttt{Schema} that includes a Set of \texttt{Type}s (\texttt{Primitive}s or \texttt{Klass}es), it also needs a \texttt{IFactory} that declares the constructors of the given schema \texttt{Klass}.
The data manager through its \texttt{factory} method interprets the schemas and builds new \texttt{IFactories}, which in turn create \texttt{Managed Objects} with the specifications of the data manager.

In the example presented in listing \ref{lst:Basic data Manager Example}, Line \ref{line:basic_data_manager_definition} defines a basic data manager.
This data manager gets the \texttt{IFactory} and the \texttt{Schema} of a state machine as input in the \texttt{factory} method. 
Next, Line \ref{line:basic_data_manager_schema_factory_definition} shows a new \texttt{IFactory} instance is being created, which builds managed objects with the specifications attached from the basic data manager.
Finally, Line \ref{line:basic_data_manager_definition_instance} illustrates how the managed object instances with those specifications can be built.

\begin{sourcecode} [H]
	\begin{lstlisting}[language=Java, escapechar=|]
// Create a basic data manager for state machines
BasicDataManager basicDataManagerForStateMachines = new BasicDataManager(); |\label{line:basic_data_manager_definition}|

// Create a factory that makes managed objects 
// with the specifications of the basic data manager.
StateMachineFactory stateMachineFactory = basicDataManagerForStateMachines
		.factory(StateMachineFactory.class, stateMachineSchema); |\label{line:basic_data_manager_schema_factory_definition}|

// Build an instance of managed object with those specifications.
Machine stateMachineInstance = stateMachineFactory.Machine(); |\label{line:basic_data_manager_definition_instance}|
	\end{lstlisting}
	\caption{Basic data Manager Example}
	\label{lst:Basic data Manager Example}
\end{sourcecode}

\subsubsection{Basic Data Manager}
As described above, we use Java interfaces to define schema Klasses that include fields. 
Those fields are dynamically discovered by a schema loading process and provided to the data manager as schemas.
A data manager has the ability to determine the fields and methods of the managed object during runtime.
In addition, when the data manager adds functionality on a managed object then it first delegates the calls to its specifications and then to the fields of an instance.
In order to dynamically interpret a schema inside a data manager and delegate its functionalities, we used Dynamic Proxies.

In our implementation we have separated the Proxy factory (\texttt{DataManager}) from the Invocation Handler (\texttt{MObject}).
This way, the \texttt{DataManager} class is responsible for creating proxy instances of \texttt{IFactories}. 
The \texttt{IFactory} creates proxy instances with the \texttt{MObject} class instances as invocation handlers.
The \texttt{MObject} instances are responsible for interpreting the schema and delegating actions using their invocation handling implementation. 
Figure \ref{fig:DataManager_and_MObject} illustrates this structure.
As it can be seen the data manager is a \textit{factory} that has a single exposed method, \texttt{factory()}, that is used to build an \texttt{IFactory} instance, which in turn builds \texttt{MObject} instances.

\begin{figure}[H]
	\centering
  	\fbox{\includegraphics[width=.85\textwidth]{figures/DataManager_and_MObject.png}}
  	\caption{Data Manager and MObject}
  	\label{fig:DataManager_and_MObject}
\end{figure}

The two level \textit{proxing} process that the \texttt{DataManager} class performs, from the basic data manager to IFactory and then to MObject, can be seen in Listing \ref{lst:Basic Data Manager}.
Note that the \texttt{createManagedObject} is \texttt{protected} for the sub data managers to override it in order to create MObjects of preference.

\begin{sourcecode} [H]
	\begin{lstlisting}[language=Java, escapechar=|]
public <T extends IFactory> T factory(
	Class<T> factoryClass, Schema schema, Class<?>... proxyInterfaces) 
{
	// add the extra proxied interfaces
	for (Class<?> proxiedInterface : proxyInterfaces) {
		this.addProxiedInterface(proxiedInterface);
	}

	// add the klass interfaces of the schema
	for (Klass klass : schema.klasses()) {
		this.addProxiedInterface(klass.classOf());
	}

	return (T) Proxy.newProxyInstance(
		factoryClass.getClassLoader(),
		new Class<?>[]{factoryClass},
		(proxy, method, args) -> // invocation handler
			createProxiedManagedObject(factoryClass, schema, method, args) // mobject
		);
}

protected Object createProxiedManagedObject(...) {
	//...
	final MObject managedObject = createManagedObject(schemaKlass, inits);
	return Proxy.newProxyInstance(
		schemaFactoryCallingMethodClassLoader,
		proxiedInterfaces.toArray(new Class[proxiedInterfaces.size()]),
		managedObject
	);
}
\end{lstlisting}
	\caption{Basic Data Manager}
	\label{lst:Basic Data Manager}
\end{sourcecode}

\subsubsection{Stacking Data Managers}
In order to create a stack of data managers that combine behavior and specifications, we can use inheritance.
Figure \ref{fig:DataManager_and_MObject} shows how this works.
In detail, \texttt{AnotherDataManager} extends \texttt{BasicDataManager} and simply overrides the \texttt{createManagedObject} and the \texttt{factory} methods.
The \texttt{createManagedObject} method is responsible for creating a new instance of an \texttt{MObject}.
In this case, the \texttt{createManagedObject()} method will create a new \texttt{AnotherMObject} instance.
The \texttt{factory} method is responsible of creating a new \texttt{IFactory} instance.
Note that it is important that the data managers inherit from a base data manager, leading to the modular aspect of the data managers.
As it can be seen, for stacking data managers we used the \textit{Decorator Pattern} \cite{gamma1995design} which is mentioned also in Cook et al. \cite{loh2012managed} as a strategy for static \ac{oop} languages.

%%%%%%%%%%%%%%%%%%%%%%%%%%%%%%%%%%%%%%%%%%%%%%%%%%%%%%%%%%%
\subsection{MObjects}\label{sec:Managed Objects}
The \texttt{MObject}, is an implementation of the \texttt{InvocationHandler} interface.
Thus, the \texttt{MObject}'s \texttt{invoke()} method is called in every field access of the managed object's instance.
To manipulate its fields' values this object has two methods, \texttt{\_set()} and \texttt{\_get()}.
In the implementation of these methods additional checks are performed to ensure the correctness of types and structure of the values.
Therefore, a type checker in the schemaKlass level has been implemented in the particular place.
The setter and getter methods can be overridden from derived \texttt{MObject}s in order to \textit{Decorate} the basic \texttt{MObject} with their functionality. 
Of course they require to call their \texttt{supers} for running the type checker.

The \texttt{MObject} is the \textit{backing object} that stores a reference to the \texttt{schemaKlass} and its implementation represents an instance of that \texttt{schemaKlass}.
That \texttt{schemaKlass} is a meta class that describes the layout of the \texttt{MObject} and keeps the \texttt{Field}s and their \texttt{Types}.
During construction, the fields of the \texttt{MObject} are specified by its \texttt{schemaKlass}.
When a field check has to be performed, the \texttt{MObject} uses its \texttt{schemaKlass}.

Overall, one can easily argue that this class defines the main functionality of managed data.
In particular, this class is the \textit{interpreter} of managed data.
Therefore, it is responsible for handling the calls to methods (invocation handler), invoke default methods, setup and initialize field values, based on its \texttt{schemaKlass}, and internally perform the \textit{type checking}.
A detailed presentation of this class and its action is explanation in Appendix \ref{apdx:MObject}.

%%%%%%%%%%%%%%%%%%%%%%%%%%%%%%%%%%%%%%%%%%%%%%%%%%%%%%%%%%%%
%%%%%%%%%%%%%%%%%%%%%%%%%%%%%%%%%%%%%%%%%%%%%%%%%%%%%%%%%%%%
\subsection{Implementing a Data Manager}\label{Implementing a Data Manager}
The implementation and the integration of a new data manager is straight forward in our framework.
As it can be seen in Figure \ref{fig:DataManager_and_MObject}, the basic components of a new data manager implementation are the \texttt{Data Manager} class (proxy) and the \texttt{MObject} class (invocation handler).

First, to follow the modularity aspect and the ability to stack data managers together combining their specifications, we need to inherit from, at least, the \texttt{BasicDataManager} and its \texttt{MObject} respectively.
A simple data manager that could be useful is a data manager that introduces immutability to its managed objects.
A \texttt{Lockable} data manager should first inherit the \texttt{BasicDataManager} to get its field access specification.
The implementation of the \texttt{LockableDataManager} is illustrated in \ref{lst:Lockable Data Manager}.

\begin{sourcecode} [H]
	\begin{lstlisting}[language=Java, escapechar=|]
public class LockableDataManager extends BasicDataManager {

	@Override
    public <T extends IFactory> T factory(
    	Class<T> factoryClass, Schema schema, Class<?>... proxyInterfaces) {
        // Add the Lockable class in order to use it in the managed object.
        return super.factory(factoryClass, schema, Lockable.class);
    }

	@Override
	protected MObject createManagedObject(Klass klass, Object... _inits) {
		return new LockableMObject(klass, _inits);
	}
}
	\end{lstlisting}
	\caption{Lockable Data Manager}
	\label{lst:Lockable Data Manager}
\end{sourcecode}

Additionally, it should add some \textit{locking} mechanism to ensure immutability of its objects.
This is defined in the \texttt{Lockable} interface, which is responsible of ensuring the implementation of the specifications. 
Listing \ref{lst:Lockable Interface} shows the specifications of the interface.

\begin{sourcecode} [H]
	\begin{lstlisting}[language=Java, escapechar=|]
public interface Lockable {
	void lock();
}
	\end{lstlisting}
	\caption{Lockable Interface}
	\label{lst:Lockable Interface}
\end{sourcecode}

Since we have the specifications and the data manager that creates the \textit{Lockable} managed object, we still need the implementation.
The implementation is located in the \texttt{MObject} and in this case the \texttt{LockableMObject}, 
Listing \ref{lst:Lockable Managed Object}.

\begin{sourcecode} [H]
	\begin{lstlisting}[language=Java, escapechar=|]
public class LockableMObject extends MObject implements Lockable {
	private boolean isLocked = false;

	public LockableMObject(Klass schemaKlass, Object... initializers) {
		super(schemaKlass, initializers);
	}

	public void lock() {
		isLocked = true;
	}

	@Override
	public void _set(String name, Object value) 
	throws NoSuchFieldError, InvalidFieldValueException, NoKeyFieldException {
		if (isLocked)
	    	throw new IllegalAccessError(
	    		"Cannot change " + name + " of locked object " + schemaKlass.name() + ".");
		super._set(name, value);
	}
}
	\end{lstlisting}
	\caption{Lockable Managed Object}
	\label{lst:Lockable Managed Object}
\end{sourcecode}

The \texttt{LockableMObject}, by extending the \texttt{MObject} and implementing the \texttt{Lockable} interface, inherits the basic functionality of a managed object and gets a specification description respectively.
Its role is to implement the logic of the immutability, which is as simple as it looks.
In order to use this functionality, one needs to create managed objects using this data manager.
An example is shown in Listing \ref{lst:Immutability Example}.

\begin{sourcecode} [H]
	\begin{lstlisting}[language=Java, escapechar=|]
LockableDataManager lockableFactory = new LockableDataManager();
PointFactory lockablePointFactory = 
	lockableFactory.factory(PointFactory.class, pointSchema);
Point2D lockablePoint = lockablePointFactory.Point2D(1, 2);

// It was mutable until now, now it is locked (immutable).
((Lockable)lockablePoint).lock();
try {
	lockablePoint.x(2); // Should throw here since its immutable.
} catch (IllegalAccessError e) {
	System.err.println("IllegalAccessError: " + e.getMessage());
}
	\end{lstlisting}
	\caption{Immutability Example}
	\label{lst:Immutability Example}
\end{sourcecode}

%%%%%%%%%%%%%%%%%%%%%%%%%%%%%%%%%%%%%%%%%%%%%%%%%%%%%%%%%%%
\section{Self-Describing Schemas}\label{Self-Describing Schemas}
As explained by Cook et al. \cite{loh2012managed}, a self-describing schema is a schema that can be used to define schemas, including itself.
Our framework is fully self-described, the schemas are also described by schemas which are both models \cite{kurtev2006model}. 
To allow schemas to be managed data we need a ``self-describing schema mechanism'' or \textit{SchemaSchema}.
Through the \textit{SchemaSchema} the approach of managed data can be applied at the meta level as well.

The reason that a self-describing schema is important is because schema schemas can be used from factories (IFactory) to create schemas.
The schema of schemas is just a schema that allows the creation of schemas, including its own schema \cite{storm2012object}.
Additionally, by presenting the schema as the first-class model\cite{kurtev2006model}, they can be extended in the same way just like ordinary models.

\subsection{SchemaSchema}\label{sec:SchemaSchema}
By using Java interfaces the \textit{Schema} classes are tightly coupled structurally to the Java interfaces used to define them.
Since we want to decouple from Java interfaces and reflection we need our own \textit{Klass system}.
In order to be self-describing we want this Klass system to be also represented as managed data. 
To model the structure of a \texttt{Schema} itself we need to be able to describe a class as a collection of \texttt{Fields}, each of which has a \texttt{name} and a \texttt{Type} \cite{loh2012managed}. 
Thus, for our \textit{SchemaSchema} definition we need a \texttt{Type}, a \texttt{Field} and a \texttt{Schema} as a collection of \texttt{Type}s. 
A \texttt{Type} could be both a \texttt{Primitive}, without \texttt{Fields}, and a \texttt{Klass}, with a set of \texttt{Fields}.
Additionally, those \texttt{Fields} may have some extra meta data attributes that are explained in Section \ref{Schema Definition}.

A schema like this can describe itself since every concept used in the explanation is de facto included in the definition.
For a self-describing implementation we need to describe our own SchemaSchema. 

Figure \ref{fig:SchemaSchema_definition} illustrates the modeling of this definition.

\begin{figure}[H]
	\centering
  	\fbox{\includegraphics[width=.85\textwidth]{figures/SchemaSchema_definition.png}}
  	\caption{The schema of schemas}
  	\label{fig:SchemaSchema_definition}
\end{figure}

\subsection{SchemaFactory}\label{sec:SchemaFactory}
Considering that we have the schema of our schema (\textit{SchemaSchema}) we need a way to create instances of those \textit{schemaSchemaKlasses}.
In this case, as we do with the normal schemas, we use an IFactory definition.
However, this time it is a \textit{SchemaFactory} that defines constructors of all the schema klasses that are needed to describe our \textit{SchemaSchema}.
Listing \ref{lst:SchemaFactory} shows its definition.

\begin{sourcecode} [H]
	\begin{lstlisting}[language=Java, escapechar=|]
public interface SchemaFactory extends IFactory {
    Schema Schema();
    Primitive Primitive();
    Klass Klass();
    Field Field();
    Field Field(
    	Boolean contain, Boolean key, Boolean many, String name, Boolean optional);
}
	\end{lstlisting}
	\caption{SchemaFactory}
	\label{lst:SchemaFactory}
\end{sourcecode}

%%%%%%%%%%%%%%%%%%%%%%%%%%%%%%%%%%%%%%%%%%%%%%%%%%%%%%%%%%%
\subsection{Schema Loading}\label{sec:Schema Loading}
To construct the Klass system we need to analyze the Java interfaces using reflection and then build actual instances of the Schema, Klass, Field etc. using the appropriate factory.
The \texttt{SchemaLoader} is responsible of this process.

\texttt{SchemaLoader}'s \texttt{load} static method takes as input a Set of interfaces, which are the schema definitions, a \texttt{SchemaFactory} that includes constructor definitions of the \texttt{SchemaSchema} and returns a new instance of \texttt{Schema}.
During the reflective analysis of the input interfaces the \texttt{SchemaLoader} builds the corresponding \texttt{Types} and \texttt{Fields} of those interfaces using the \texttt{SchemaFactory}.
A \texttt{Schema} consists of the Set of these \texttt{Types}.
An example taken from Chapter \ref{Example Application}, is shown in Listing \ref{lst:SchemaLoader Example}.

\begin{sourcecode} [H]
	\begin{lstlisting}[language=Java, escapechar=|]
Schema schemaSchema = ...;
SchemaFactory sf = basicFactory.factory(SchemaFactory.class, schemaSchema);

Schema stateMachineSchema = SchemaLoader.load(
	sf, Machine.class, State.class, Transition.class);
	\end{lstlisting}
	\caption{SchemaLoader Example}
	\label{lst:SchemaLoader Example}
\end{sourcecode}

In its implementation, the \texttt{SchemaLoader} gets as input a \texttt{SchemaFactory} and a set of interfaces that describe the state machine schema.
Next, it returns a new instance of the state machine Schema.
This schema consists of a set of schema Klasses that are described by interfaces, namely \texttt{Machine.class}, \texttt{State.class} and \texttt{Transition.class}.
Next, the \texttt{SchemaLoader} analyzes the definition of those schemas using reflection and then makes a \texttt{Schema} by using the \texttt{SchemaFactory} that it has been given.
A more detailed description of this process is given in Appendix \ref{appdx:SchemaLoading}.
% #important
In general \texttt{SchemaLoader} can be seen as our parser, which accepts a description of language (the interfaces) and a method create objects of its components (the schema factory).

%%%%%%%%%%%%%%%%%%%%%%%%%%%%%%%%%%%%%%%%%%%%%%%%%%%%%%%%%%%
\section{Bootstrapping}\label{sec:Bootstrapping}
Considering that SchemaSchema is managed data itself, we can use the SchemaLoader to build a new SchemaSchema.
Nonetheless, we need a description of that SchemaSchema, which will be used during the loading process to build the schema Klasses.
As a result, we need a \textit{Bootstrap Schema} to jumpstart this process.
The \textit{Bootstrap Schema} is exclusively self-describing, as it must manage itself \cite{loh2012managed}, and hardcoded in its own class, \texttt{BootSchema}.

\subsection{Cutting the umbilical cord}\label{subsec:Cutting the umbilical cord}
Having a \texttt{BootSchema} in place we can now create ``real'' \texttt{SchemaSchema}s \footnote{
	We call them real because they are managed data and not hard-coded.}.
For consistency, we use those ``real'' \texttt{SchemaSchema}s in order to build other schemas, this way everything is managed data.
After building a real SchemaSchema we no longer need the \texttt{BootSchema}, which leads to a process that we call ``Cutting the umbilical cord''.
An example of ``Cutting the umbilical cord'' is shown in Listing \ref{subsec:Cutting the umbilical cord}, where we use the \texttt{BootSchema} to build the \texttt{realSchemaSchema} and then we use this \texttt{realSchemaSchema} to build another \texttt{realSchemaSchema} (\texttt{realSchemaSchema2}).

\begin{sourcecode} [H]
	\begin{lstlisting}[language=Java, escapechar=|]
final BasicDataManager basicFactory = new BasicDataManager();
final SchemaFactory schemaFactory = 
	basicFactory.factory(SchemaFactory.class, new BootSchema());
final Schema realSchemaSchema = SchemaLoader.load(
        	schemaFactory,
         	Schema.class, Type.class, Primitive.class, Klass.class, Field.class,
        	Primitives.class); |\label{line:Primitives}|

final BasicDataManager basicFactory2 = new BasicDataManager();
final SchemaFactory schemaFactory2 = 
	basicFactory2.factory(SchemaFactory.class, realSchemaSchema);
final Schema realSchemaSchema2 = SchemaLoader.load(
        	schemaFactory2, 
        	Schema.class, Type.class,  Primitive.class, Klass.class, Field.class);
	\end{lstlisting}
	\caption{Cutting the umbilical cord}
	\label{lst:Cutting the umbilical cord}
\end{sourcecode}

Figure \ref{fig:schema_schema_models} illustrates the models during a bootstrapping process.
As it can be seen, the \texttt{BootSchema} is used in order to describe the Schema Schema, making the Schema Schema independent and managed data itself.
Thus, it can be used to create other schemas like the Machine schema or even itself.

\begin{figure}[H]
	\centering
  	\fbox{\includegraphics[width=.4\textwidth]{figures/schema_schema_models.png}}
  	\caption{Boot Schema models}
  	\label{fig:schema_schema_models}
\end{figure}

\subsection{Primitives Definition}\label{Primitives Definition}
Since the Bootstrap Schema defines the primitive types for its description, the real schema schema needs a way to include them as well.
These initial Java primitives supported in our implementation are shown in Table \ref{tbl:primivites_table}.

\begin{table}[H]
	\centering
	\begin{tabular}{@{}lccc@{}}
	\toprule
	                 & \textbf{Class} & \textbf{Name} & \textbf{Default Value} \\ \midrule
	\textbf{Integer} & Integer.class  & ``Integer''   & 0                      \\
	\textbf{int}     & int.class      & ``int''       & 0                      \\
	\textbf{Boolean} & Boolean.class  & ``Boolean''   & false                  \\
	\textbf{boolean} & boolean.class  & ``boolean''   & false                  \\
	\textbf{String}  & String.class   & ``String''    & ``''                   \\
	\textbf{Double}  & Double.class   & ``Double''    & 0.                     \\
	\textbf{Float}   & Float.class    & ``Float''     & 0.f                    \\
	\textbf{Class}   & Class.class    & ``Class''     & null                   \\
	\textbf{Object}  & Object.class   & ``Object''    & null                   \\ \bottomrule
	\end{tabular}
	\caption{Primitives Table}
	\label{tbl:primivites_table}
\end{table}

To define those primitives we use an interface called \textit{Primitives}, introduced during the loading of the real schema, as seen in Line \ref{line:Primitives} of Listing \ref{lst:Cutting the umbilical cord}.
The definition of this interface is shown in Listing \ref{Primitives Definition} which is a simple \texttt{Class/Name} mapping \footnote{We use the ``\_'' prefix convention in order to define names of primitives that are reserved words in Java.}. 

\begin{sourcecode} [H]
	\begin{lstlisting}[language=Java, escapechar=|]
public interface Primitives {
	Integer Integer();
	int _int();
	Boolean Boolean();
	boolean _boolean();
	String String();
	Class Class();
	Float Float();
	Double Double();
}
	\end{lstlisting}
	\caption{Primitives Definition}
	\label{lst:Primitives Definition}
\end{sourcecode}

The benefits of such a definition is that the \texttt{Primitives} interface is extensible.
By extending it one can add more primitives in the schema as long as it is introduced during the schema loading.

%%%%%%%%%%%%%%%%%%%%%%%%%%%%%%%%%%%%%%%%%%%%%%%%%%%%%%%%%%%%
\section{Implementation Issues}\label{Implementation Issues}
The fact that we use Java reflection and dynamic proxies, along with the fact that everything is managed data, even the schemaSchema, introduces some issues, including the methods ordering problem described in Section \ref{Methods ordering}.

\subsection{Equivalence}\label{Managed Object equivalence}
The \texttt{bootstrapSchema}, \texttt{realSchemaSchema} and \texttt{realSchemaSchema2} managed objects from the Listing \ref{subsec:Cutting the umbilical cord} should be equal because they ultimately describe the same \textit{Schema}.

However, since, apart from the \texttt{bootstrapSchema}, they are managed data and not normal Java objects, we need a way to check for equality on managed objects.
We have implemented the equivalence functionality for managed objects, using the \textit{Equality Checking for Trees and Graphs
algorithm} by Michael D. Adams and R. Kent Dybvig \cite{adams2008efficient}.

\subsection{The classOf field}\label{The classOf field}
As it has be presented in Section \ref{Dynamic Proxies}, for a proxy object to conform with interfaces and be casted to any of them, it needs these interfaces during its initialization.
To support that, we have added the \texttt{classOf} field in the \texttt{Type} schema Klass, which is of type \texttt{java.lang.Class} and is a reference of the Java class that this schema Klass is described to.

\subsection{Hash-code of Managed Objects}\label{Hashcode of Managed Objects}
To avoid any unpredictable activities that a \texttt{hashCode} invocation would bring in managed objects, we have omitted it. 
We do not depend on the ordinary \texttt{hashCode} for managed objects, we do not call it and therefore we have not implemented it.
If it is a collection field type, then the field has to have a \texttt{Key} field. 
In this case, we obtain the value of the key field and index it into a \texttt{HashMap}. 

Using the \texttt{Key} field as the key of the hashmap works whether it is a primitive or not since we get the \texttt{Object.hashCode()} of that key.
However, that suggests that the key is not of our schema Klass system but a Java type.
Finally, the \texttt{MObject} invocation handler delegates the call of the \texttt{hashCode} method to the real object so that it would never fail, although this is not suggested because it may lead to unpredictable results.

\subsection{Java 8 Default Methods}\label{Java 8 Default Methods}
Java 8 supports the definition of default methods in interfaces.
According to the specification\footnote{\url{https://docs.oracle.com/javase/tutorial/java/IandI/defaultmethods.html}}, default methods enable the programmer to add new functionalities to the interfaces and can be used as method implementation in abstract classes.
We use Java 8 default methods in order to add functionality to our schema definitions. 
In particular, methods that are defined as \textit{default} are ignored during the interpretation and no fields are created for them.
We consider this as a helpful mechanism for defining functionality inside the schemas.
A notable feature is that the default method invocation in the MObject is \texttt{protected}, which makes it possible for the derived data managers to ``monitor'' when a default method is invoked.

\section{Benefits and Limitations}\label{Benefits and Limitations}
One of the advantages of this language is the simplicity of its usage. 
A programmer simply needs to define the schemas, followed by the data managers, and can easily write a program using them.
The language takes care of the dependencies, references and any other underline mechanisms.
Moreover, it uses Java concepts, which makes it safer in terms of type checking and definitions making it easier for Java developers to adapt.
Furthermore, by being a self-describing language it is no longer bounded to the Java constructs transforming everything into managed data.
Finally, the effortless mechanism of stacking data managers makes it significantly modular on every level, meta or not.

However, in addition to the implementation issues described in the previous section, there are significant performance implications since we use Java reflection and dynamic proxies to dynamically interpret the schemas. 
This makes it unfavorable for applications that focus on performance and are based on \ac{jvm} optimizations.

Another issue that arises is that integration in existing systems is complicated considering every model has to be redefined as a schema and every functionality has to be reimplemented in data managers.
However, an existing system integration is presented in Chapter \ref{AspectRefactoring}.

% \section{Claims}\label{Implementation Claims}
% We claim that managed data leads to a powerful data abstraction that gives the programmer control over fundamental mechanisms of data creation and manipulation \cite{loh2012managed}.
% Those mechanisms are traditionally predefined by the programming languages. 
% Managed data gives control over them by using data managers.
% Moreover, we claim that managed data introduces a modular way to define data and aspects of data. 
% In Chapter \ref{AspectRefactoring} we present how to \textit{aspect refactor} an existing application using managed data.

% !TEX root = ../thesis.tex

%%%%%%%%%%%%%%%%%%%%%%%%%%%%%%%%%%%%%%%%%%%%%%%%%%%%%%%%%%%%%%%%%%%%%%%%%%%%%%%
% Chapter: Aspect refactoring 
%%%%%%%%%%%%%%%%%%%%%%%%%%%%%%%%%%%%%%%%%%%%%%%%%%%%%%%%%%%%%%%%%%%%%%%%%%%%%%%
\chapter{Taming Aspects of JHotDraw with managed data}\label{AspectRefactoring}

%%%%%%%%%%%%%%%%%%%%%%%%%%%%%%%%%%%%%%%%%%%%%%%%%%%%%%%%%%%%%%%%%%%%%%%%%%%%%%%
\section{Crosscutting Concerns Identification}
Our managed data framework addresses the problem of \ac{ccc} by capturing them in modular data managers.
Yet, to solve the problem of \ac{ccc} it is first required to identify them in the source code.
This leads to a process called \textit{aspect mining}.
\textit{Aspect mining} is a reverse engineering process that aims at finding \ac{ccc} in existing systems \cite{marin2004identifying}.
The aspect mining topic has been addressed in previous research that include methods such as clone detection \cite{bruntink2005use}, machine learning \cite{shepherd2004design}, IDE tools \cite{robillard2002concern} and more. 
Marin et al. \cite{marin2004identifying} introduced a technique constructed by spotting methods that are invoked from many different places (high fan-in), in order to identify candidate aspects in open-source Java systems.
One of these projects include the JHotDraw.
In this thesis we focused on their concern findings in refactoring JHotDraw.
In particular, we focused on the \textit{FigureSelection}, concern, which is an observer pattern implementation.
% TODO: Undo
% In particular, we focused on two main concerns, the \textit{FigureSelection}, which is an observer pattern implementation and the \textit{Undo} concern that is part of the command pattern.

\section{Aspect Refactoring in Managed Data}
In order to evaluate the ability of managed data to tame aspects, we have refactored the aforementioned concerns of JHotDraw.
More specifically, in this chapter we present the refactoring of the \textit{FigureSelectionListener} observer pattern.
% TODO: Undo
% More specifically, in this chapter we present the refactoring of the \textit{FigureSelectionListener} observer pattern as well as the \textit{Undo} concern.
The choice of those concerns has been made on purpose, since those are the concerns that AJHotDraw refactors using AspectJ and \ac{aop} techniques.
For the refactoring we used our implementation of managed data in Java, presented in the previous chapter.
Therefore, by having three versions of the same application (JHotDraw) and by solving the same concerns we will be able to perform a comparative evaluation.
The three systems included in our assessment are: \textbf{JHotDraw}\footnote{\url{http://www.jhotdraw.org/}}, the original \ac{oop} version, \textbf{AJHotDraw}\footnote{\url{https://sourceforge.net/projects/ajhotdraw/}}, the \ac{aop} refactored version and our \textbf{ManagedDataJHotDraw}\footnote{\url{https://github.com/TheolZacharopoulos/ManagedDataJHotDraw}}, the managed data refactored version.
We focused on those concerns because they were also identified, solved, analyzed and presented in AJHotDraw.
Note that, for compatibility and comparison reliance, we used the version \textit{JHotDraw v.5.4b1} since AJHotDraw also refactors the same version.

In order to refactor JHotDraw, we first had to migrate it in managed data.
The result of this migration is available on an open-source project, the ManagedDataJHotDraw.
We claim that this is the first aspect refactoring of an application using managed data to date, since this project aims on showing how managed data can deal with \ac{ccc} in existing systems.

\section{Migration Process}
The refactoring of an application of JHotDraw's size required a significant amount of time to study and familiarize with, yet, its well-designed \ac{oop} code, made it easy to grasp.
We solely focused on the parts that were going to be refactored, based on refactorings that AJHotDraw developers \cite{marinajhotdraw} performed.
Thanks to their fan-in analysis \cite{marin2004identifying}, we targeted the same concerns in order to make a fair comparison.
Furthermore, during the implementation of ManagedDataJHotDraw we focused on maintaining behavioral coherence and the original design.

\subsection{DrawingView}
One of the main components of JHotDraw is the \textit{DrawingView} interface.
As Figure \ref{fig:JHotDraw_DrawingView} illustrates, the \textit{DrawingView} is responsible for rendering \texttt{Drawings} and listening to its changes.
Additionally, it is responsible for receiving the user input and delegating it to the current tool.

\begin{figure}[H]
	\centering
  	\fbox{\includegraphics[width=.8\textwidth]{figures/JHotDraw_DrawingView.png}}
  	\caption{DrawingView of JHotDraw}
  	\label{fig:JHotDraw_DrawingView}
\end{figure}

Conclusively, \texttt{DrawingView} makes a good candidate for managed data migration.
The reason is that the specifications of that class can be implemented in data managers and dynamically added to it.

\subsection{Managed Data DrawingView}
To support sub-typing on the \texttt{DrawingView} interface, we have implemented the \texttt{MDDrawingView}, namely Managed Data DrawingView, which replaced the \texttt{DrawingView} in JHotDraw.
Having this interface for super type, we still needed the actual managed data schemas.
As Figure \ref{fig:JHotDraw_DrawingView} shows, there are two implementations of the \texttt{DrawingView}.
In particular, the \texttt{StandardDrawingView}, which is the implementation that is used when a new drawing view is created in the application and the \texttt{NullDrawingView}, which represents a null drawing view as for the \textit{null-object} pattern.

Following their original design, we have implemented two schemas, one for the \texttt{StandardDrawingView} and one for the \texttt{NullDrawingView}, namely \texttt{MDStandardDrawingView} and \texttt{MDNullDrawingView} respectively.
The instances of those schemas have been used in the same way their counterparts are used in JHotDraw.
A snippet of the \texttt{MDStandardDrawingView} is shown in Listing \ref{lst:MDStandardDrawingView schema} \footnote{Most of the implementation has been omitted for brevity.}.

\begin{sourcecode}[H]
	\begin{lstlisting}[language=Java, escapechar=|]
public interface MDStandardDrawingView extends M, MDDrawingView { |\label{line:MDStandardDrawingView extends M, MDDrawingView}|
	...
	// Composition over inheritance, the original inherits the JPanel
	JPanel panel(JPanel... panel); |\label{line:jpanel composition}|

	default JPanel getPanel() { |\label{line:jpanel getter}|
	    return panel();
	}

	default void setPanel(JPanel _panel) {|\label{line:jpanel setter}|
	    panel(_panel);
	}
	...
	Rectangle damage(Rectangle... damage);
	Drawing drawing(Drawing... drawing);
	...
	default FigureEnumeration selectionZOrdered() { |\label{line:selectionZOrdered}|
		List result = CollectionsFactory.current().createList(selectionCount());
		FigureEnumeration figures = drawing().figures();

		while (figures.hasNextFigure()) {
			Figure f= figures.nextFigure();
			if (isFigureSelected(f)) {
				result.add(f);
			}
		}
		return new ReverseFigureEnumerator(result);
	}
	...
	default void repairDamage() { |\label{line:repairDamage}|
		if (getDamage() != null) {
			panel().repaint(damage().x, damage().y, damage().width, damage().height);
			setDamage(null);
		}
	}
	...
}
	\end{lstlisting}
	\caption{MDStandardDrawingView schema}
	\label{lst:MDStandardDrawingView schema}
\end{sourcecode}

Listing \ref{lst:MDStandardDrawingView schema} shows that the \texttt{MDStandardDrawingView} interface extends both \texttt{M} interface, defining that this is a schema definition, and \texttt{MDDrawingView}, for sub-type support.
Additionally, all the functionalities implemented in methods of the original \texttt{DrawingView}, in managed data they are implemented in default methods of the schema interface.
The fields of a schema can provide those methods with the managed object's current state.
As Lines \ref{line:selectionZOrdered} and \ref{line:repairDamage} show, the fields of the schema can be used to query their values inside the default methods.
Note that the code in the default methods is identical to the original \texttt{DrawingView}.
Furthermore, for consistency with the legacy code, we have implemented setters and getters, Lines \ref{line:jpanel setter} and \ref{line:jpanel getter}, for field values accessors.
This way we maintained consistency across in accessing values of the managed object.

A notable issue is that the original \texttt{StandardDrawingView} extends the \texttt{javax.swing.jpanel} class as Figure \ref{fig:JHotDraw_DrawingView} shows.
However, such a structure is not supported in managed data. 
Schema definitions can not extend classes.
To overcome this issue we defined the \texttt{JPanel} as a field in the schema, namely \textit{panel}.
To support the \texttt{JPanel} as a type of a field though, it is needs ti be defined as managed data.
By all means, the same holds for the remaining fields, such as \texttt{Rectangle} and \texttt{Drawing}.

As explained in Section \ref{Primitives Definition}, our framework provides external primitives definition by inheriting the \texttt{Primitives} interface.
The JHotDraw primitives definition is shown in Listing \ref{lst:JHotDraw Primitives Definition}.

\begin{sourcecode}[H]
	\begin{lstlisting}[language=Java, escapechar=|]
public interface JHotDrawPrimitives extends Primitives {
	javax.swing.JPanel JPanel();

	java.awt.Color Color();
	java.awt.Cursor Cursor();
	java.awt.Point Point();
	java.awt.Dimension Dimension();
	java.awt.Rectangle Rectangle();

	CH.ifa.draw.framework.DrawingEditor DrawingEditor();
	CH.ifa.draw.framework.Drawing Drawing();
	CH.ifa.draw.framework.Painter Painter();
	CH.ifa.draw.framework.PointConstrainer PointConstrainer();

	CH.ifa.draw.framework.Handle Handle();
	CH.ifa.draw.framework.Figure Figure();
}
	\end{lstlisting}
	\caption{JHotDraw Primitives Definition}
	\label{lst:JHotDraw Primitives Definition}
\end{sourcecode}

This has been proven very helpful since we did not need to re-implement every field as managed data during the refactoring. 
Especially, classes that are provided by libraries such as \texttt{javax.swing} and \texttt{java.awt}.

\subsubsection{Limitations}
However, extending our framework's primitives with the JHotDrawPrimitives we lost the ``pureness'' of managed data.
That led to an application that partly managed data.
Generally, this may be the case when refactoring big applications like JHotDraw.

Another limitation is that some Java keywords such as ``synchronized'' can not be supported on default methods.
Instead, as future work, we could use annotations that define these properties to the default methods and add them during the interpretation of the schemas.
Moreover, privacy is another an issue.
All default methods are \texttt{public}, which means that the encapsulation is violated.
Finally, private classes definition is not possible inside schemas, although they can be defined outside as managed data.

\subsection{MDDrawingView Schema Factories}
In order to create instances of the defined \texttt{MDStandardDrawingView} and \texttt{MDNullDrawingView} schemas, we needed their factories.
Besides the schema factories, which is as simple as Listing \ref{lst:DrawingViewSchemaFactory} shows, we still needed a way to give initialization values to the schema instances the same way that the original \texttt{StandardDrawingView} does during construction.
Additionally, this factory should be used like Java's \texttt{new} keyword in the source code.
This factory just replicates the original \texttt{StandardDrawingView} constructor and is used from the program to create new instances of the schemas.
The code of the \texttt{MDStandardDrawingView} \textit{instances factory} is illustrated in Listing \ref{lst:MDStandardDrawingView Instances Factory}, in comparison to the original constructor, illustrated in Listing \ref{lst:DrawingView Constructor}.

\begin{sourcecode}[H]
	\begin{lstlisting}[language=Java, escapechar=|]
public interface DrawingViewSchemaFactory {
	MDStandardDrawingView DrawingView();
	MDNullDrawingView NullDrawingView();
}
	\end{lstlisting}
	\caption{DrawingView Schema Factory}
	\label{lst:DrawingViewSchemaFactory}
\end{sourcecode}

\begin{sourcecode}[H]
	\begin{lstlisting}[language=Java, escapechar=|]
public StandardDrawingView(DrawingEditor editor, int width, int height) {
	setAutoscrolls(true);
	fEditor = editor;
	fViewSize = new Dimension(width,height);
	setSize(width, height);
	fSelectionListeners = CollectionsFactory.current().createList();
	addFigureSelectionListener(editor()); |\label{line:addFigureSelectionListener_contructor}|
	setLastClick(new Point(0, 0));
	fConstrainer = null;
	fSelection = CollectionsFactory.current().createList();
	setDisplayUpdate(createDisplayUpdate());
	setBackground(Color.lightGray);
	addMouseListener(createMouseListener());
	addMouseMotionListener(createMouseMotionListener());
	addKeyListener(createKeyListener());
}
	\end{lstlisting}
	\label{lst:DrawingView Constructor}
	\caption{Original StandardDrawingView Constructor}
\end{sourcecode}

\begin{sourcecode}[H]
	\begin{lstlisting}[language=Java, escapechar=|]
public static MDDrawingView newDrawingView(
	DrawingEditor editor, int width, int height) {
	final MDStandardDrawingView drawingView = drawingViewSchemaFactory.DrawingView();
	MyJPanel jPanel = new MyJPanel();
	jPanel.setAutoscrolls(true);
 	jPanel.setSize(width, height);
	jPanel.setBackground(Color.lightGray);
	drawingView.panel(jPanel);
	jPanel.setDrawingView(drawingView);

	drawingView.editor(editor);
	drawingView.size(new Dimension(width, height));
	jPanel.setSize(width, height);
	drawingView.lastClick(new Point(0, 0));
	drawingView.constrainer(null);
	drawingView.setDisplayUpdate(new SimpleUpdateStrategy());
	drawingView.setBackground(Color.lightGray);
	drawingView.drawing(new StandardDrawing());

	jPanel.addMouseListener(...);
	jPanel.addMouseMotionListener(...);
	jPanel.addKeyListener(...);
	return drawingView;
}
	\end{lstlisting}
	\label{lst:MDStandardDrawingView Instances Factory}
	\caption{MDStandardDrawingView Instances Factory}
\end{sourcecode}

\subsection{MDDrawingView Integration}
Finally, in order to integrate the \texttt{MDDrawingView} managed objects in the existing system, first we had to replace every instance of \texttt{DrawingView} with \texttt{MDDrawingView}, every \texttt{StandardDrawingView} with \texttt{MDStandardDrawingView} and every \texttt{NullDrawingView} with \texttt{MDNullDrawingView} accordingly.
In addition, everywhere a new instance of these is created, we replaced it with our \textit{instances factory}.

%%%%%%%%%%%%%%%%%%%%%%%%%%%%%%%%%%%%%%%%%%%%%%%%%%%%%%%%%%%%%%%%%%%%%%%%%%%%%%%
\section{Aspect Refactoring of JHotDraw}
Aspect refactoring usually refers to the refactoring of legacy code in aspect oriented code. 
However, in this section we present an aspect refactoring of JHotDraw legacy code in managed data.

%%%%%%%%%%%%%%%%%%%%%%%%%%%%%%%%%%%%%%%%%%%%%%%%%%%%%%%%%%%%%%%%%%%%%%%%%%%%%%%
% FigureSelectionListener
%%%%%%%%%%%%%%%%%%%%%%%%%%%%%%%%%%%%%%%%%%%%%%%%%%%%%%%%%%%%%%%%%%%%%%%%%%%%%%%
\subsection{FigureSelectionListener}
The \texttt{FigureSelectionListener} observer pattern of JHotDraw is a concern first presented by Hannemann et al. \cite{hannemann2005role} in their role-based refactoring of design patterns in AspectJ. 
Later, Marin et al. used the same concern and migrated it into their AJHotDraw implementation \cite{marin2005approach}.
Likewise, we have also implemented the same aspect for our refactoring in order to compare our aspect solution with the existing one.

\subsection{FigureSelectionListener in JHotDraw}
The original \texttt{FigureSelectionListener} observer pattern of JHotDraw is illustrated in Figure \ref{fig:JHotDraw_FigureSelectionListener_OOP}.

\begin{figure}[H]
	\centering
  	\fbox{\includegraphics[width=1\textwidth]{figures/JHotDraw_FigureSelectionListener_OOP.png}}
  	\caption{FigureSelectionListener in JHotDraw}
  	\label{fig:JHotDraw_FigureSelectionListener_OOP}
\end{figure}

As this figure illustrates, the \texttt{FigureSelectionListener} interface defines the \textit{Observer} role.
The classes that are interested in the changes of selection of figures in a \texttt{DrawingView} implement this interface.
Accordingly, the \texttt{DrawingView} defines the \textit{Subject} role, providing methods for adding and removing figure selection listeners.
Practically, the only class that implements the \textit{Subject} role is \texttt{StandardDrawingView}, while \texttt{NullDrawingView} has an empty implementation.

\texttt{StandardDrawingView} keeps the selection listeners in a list, the \texttt{fSelectionListeners}, and notifies them in the invocation of the \texttt{fireSelectionChanged} method.
This method is called in the methods: \texttt{addToSelection}, \texttt{removeFromSelection}, \texttt{toggleSelection} and \texttt{clearSelection}, which indicate the change of figure selection.
On the observers' side, the figure selection listeners implement the \texttt{figureSelectionChanged} method that is executed in case they have been notified by the subject.

Concluding, as described above, the ``pattern code'' of the observer pattern is scattered in many places, including the list of listeners on the subject, the add / remove methods, along with the pointcut methods that call the method which notifies the listeners.

%%%%%%%%%%%%%%%%%%%%%%%%%%%%%%%%%%%%%%%%%%%%%%%%%%%%%%%%%%%%%%%%%%%%%%%%%%%%%%%
\subsection{Refactoring FigureSelectionListener in AJHotDraw}
Marin et al. presented a refactoring of this concern in AJHotDraw \cite{marin2005approach}. 
Their refactoring is illustrated in Figure \ref{fig:JHotDraw_FigureSelectionListener_AOP}. 

\begin{figure}[H]
	\centering
  	\fbox{\includegraphics[width=.8\textwidth]{figures/JHotDraw_FigureSelectionListener_AOP.png}}
  	\caption{FigureSelectionListener in AJHotDraw}
  	\label{fig:JHotDraw_FigureSelectionListener_AOP}
\end{figure}

In their proposed type-based refactoring, they have used two crosscut sorts, namely \textit{role superimposition} and \textit{consistent behavior}.

\subsubsection{Role Superimposition}
As defined by the authors \cite{marin2005classification}, ``the role superimposition refers to the implementation of a specific secondary role or responsibility''.
In the case of \texttt{FigureSelectionListener}, they used it twice, one for each of the roles.
More specifically, they defined an abstract \texttt{GenericRole} and concrete roles, observer and subject which extend the abstract one.

\subsubsection{Consistent Behavior}
According to the authors\cite{marin2005classification}, ``the consistent behavior sort implements a consistent behavior for a number of method elements that can be captured by a natural pointcut''.
In this case it is used to notify the \textit{Observers} of the changes in the \textit{Subject} object.
More specifically, the methods \texttt{addToSelection}, \texttt{removeFromSelection}, \texttt{toggleSelection} and \texttt{clearSelection} are consistent behavior.
They implement it as a pointcut in AspectJ shown in Listing \ref{lst:Consistent Behavior in FigureSelectionListener}.

\begin{sourcecode} [H]
	\begin{lstlisting}[language=AspectJ]
public aspect SelectionChangedNotification {
	pointcut invalidateSelFigure(StandardDrawingView sdw) :
		(   withincode(boolean StandardDrawingView.addToSelectionImpl(Figure)) 
		 || withincode(void StandardDrawingView.removeFromSelection(Figure)))
		&& call(void Figure.invalidate()) 
		&& this(sdw);

	pointcut clear_toggleSelection(StandardDrawingView sdw):
		(execution(void StandardDrawingView.clearSelection()) ||
		 execution(void StandardDrawingView.toggleSelection(Figure)))
		&& this(sdw);

	after(StandardDrawingView sdw): invalidateSelFigure(sdw) {
		sdw.fireSelectionChanged();
	}

	after(StandardDrawingView sdw): clear_toggleSelection(sdw) {
		sdw.fireSelectionChanged();
	}
}
	\end{lstlisting}
	\caption{AJHotDraw: Consistent Behavior in FigureSelectionListener}
	\label{lst:Consistent Behavior in FigureSelectionListener}
\end{sourcecode}

\subsubsection{Benefits and Limitations}
According to the authors \cite{marin2005approach}, such refactoring allows the crosscutting elements to be addressed individually, which leads to a modular solution, and any deviations from the pattern implementation can be addressed separately.

However, as they mention, the definition of pointcuts in order to capture the calls to the notifier is difficult when many consistent behavior instances occur. 
As Listing \ref{lst:StandardDrawingView clearSelection Method} shows, the original \texttt{clearSelection} method in JHotDraw calls \texttt{fireSelectionChanged} under specific conditions.
Considering the \ac{aop} solution of AJHotDraw, Listing \ref{lst:Consistent Behavior in FigureSelectionListener}, this is not the case.
In the pointcut definition, the pattern refactoring solution notifies the observers independently of the condition in the caller.
Although, according to Marin et al. it is potentially harmless in this case, this implementation deviates from the behavior of the original JHotDraw, leading to a harmful, for the functionality, implementation.
Finally, the problem of the unconditional call of a method in a pointcut is clearly a problem of the language.
AspectJ mechanisms do not support such functionality.
But can managed data solve this problem?

\begin{sourcecode}
	\begin{lstlisting}[language=AspectJ, escapechar=|]
public void clearSelection() {
	if (selectionCount() == 0) {
		// avoid unnecessary selection changed event when nothing has to be cleared
		return;
	}
	FigureEnumeration fe = selection();
	while (fe.hasNextFigure()) {
		fe.nextFigure().invalidate();
	}
	...
	fireSelectionChanged();
}
	\end{lstlisting}
	\caption{StandardDrawingView clearSelection Method}
	\label{lst:StandardDrawingView clearSelection Method}
\end{sourcecode}

%%%%%%%%%%%%%%%%%%%%%%%%%%%%%%%%%%%%%%%%%%%%%%%%%%%%%%%%%%%%%%%%%%%%%%%%%%%%%%%
\subsection{Refactoring FigureSelectionListener in ManagedDataJHotDraw}
In JHotDraw's original code, the \textit{observer} \texttt{DrawApplication} creates a new \texttt{StandardDrawingView} instance using the \texttt{createDrawingView} method.
During the construction the \texttt{DrawApplication} passes itself to the constructor of the \texttt{StandardDrawingView} and this in turn adds it to its listeners list, using the \texttt{addFigureSelectionListener} method.
This is shown in Line \ref{line:addFigureSelectionListener_contructor} of Listing \ref{lst:DrawingView Constructor}.
Likewise, the rest of the classes that implement the \texttt{FigureSelectionListener} interface, perform the same mechanism, adding or removing themselves from the \texttt{DrawingView} \textit{Subject}.
Consequently, the pattern code is scattered among all of its participants.

In this section we present our managed data refactoring of the \texttt{FigureSelectionListener} concern.
Managed data implements aspects using data managers, by adding specifications to the data.
For this case, we needed a similar mechanism to the \textit{role superimposition} of \ac{aop}.
This mechanism should be defined in a data manager that will produce managed data instances (managed objects) with a specific role.
Additionally, the data manager has to support something similar to the \textit{consistent behavior} as a pointcut.

In detail, since the \texttt{DrawingView} is managed data, and it is the \textit{Subject} to the \textit{Listeners} of the \textit{FigureSelectionListener} case, we can implement a data manager that attaches the \textit{Subject} \texttt{MDStandardDrawingView}.
Therefore, no \textit{Subject} role specific code will be tangled with the \texttt{DrawingView}, but a data manager will attach this role later.

More specifically, we needed a data manager that performs the following:

\begin{enumerate}
	\item Attaches the \textit{Subject} role to the \texttt{MDStandardDrawingView} since this object implements the pattern.
	Initially, \texttt{MDStandardDrawingView} has no \textit{Subject} role related code.

	\item Enables the \textit{Subject} to \textit{add} and \textit{remove} listener objects to and from itself. 
	\newline
	In this case the \texttt{FigureSelectionListeners}.

	\item Defines an \textit{Action} that will be executed on the listeners in case of the \textit{Subject}'s notification.
	\newline
	In this case the code in the \texttt{figureSelectionChanged()} method.

	\item Finally, it defines a pointcut for the consistent behavior that executes the actions on the listeners. 
	In this case the \texttt{addToSelection}, \texttt{removeFromSelection}, \texttt{toggleSelection} and \texttt{clearSelection} methods.
\end{enumerate}

\subsubsection{Data manager}
As it is mentioned in Chapter \ref{Implementation}, the role of a data manager class is to create a \texttt{MObject}, which interprets and handles a managed object instance.
First, this \texttt{MObject} namely FigureSelectionListenerSubjectRole MObject had to implement our specifications.
In this case the specification was the \textit{Subject} role.

\subsubsection{SubjectRole specification}
We define the functionality of the \textit{SubjectRole} in an interface, shown in Figure \ref{lst:SubjectRole Interface}.
The subject role simply needs to add and remove listener object to and from a managed object.

\begin{sourcecode} [H]
	\begin{lstlisting}[language=Java, escapechar=|]
public interface SubjectRole {
	void addListener(Object listener, Action action);
	void removeListener(Object listener);
}
	\end{lstlisting}
	\caption{SubjectRole Interface}
	\label{lst:SubjectRole Interface}
\end{sourcecode}

\subsubsection{Action}
Additional to the listener object a \textit{SubjectRole} has to define an \textit{Action} for that listener.
That \textit{Action} determines the method which will be executed in that \textit{Listener} in case a notification is retrieved from the \textit{Subject}.
As Listing \ref{lst:Action Interface} shows, this is simply a functional interface that represents an executable action.

\begin{sourcecode} [H]
	\begin{lstlisting}[language=Java, escapechar=|]
@FunctionalInterface
public interface Action {
	void execute();
}
	\end{lstlisting}
	\caption{Action Interface}
	\label{lst:Action Interface}
\end{sourcecode}

\subsubsection{Consistent Behavior Pointcut}
Next, as with the \ac{aop} version, we also needed to define the pointcut for the consistent behavior concern sort.
For practical reasons, we used an interface to define them.
As Figure \ref{lst:FigureSelectionPointcut} shows, a list of the methods that execute the \texttt{Action} for each listener, are defined in the \texttt{FigureSelectionPointcut} interface. 

\begin{sourcecode} [H]
	\begin{lstlisting}[language=Java, escapechar=|]
public interface FigureSelectionPointcut {
	void addToSelection(Figure figure);
	void removeFromSelection(Figure figure);
	void toggleSelection(Figure figure);
	void clearSelection();
}
	\end{lstlisting}
	\caption{FigureSelectionPointcut Interface}
	\label{lst:FigureSelectionPointcut}
\end{sourcecode}

\subsubsection{FigureSelectionListenerSubjectRole MObject}
Finally, having all of our specifications in place, we needed to implement the actual MObject that uses them.
In particular, the FigureSelectionListenerSubjectRole MObject implements those specifications and provides its managed objects with the ability to use them.

\paragraph{Role Superimposition}\mbox{}\\
The implementation of the FigureSelectionListenerSubjectRole MObject is presented in Listing \ref{lst:FigureSelectionListenerSubjectRoleMObject}.
First, the FigureSelectionListenerSubjectRoleMObject extends the MObject, inheriting the functionalities of the base data manager, followed by the implementation of the \texttt{SubjectRole} specifications.
By implementing the \texttt{SubjectRole} interface, the MObject has to implement the \texttt{addListener} and \texttt{removeListener} methods that provide the subject role specifications.
Having added a listener object along with its \texttt{Action} to be executed on each notification, the method \texttt{executeListenerActions} executes all the actions for each of the listeners.
Since we have implemented a form of \textit{role superimposition} what is left is the \textit{consistent behavior} pointcut.

\paragraph{Consistent Behavior Pointcut}\mbox{}\\
Considering that an MObject is an Invocation Handler, every method invocation passes through that object first.
By defining the pointcut in an interface and extending that interface in this MObject, we proxy the execution of the real object's methods, starting with MObject first.
This allows the programmer to add functionalities in these methods which in other cases would scatter the real object's methods.
Similarly to the \ac{aop} solution, the pointcut includes the three methods that call the \texttt{fireSelectionChanged} method.
However, in managed data, we are not limited to a specific method of a class but to an \texttt{Action} passed for the specific listener.
Invoking the \texttt{executeListenerActions} method on each of the methods defined by the pointcut, we have implemented the concern as a modular aspect.

\paragraph{Conditions in Pointcuts}\mbox{}\\
As it has been seen from the \ac{aop} solution, during the pointcut definition, the language did not allow to add any kind of conditions or other functionalities based on the state of the object.
However, since the MObject is proxied to a \texttt{MDStandardDrawingView} instance, the programmer can access the current state of the instance inside the data manager implementation.
Therefore, the programmer can use the state of the program.
In this case, the execution of the action on the listener is performed under a specific condition, (Line \ref{line:fig_lis_mobj_clear} of Listing \ref{lst:FigureSelectionListenerSubjectRoleMObject}), which is similar to the one defined on the original program \ref{lst:StandardDrawingView clearSelection Method}.

\begin{sourcecode} [H]
	\begin{lstlisting}[language=Java, escapechar=|]
public class FigureSelectionListenerSubjectRoleMObject 
	extends MObject implements FigureSelectionPointcut, SubjectRole { |\label{line:fig_lis_mobj_extends}|
	private Map<Object, Action> listeners;
	...
	private void executeListenerActions() { |\label{line:fig_lis_mobj_execute}|
		listeners.values().forEach(Action::execute);
	}

	public void addToSelection(Figure figure) {
		executeListenerActions();
	}

	public void removeFromSelection(Figure figure) {
		executeListenerActions();
	}

	public void toggleSelection(Figure figure) {
		executeListenerActions();
	}

	public void clearSelection() { |\label{line:fig_lis_mobj_clear}|
		if (((MDStandardDrawingView) this.getProxy()).selectionCount() > 0) {
			executeListenerActions();
		}
	}

	public void addListener(Object listener, Action action) {
		listeners.put(listener, action);
	}

	public void removeListener(Object listener) {
		listeners.remove(listener);
	}
}
	\end{lstlisting}
	\caption{FigureSelectionListenerSubjectRoleMObject}
	\label{lst:FigureSelectionListenerSubjectRoleMObject}
\end{sourcecode}

\begin{figure}
	\centering
  	\fbox{\includegraphics[width=1\textwidth]{figures/JHotDraw_FigureSelectionListener_MD.png}}
  	\caption{FigureSelectionListener in ManagedDataJHotDraw}
  	\label{fig:JHotDraw_FigureSelectionListener_MD}
\end{figure}

\subsection{Results}
Figure \ref{fig:JHotDraw_FigureSelectionListener_MD} illustrates the refactored version of the FigureSelectionListener concern in ManagedDataJHotDraw.
Comparing it to the original, Figure \ref{fig:JHotDraw_FigureSelectionListener_OOP}, it can be seen that, first, the list of listeners has been removed from the \texttt{DrawingView}. 
Next, the \texttt{addListener} and \texttt{removeListener} methods have also been removed from the class.
Every call of the \texttt{fireSelectionChanged} method in the pointcut methods has also been omitted.
Finally, conditions on the pointcuts have been defined, something that is not supported by the \ac{aop} version, AJHotDraw.

The integration of the data manager was executed simply by using our schema factories and adding the listeners during construction.
Most importantly, the behavior of the application remained equivalent to the original.
ManagedDataJHotDraw conserved the behavior of JHotDraw, which we evaluated through its own test suite along with manual tests.

% TODO: Undo
%%%%%%%%%%%%%%%%%%%%%%%%%%%%%%%%%%%%%%%%%%%%%%%%%%%%%%%%%%%%%%%%%%%%%%%%%%%%%%%
% Undo Concern
%%%%%%%%%%%%%%%%%%%%%%%%%%%%%%%%%%%%%%%%%%%%%%%%%%%%%%%%%%%%%%%%%%%%%%%%%%%%%%%
% \subsection{Undo Concern}
% The ``Undo'' functionality is used in several places in the original JHotDraw.
% Marin's fan-in analysis \cite{marin2004identifying}, identified about 30 undo activities defined for various elements of JHotDraw. 
% For our assessment we focused on the refactoring of the undo concern in the \textit{ChangeAttributeCommand} class.
% We choose the specific case since is the same that is used by Marin et al. on their Undo refactoring in AJHotDraw \cite{marin2004refactoring}.

% \subsubsection{Undo Concern in JHotDraw}
% The original Undo concern in \textit{ChangeAttributeCommand} of JHotDraw is illustrated in Figure \ref{fig:JHotDraw_Undo_Command_ChangeAttributeCommand_OOP}.

% \begin{figure} [H]
% 	\centering
%   	\fbox{\includegraphics[width=0.8\textwidth]{figures/JHotDraw_Undo_Command_ChangeAttributeCommand_OOP.png}}
%   	\caption{Undo in JHotDraw}
%   	\label{fig:JHotDraw_Undo_Command_ChangeAttributeCommand_OOP}
% \end{figure}

% As this figure illustrates, the Undo concern is used through an implementation of the \textit{Command} design pattern.
% More specifically, every activity in the application is executed though a command.
% This command is represented by the \texttt{Command} interface in the figure. 
% Some of the activities support the undo functionality, which in JHotDraw is implemented in nested (undo) classes.

% In the case of the \textit{ChangeAttributeCommand}, the command is called when an attribute is applied to a figure.
% An attribute can be a color, a font, a url etc.
% When an attribute has been applied using an \textit{ChangeAttributeCommand} object, the object defines its \texttt{UndoActivity} through the \textit{UndoActivity} private class.
% JHotDraw supports repeated undo operations by recording the last executed commands in reversed order. 

% This is the role of the  \texttt{UndoableCommand} class, which wraps the commands that can be undone.
% In particular, this class first gets the the request to execute the command, then, it delegates the command's execution to the wrapped command, and last, acquires a reference to the undo activity associated with the wrapped command and it pushes it into a stack managed by an \texttt{UndoManager} \cite{marin2004refactoring}.

% Therefore, as Marin et al. extracted \cite{marin2004identifying} the Undo concern code is scattered in several places of the Command classes.
% First, the \texttt{myUndoableActivity} field in the \texttt{AbstractCommandClass} along with its accessors, \texttt{getUndoActivity} and \texttt{setUndoActivity} accordingly.
% Next, the private nested classes that are implemented by the most of the concrete commands that support undo.
% In addition, the factory methods, \texttt{createUndoActivity},which create instances of the private classes.
% Finally, the references to the before enumerated elements from non-undo related members.

% \subsubsection{Refactoring Undo in AJHotDraw}
% The refactoring that Marin et al. proposed can be seen in Figure \ref{fig:JHotDraw_Undo_Command_ChangeAttributeCommand_AOP}.

% \begin{figure} [H]
% 	\centering
%   	\fbox{\includegraphics[width=0.65\textwidth]{figures/JHotDraw_Undo_Command_ChangeAttributeCommand_AOP.png}}
%   	\caption{Undo in AJHotDraw}
%   	\label{fig:JHotDraw_Undo_Command_ChangeAttributeCommand_AOP}
% \end{figure}

% As the figure shows, a new aspect is created for the \textit{ChangeAttributeCommand}.
% In this aspect the entire undo functionality is implemented and the undo code is removed from the actual \textit{ChangeAttributeCommand} class. 
% Additionally, the private class that implements the \textit{UndoActivity} has moved to this aspect along with its factory method (\textit{createUndoActivity}).
% However, by convention, each aspect will consistently be named by appending ``UndoActivity'' to the name of its associated command class to enforce the relation between the two.
% All the abstract undo functionality has been defined in a \textit{CommandUndo} aspect.
% This aspect defined the undo as a role while it defines the \textit{Undoable} field and its accessors of the \textit{AbstractCommand}.

% \subsubsection{Refactoring Undo in ManagedDataJHotDraw}

%%%%%%%%%%%%%%%%%%%%%%%%%%%%%%%%%%%%%%%%%%%%%%%%%%%%%%%%%%%%%%%%%%%%%%%%%%%%%%%
% Claims
%%%%%%%%%%%%%%%%%%%%%%%%%%%%%%%%%%%%%%%%%%%%%%%%%%%%%%%%%%%%%%%%%%%%%%%%%%%%%%%
\section{Claims}
We claim that our framework successfully performed aspect refactoring of the \texttt{FigureSelectionListener} concern in JHotDraw.
Further, it evolved the pointcut concept of \ac{aop} by adding conditions on them.
This leads to a solution that focuses on behavior conservation, lost in the case of the \ac{aop} refactoring.
Finally, although it is moderately easy to migrate an existing application in managed data, it is still time consuming with a lot of boilerplate code if one wants to make everything managed data.

%%%%%%%%%%%%%%%%%%%%%%%%%%%%%%%%%%%%%%%%%%%%%%%%%%%%%%%%%%%%%%%%%%%%%%%%%%%%%%%
% Chapter: Evaluation
%%%%%%%%%%%%%%%%%%%%%%%%%%%%%%%%%%%%%%%%%%%%%%%%%%%%%%%%%%%%%%%%%%%%%%%%%%%%%%%
\chapter{Evaluation}\label{Evaluation}

%%%%%%%%%%%%%%%%%%%%%%%%%%%%%%%%%%%%%%%%%%%%%%%%%%%%%%%%%%%%%%%%%%%%%%%%%%%%%%%
% Section: JHotDraw
%%%%%%%%%%%%%%%%%%%%%%%%%%%%%%%%%%%%%%%%%%%%%%%%%%%%%%%%%%%%%%%%%%%%%%%%%%%%%%%
\section{JHotDraw And AJHotDraw}\label{JHotDraw And AJHotDraw}

%%%%%%%%%%%%%%%%%%%%%%%%%%%%%%%%%%%%%%%%%%%%%%%%%%%%%%%%%%%%%%%%%%%%%%%%%%%%%%%
\subsection{Refactoring of Crosscutting Concerns}

%%%%%%%%%%%%%%%%%%%%%%%%%%%%%%%%%%%%%%%%%%%%%%%%%%%%%%%%%%%%%%%%%%%%%%%%%%%%%%%
\subsubsection{Role-based Refactoring of Crosscutting Concerns.}

%%%%%%%%%%%%%%%%%%%%%%%%%%%%%%%%%%%%%%%%%%%%%%%%%%%%%%%%%%%%%%%%%%%%%%%%%%%%%%%
\subsubsection{Evaluation}

%%%%%%%%%%%%%%%%%%%%%%%%%%%%%%%%%%%%%%%%%%%%%%%%%%%%%%%%%%%%%%%%%%%%%%%%%%%%%%%
\subsection{The Undo Concern of JHotDraw}

%%%%%%%%%%%%%%%%%%%%%%%%%%%%%%%%%%%%%%%%%%%%%%%%%%%%%%%%%%%%%%%%%%%%%%%%%%%%%%%
\subsubsection{Evaluation}

%%%%%%%%%%%%%%%%%%%%%%%%%%%%%%%%%%%%%%%%%%%%%%%%%%%%%%%%%%%%%%%%%%%%%%%%%%%%%%%
\subsubsection{AspectJ Drawbacks in the Undo Solution}

%%%%%%%%%%%%%%%%%%%%%%%%%%%%%%%%%%%%%%%%%%%%%%%%%%%%%%%%%%%%%%%%%%%%%%%%%%%%%%%
\subsection{The Persistence Concern of JHotDraw}

%%%%%%%%%%%%%%%%%%%%%%%%%%%%%%%%%%%%%%%%%%%%%%%%%%%%%%%%%%%%%%%%%%%%%%%%%%%%%%%
% Section: Research Questions and Answers
%%%%%%%%%%%%%%%%%%%%%%%%%%%%%%%%%%%%%%%%%%%%%%%%%%%%%%%%%%%%%%%%%%%%%%%%%%%%%%%
\section{Research Questions and Answers}\label{Research Questions and Answers}

%%%%%%%%%%%%%%%%%%%%%%%%%%%%%%%%%%%%%%%%%%%%%%%%%%%%%%%%%%%%%%%%%%%%%%%%%%%%%%%
\section{Evidence}\label{Evidence}

%%%%%%%%%%%%%%%%%%%%%%%%%%%%%%%%%%%%%%%%%%%%%%%%%%%%%%%%%%%%%%%%%%%%%%%%%%%%%%%
\subsection{Design Patterns}\label{Design Patterns JHotDraw}

%%%%%%%%%%%%%%%%%%%%%%%%%%%%%%%%%%%%%%%%%%%%%%%%%%%%%%%%%%%%%%%%%%%%%%%%%%%%%%%
\subsection{Undo Concern of JHotDraw}\label{Undo JHotDraw}

%%%%%%%%%%%%%%%%%%%%%%%%%%%%%%%%%%%%%%%%%%%%%%%%%%%%%%%%%%%%%%%%%%%%%%%%%%%%%%%
\subsection{Persistence Concern of JHotDraw}\label{Persistence JHotDraw}

%%%%%%%%%%%%%%%%%%%%%%%%%%%%%%%%%%%%%%%%%%%%%%%%%%%%%%%%%%%%%%%%%%%%%%%%%%%%%%%
\section{Results}\label{Results}

%%%%%%%%%%%%%%%%%%%%%%%%%%%%%%%%%%%%%%%%%%%%%%%%%%%%%%%%%%%%%%%%%%%%%%%%%%%%%%%
\section{Claims}\label{Claims}
% !TEX root = ../thesis.tex

%%%%%%%%%%%%%%%%%%%%%%%%%%%%%%%%%%%%%%%%%%%%%%%%%%%%%%%%%%%%%%%%%%%%%%%%%%%%%%%
% Chapter: Conclusion
%%%%%%%%%%%%%%%%%%%%%%%%%%%%%%%%%%%%%%%%%%%%%%%%%%%%%%%%%%%%%%%%%%%%%%%%%%%%%%%
\chapter{Conclusion}\label{Conclusion}
% TODO:
% In this research we have presented a managed data implementation in Java using its reflection API and dynamic proxies.
% By doing that we allow the programmer to take control over the mechanisms of data creation and manipulation.
% We defined our language using Java's syntax by using interfaces and annotations.

% Having the managed data implementation in place, we refactored an existing Object Oriented use case, the JHotDraw.
% This use case is considered as a well-designed \ac{oop} system; however, it includes the problem of the \ac{ccc}.
% Thus, we refactored this system using managed data in order to solve the \ac{ccc} problem.
% More specifically, we have migrated some main components of JHotDraw to managed data, then we removed the \ac{ccc} and finally we used data managers to implement them.
% This refactoring led to a new version of JHotDraw the ManagedDataJHotDraw which solves the problem of some main \ac{ccc} of the original application.

% During the assessment of our refactoring we have collected a number of metrics that we used in order to evaluate our refactoring in comparison with the original application.
% Moreover, we extensively presented the refactoring process of JHotDraw and compared with AJHotDraw, the Aspect Oriented implementation of JHotDraw.
% Finally, by presenting a set of metrics and a number of modularity properties, we assessed our results comparing them in with the original and the Aspect Oriented version.

% Overall, we showed that managed data can be implemented in a static language and it can tame aspects by using data managers for concern implementation.
% Moreover, it extends some capabilities of AspectJ and deviates from the problem of the coupling between aspect definition and components.


% \chapter*{Acknowledgments}
% This is all started when my supervisor, Tijs van der Storm, provided us with the classic Aspect Oriented Programming paper during our software construction course of this Master's program.
% I was always thinking that crosscutting concerns is a problem of modularity and code cloning, scattering was annoying and I knew this was bad, but I never knew its name.
% But then the Aspect Oriented Programming paper explained it to me explicitly.
% First thing was to go to Tijs and ask him if AOP really works since I was not really convinced.
% How can no one use this technique if it really works?
% Tijs then explained to me that he and William R. Cook worked together on something different, that would solve that problem in a different way, the Managed Data.
% That day I left from his office with a list of papers discussing AOP and of course Managed Data.
% I've started implement it in Java right away in order to see if it actually works.
% Well, it worked, and I got very excited.
% This was the time I knew to do this.

% For this thesis, first I would like to express my sincere gratitude to Tijs van der Storm, my supervisor who introduced me to this idea.
% Tijs was a great inspiration for me from the first time.
% He helped me with every single problem I faced during this process.
% I will never forget the bootstraping process, Tijs had the patience to explain it to me a number of times, it was always mind-blowing, until I understand it.
% Additionally I want to thank Tijs because he was a real inspiration for me on his way of thinking and of course perfectionism. 
% He changed my way of programming and thinking about programming completely.

% Moreover, I would like to thank my colleagues and friends from CWI for the interesting and inspiring conversations.
% However, my biggest support was from Ifigeneia.
% The writing of this thesis would never be complete if she was not on my side.
% I express my thanks to Ifigeneia for her constant help, inspiration and the interest she showed on the subject during its completion.

\appendix
% !TEX root = ../thesis.tex

%%%%%%%%%%%%%%%%%%%%%%%%%%%%%%%%%%%%%%%%%%%%%%%%%%%%%%%%%%%%%%%%%%%%%%%%%%%%%%%
% Chapter: MObject
%%%%%%%%%%%%%%%%%%%%%%%%%%%%%%%%%%%%%%%%%%%%%%%%%%%%%%%%%%%%%%%%%%%%%%%%%%%%%%%
\chapter{The MObject class}\label{apdx:MObject}
This class is the managed data interpreter; therefore, it first it setups the fields of a managed object based on the schema klass.
Next, it handles all the calls to the managed objects' methods and finally is responsible for type checking of the fields' values during initialization or assignment.

The usage of the \texttt{schemaKlass} for setting up the fields is shown in Listing \ref{lst:setup_fields}.

\begin{sourcecode}
	\begin{lstlisting}[language=Java, escapechar=|]
public MObject(Klass schemaKlass, Object... initializers) {
	this.schemaKlass = schemaKlass;
	this.schemaKlass.fields().forEach(this::safeSetupField);|\label{line:setup_fields}|
	this.safeInitializeProps(initializers);
}

protected void setupField(Field field) {
	if (!field.many()) { |\label{line:setup_field_many_check}|
		if (field.type().schemaKlass().name().equals("Primitive")) {|\label{line:instanceof}|
			this.props.put(field.name(), new MObjectFieldSinglePrimitive(this, field));
		} else {
			this.props.put(field.name(), new MObjectFieldSingleMObj(this, field));
		}
	} else {
		if (field.type().schemaKlass().name().equals("Primitive")) {
			this.props.put(field.name(), new MObjectFieldManyList(this, field));
		} else {

			Klass klassType = (Klass) field.type();
			if (klassType.key()!= null) {
				this.props.put(field.name(), new MObjectFieldManySet(this, field));
			} else {
				this.props.put(field.name(), new MObjectFieldManyList(this, field));
			}
		}
	}
}
	\end{lstlisting}
	\caption{MObject: setup fields}
	\label{lst:setup_fields}
\end{sourcecode}

The \texttt{schemaKlass} is given to the \texttt{MObject} by the \texttt{DataManager} that is responsible for creating it.
Using this \texttt{schemaKlass} the \texttt{MObject} setups the \texttt{Fields} of the Klass, Line \ref{line:setup_fields}.
Inside the \texttt{setupField} method the interpretation of the schema is performed.
In particular, in Line \ref{line:setup_field_many_check} we check if that field is a multi-value field, and if not, we just set it up as a \texttt{Primitive} or a \texttt{Klass} accordingly. 
Consider that the \texttt{field.type().schemaKlass().name()} is used like a common \texttt{instanceof} in Line \ref{line:instanceof}.
In case the field has many values, we first check if it is \texttt{Primitive}, since we do not support Set of \texttt{Primitive}s.
Following that, we check if a \texttt{Key} field exists on that field's type and in that case this field is a Set, otherwise it is a List.

The invocation handling process of managed object is showed in Listing \ref{lst:mobject_invocation_handler}.

\begin{sourcecode}
	\begin{lstlisting}[language=Java, escapechar=|]
public Object invoke(
	Object proxy, Method method, Object[] args) throws Throwable {
	final String fieldName = method.getName();

	if (method.isDefault()) { // if the method is default, invoke this one
		return _callDefaultMethod(proxy, method, args);
	}

	// This is a way to execute the "attached" methods of the derived Managed Objects,
	for (Method declaredMethod : this.getClass().getMethods()) {
		if (declaredMethod.getName().equals(fieldName)) {
			return method.invoke(this, args);
		}
	}

	// Managed Object
	MObjectField mObjectField = this.props.get(fieldName);
	boolean isMany = mObjectField.getField().many();

	if (args == null) {
		return _get(fieldName); // return the field's value
	}

	boolean isAssignment = false;
	Object fieldArgs = args[0];

	if (fieldArgs.getClass().isArray() && ((Object [])fieldArgs).length > 0) {
		isAssignment = true;
	}

	if (isAssignment) {
		if (((Object [])fieldArgs).length == 1 && !isMany) {
			_set(fieldName, ((Object [])fieldArgs)[0]);
		} else {
			_set(fieldName, fieldArgs);
		}
		return null;
	}
	return _get(fieldName);
}
	\end{lstlisting}
	\caption{MObject: invocation handler}
	\label{lst:mobject_invocation_handler}
\end{sourcecode}

The type checking for each field is performed by the classes \texttt{MObjectFieldSinglePrimitive}, \texttt{MObjectFieldSingleMObj}, \texttt{MObjectFieldManyList} and \texttt{MObjectFieldManySet}.

The basic structure of such class is given in Listing \ref{lst:MObjectField}.

\begin{sourcecode}
	\begin{lstlisting}[language=Java, escapechar=|]
// A field of managed data
public abstract class MObjectField {

	// the owner of the field as an Managed Object.
	protected final MObject owner;

	// the Field.
	protected final Field field;

	// the Inverse of the field.
	protected final Field inverse;

	public MObjectField(MObject owner, Field field) {
		this.owner = owner;
		this.field = field;
		this.inverse = field.inverse();
	}

	// Initializes the field with a value
	public abstract void init(Object value) 
		throws InvalidFieldValueException, NoKeyFieldException;

	// Checks the given value if it is valid
	protected abstract void check(Object value) 
		throws InvalidFieldValueException;

	// Returns a default value for this kind of field.
	protected abstract Object defaultValue() 
		throws UnknownTypeException;

	// Sets a value to the field.
	public abstract void set(Object value) 
		throws InvalidFieldValueException, NoKeyFieldException;

	// Returns the value of the field
	public abstract Object get();

	// Returns the Field object that is wrapped.
	public Field getField() {
		return this.field;
	}
}
\end{lstlisting}
	\caption{MObjectField abstract class}
	\label{lst:MObjectField}
\end{sourcecode}
% !TEX root = ../thesis.tex

%%%%%%%%%%%%%%%%%%%%%%%%%%%%%%%%%%%%%%%%%%%%%%%%%%%%%%%%%%%%%%%%%%%%%%%%%%%%%%%
% Chapter: Schema Loading
%%%%%%%%%%%%%%%%%%%%%%%%%%%%%%%%%%%%%%%%%%%%%%%%%%%%%%%%%%%%%%%%%%%%%%%%%%%%%%%
\chapter{Schema Loading}\label{appdx:SchemaLoading}

\section{Load method}

\begin{sourcecode} [H]
	\begin{lstlisting}[language=Java, escapechar=|]
public static Schema load(
	SchemaFactory factory, Class<?>... schemaKlassesDef) {

	// Filter out primitives by loading them separately
	final List<Class<?>> schemaKlasses = new LinkedList<>();
	for (Class<?> schemaClass : schemaKlassesDef) {
		if (Primitives.class.isAssignableFrom(schemaClass)) {
			primitiveManager.loadPrimitives(schemaClass);
		} else {
			schemaKlasses.add(schemaClass);
		}
	}

	// create an empty schema using the factory, will wire it later
	final Schema schema = factory.Schema();

	// build the types from the schema klasses definition
	final Set<Type> types = buildTypesFromClasses(factory, schema, schemaKlasses);

	// wire the types on schema
	// it is inverse so it will refer to schema.types() directly
	types.forEach(type -> type.schema(schema));

	// get the schema's schemaKlass
	final Klass schemaSchemaKlass = factory.Klass().schemaKlass();

	// wire the schema's schemaKlass
	schema.schemaKlass(schemaSchemaKlass);
	return schema;
}
	\end{lstlisting}
	\caption{SchemaLoader load method}
	\label{lst:SchemaLoader_load}
\end{sourcecode}

\section{Build Types Method}

\begin{sourcecode} [H]
	\begin{lstlisting}[language=Java, escapechar=|]
private static Set<Type> buildTypesFromClasses(
	SchemaFactory factory,
	Schema schema,
	List<Class<?>> schemaKlassesDefinition) {
	Map<Type, TypeWithClass> types = new LinkedHashMap<>();

	// <classNameFieldNameCombo, FieldWithMethod>
	Map<String, FieldWithMethod> allFieldsWithReturnType = new LinkedHashMap<>();

	// Klasses
	for (Class<?> schemaKlassDefinition : schemaKlassesDefinition) {
		String klassName = schemaKlassDefinition.getSimpleName();

		Map<String, Field> fieldsForKlass =
			buildFieldsFromMethods(
				klassName, factory, schemaKlassDefinition, allFieldsWithReturnType);

		// create a new klass
		Klass klass = factory.Klass();
		klass.name(klassName);
		klass.schema(schema);

		// wire the owner klass in fields, it is inverse refering to klass.fields()
		fieldsForKlass.values().forEach(field -> field.owner(klass));

		typesCache.put(klass.name(), klass);

		// add the a new klass
		types.put(klass, new TypeWithClass(klass, schemaKlassDefinition));
	}

	// wiring
	wireFieldTypes(factory, schema, allFieldsWithReturnType);
	wireFieldInverse(allFieldsWithReturnType);
	wireFieldTypeKeys(types);

	wireSchemaKlasses(schema.schemaKlass().schema());

	wireKlassSupers(types, typesCache);
	wireKlassSubs(types, typesCache);
	wireKlassClassOf(types, schemaKlassesDefinition);

	typesCache.clear();

	return types.keySet();
}
    \end{lstlisting}
	\caption{SchemaLoader buildTypesFromClasses method}
	\label{lst:SchemaLoader_buildTypesFromClasses}
\end{sourcecode}

\section{Build Fields Method}

\begin{sourcecode} [H]
	\begin{lstlisting}[language=Java, escapechar=|]
public static Map<String, Field> buildFieldsFromMethods(
	String klassName,
	SchemaFactory factory,
	Class<?> schemaKlassDefinition,
	Map<String, FieldWithMethod> allFieldsWithReturnType) {
	Map<String, Field> fieldsForKlass = new LinkedHashMap<>();

	final Method[] fields = schemaKlassDefinition.getMethods();
	// sort methods
	...

	for (Method schemaKlassField : fields) {
		final String fieldName = schemaKlassField.getName();
		final Class<?> fieldReturnClass = schemaKlassField.getReturnType();

		// skip default method declarations
		if (schemaKlassField.isDefault()) {
			continue;
		}

		// check for many
		final boolean many = primitiveManager.isMany(fieldReturnClass);

		// check for optional
		final boolean optional = schemaKlassField.isAnnotationPresent(Optional.class);

		// check for key
		final boolean key = schemaKlassField.isAnnotationPresent(Key.class);

		// check for contain
		final boolean contain = schemaKlassField.isAnnotationPresent(Contain.class);

		// add its fields, the owner Klass will be added later
		final Field field = factory.Field();
		field.name(fieldName);
		field.contain(contain);
		field.key(key);
		field.many(many);
		field.optional(optional);

		fieldsForKlass.put(fieldName, field);

		// use klassName and fieldName combo here, 
		// because the real hashCode can not be calculated.
		allFieldsWithReturnType.put(
			klassName + fieldName, new FieldWithMethod(field, schemaKlassField));
	}
	return fieldsForKlass;
}
    \end{lstlisting}
	\caption{SchemaLoader buildFieldsFromMethods method}
	\label{lst:SchemaLoader_buildFieldsFromMethods}
\end{sourcecode}

\section{Wire Types Method}

\begin{sourcecode} [H]
	\begin{lstlisting}[language=Java, escapechar=|]
private static void wireFieldTypes(
	SchemaFactory factory,
	Schema schema,
	Map<String, FieldWithMethod> allFieldsWithReturnType) {

	for (String klassNameFieldNameCombo : allFieldsWithReturnType.keySet()) {
		Method method = allFieldsWithReturnType.get(klassNameFieldNameCombo).method;
		Field field = allFieldsWithReturnType.get(klassNameFieldNameCombo).field;

		// In case the field is multi value (many), that means that the real type is
		// not given in the method.getReturnType() because this will give Set ot List,
		// BUT the real type is in method.getGenericReturnType().
		Class<?> fieldTypeClass;

		// in case it is multi field, get the return the Generic Return Type
		if (field.many()) {
			// The type in this case will be Set or List,
			// but the Generic Return Type will be the actual type.
			ParameterizedType fieldManyType = 
				(ParameterizedType) method.getGenericReturnType();
			fieldTypeClass = (Class<?>) fieldManyType.getActualTypeArguments()[0];
		} else {
			fieldTypeClass = method.getReturnType();
		}

		Type fieldType = getFieldType(fieldTypeClass, schema, factory);
		field.type(fieldType);
	}
}
    \end{lstlisting}
	\caption{SchemaLoader wireFieldTypes method}
	\label{lst:SchemaLoader_wireFieldTypes}
\end{sourcecode}

Listing \ref{lst:SchemaLoader_load} demonstrates the loading process in managed data.
As illustrated we first implement the instances and following that we use setters to wire them up. 
The reason for this is that not everything exists at the time that it needs to be set.
Listing \ref{lst:SchemaLoader_buildTypesFromClasses} show how the klass creation is performed during the schema loading.
Listing \ref{lst:SchemaLoader_buildFieldsFromMethods} illustrates how the fields of a klass are created reflection.
Listing \ref{lst:SchemaLoader_wireFieldTypes} presents the way the types are wired during the schema creation.

This implementation shows the usage of Java reflection in our implementation.
However, because Java reflection capabilities are limited, this restricted our implementation.

% !TEX root = ../thesis.tex

%%%%%%%%%%%%%%%%%%%%%%%%%%%%%%%%%%%%%%%%%%%%%%%%%%%%%%%%%%%%%
% Chapter: Migration
%%%%%%%%%%%%%%%%%%%%%%%%%%%%%%%%%%%%%%%%%%%%%%%%%%%%%%%%%%%%%
\chapter{JHotDraw Migration Process}\label{Migration Process}

\section{DrawingView}
One of the main components of JHotDraw is the \textit{DrawingView} interface.
As Figure \ref{fig:JHotDraw_DrawingView} illustrates, the \textit{DrawingView} is responsible for rendering \texttt{Drawings} and listening to its changes.
Additionally, it is responsible for receiving the user input and delegating it to the current tool.

\begin{figure}[H]
	\centering
  	\fbox{\includegraphics[width=.8\textwidth]{figures/JHotDraw_DrawingView.png}}
  	\caption{DrawingView of JHotDraw}
  	\label{fig:JHotDraw_DrawingView}
\end{figure}

Conclusively, \texttt{DrawingView} makes a good candidate for managed data migration.
The reason is that the specifications of that class can be implemented in data managers and dynamically added to it.

\section{Managed Data DrawingView}
To support sub-typing on the \texttt{DrawingView} interface, we have implemented the \texttt{MDDrawingView}, namely Managed Data DrawingView, which replaced the \texttt{DrawingView} in JHotDraw.
Having this interface for super type, we still needed the actual managed data schemas.
As Figure \ref{fig:JHotDraw_DrawingView} shows, there are two implementations of the \texttt{DrawingView}.
In particular, the \texttt{StandardDrawingView}, which is the implementation that is used when a new drawing view is created in the application and the \texttt{NullDrawingView}, which represents a null drawing view as for the \textit{null-object} pattern.

Following their original design, we have implemented two schemas, one for the \texttt{StandardDrawingView} and one for the \texttt{NullDrawingView}, namely \texttt{MDStandardDrawingView} and \texttt{MDNullDrawingView} respectively.
The instances of those schemas have been used in the same way their counterparts are used in JHotDraw.
A snippet of the \texttt{MDStandardDrawingView} is shown in Listing \ref{lst:MDStandardDrawingView schema} \footnote{Most of the implementation has been omitted for brevity.}.

\begin{sourcecode}[H]
	\begin{lstlisting}[language=Java, escapechar=|]
public interface MDStandardDrawingView extends M, MDDrawingView { |\label{line:MDStandardDrawingView extends M, MDDrawingView}|
	...
	// Composition over inheritance, the original inherits the JPanel
	JPanel panel(JPanel... panel); |\label{line:jpanel composition}|

	default JPanel getPanel() { |\label{line:jpanel getter}|
	    return panel();
	}

	default void setPanel(JPanel _panel) {|\label{line:jpanel setter}|
	    panel(_panel);
	}
	...
	Rectangle damage(Rectangle... damage);
	Drawing drawing(Drawing... drawing);
	...
	default FigureEnumeration selectionZOrdered() { |\label{line:selectionZOrdered}|
		List result = CollectionsFactory.current().createList(selectionCount());
		FigureEnumeration figures = drawing().figures();

		while (figures.hasNextFigure()) {
			Figure f= figures.nextFigure();
			if (isFigureSelected(f)) {
				result.add(f);
			}
		}
		return new ReverseFigureEnumerator(result);
	}
	...
	default void repairDamage() { |\label{line:repairDamage}|
		if (getDamage() != null) {
			panel().repaint(damage().x, damage().y, damage().width, damage().height);
			setDamage(null);
		}
	}
	...
}
	\end{lstlisting}
	\caption{MDStandardDrawingView schema}
	\label{lst:MDStandardDrawingView schema}
\end{sourcecode}

Listing \ref{lst:MDStandardDrawingView schema} shows that the \texttt{MDStandardDrawingView} interface extends both \texttt{M} interface, defining that this is a schema definition, and \texttt{MDDrawingView}, for sub-type support.
Additionally, all the functionalities implemented in methods of the original \texttt{DrawingView}, in managed data they are implemented in default methods of the schema interface.
The fields of a schema can provide those methods with the managed object's current state.
As Lines \ref{line:selectionZOrdered} and \ref{line:repairDamage} show, the fields of the schema can be used to query their values inside the default methods.
Note that the code in the default methods is identical to the original \texttt{DrawingView}.
Furthermore, for consistency with the legacy code, we have implemented setters and getters, Lines \ref{line:jpanel setter} and \ref{line:jpanel getter}, for field values accessors.
This way we maintained consistency across in accessing values of the managed object.

A notable issue is that the original \texttt{StandardDrawingView} extends the \texttt{javax.swing.jpanel} class as Figure \ref{fig:JHotDraw_DrawingView} shows.
However, such a structure is not supported in managed data. 
Schema definitions can not extend classes.
To overcome this issue we defined the \texttt{JPanel} as a field in the schema, namely \textit{panel}.
To support the \texttt{JPanel} as a type of a field though, it is needs ti be defined as managed data.
By all means, the same holds for the remaining fields, such as \texttt{Rectangle} and \texttt{Drawing}.

As explained in Section \ref{Primitives Definition}, our framework provides external primitives definition by inheriting the \texttt{Primitives} interface.
The JHotDraw primitives definition is shown in Listing \ref{lst:JHotDraw Primitives Definition}.

\begin{sourcecode}[H]
	\begin{lstlisting}[language=Java, escapechar=|]
public interface JHotDrawPrimitives extends Primitives {
	javax.swing.JPanel JPanel();

	java.awt.Color Color();
	java.awt.Cursor Cursor();
	java.awt.Point Point();
	java.awt.Dimension Dimension();
	java.awt.Rectangle Rectangle();

	CH.ifa.draw.framework.DrawingEditor DrawingEditor();
	CH.ifa.draw.framework.Drawing Drawing();
	CH.ifa.draw.framework.Painter Painter();
	CH.ifa.draw.framework.PointConstrainer PointConstrainer();

	...
}
	\end{lstlisting}
	\caption{JHotDraw Primitives Definition}
	\label{lst:JHotDraw Primitives Definition}
\end{sourcecode}

This has been proven very helpful since we did not need to re-implement every field as managed data during the refactoring. 
Especially, classes that are provided by libraries such as \texttt{javax.swing} and \texttt{java.awt}.

\section{MDDrawingView Schema Factories}
In order to create instances of the defined \texttt{MDStandardDrawingView} and \texttt{MDNullDrawingView} schemas, we needed their factories.
Besides the schema factories, which is as simple as Listing \ref{lst:DrawingViewSchemaFactory} shows, we still needed a way to give initialization values to the schema instances the same way that the original \texttt{StandardDrawingView} does during construction.
Additionally, this factory should be used like Java's \texttt{new} keyword in the source code.
This factory just replicates the original \texttt{StandardDrawingView} constructor and is used from the program to create new instances of the schemas.
The code of the \texttt{MDStandardDrawingView} \textit{instances factory} is illustrated in Listing \ref{lst:MDStandardDrawingView Instances Factory}, in comparison to the original constructor, illustrated in Listing \ref{lst:DrawingView Constructor}.

\begin{sourcecode}[H]
	\begin{lstlisting}[language=Java, escapechar=|]
public interface DrawingViewSchemaFactory extends IFactory {
	MDStandardDrawingView DrawingView();
	MDNullDrawingView NullDrawingView();
}
	\end{lstlisting}
	\caption{DrawingView Schema Factory}
	\label{lst:DrawingViewSchemaFactory}
\end{sourcecode}

\begin{sourcecode}[H]
	\begin{lstlisting}[language=Java, escapechar=|]
public StandardDrawingView(DrawingEditor editor, int width, int height) {
	setAutoscrolls(true);
	fEditor = editor;
	fViewSize = new Dimension(width,height);
	setSize(width, height);
	fSelectionListeners = CollectionsFactory.current().createList();
	addFigureSelectionListener(editor()); |\label{line:addFigureSelectionListener_contructor}|
	setLastClick(new Point(0, 0));
	fConstrainer = null;
	fSelection = CollectionsFactory.current().createList();
	setDisplayUpdate(createDisplayUpdate());
	setBackground(Color.lightGray);
	addMouseListener(createMouseListener());
	addMouseMotionListener(createMouseMotionListener());
	addKeyListener(createKeyListener());
}
	\end{lstlisting}
	\label{lst:DrawingView Constructor}
	\caption{Original StandardDrawingView Constructor}
\end{sourcecode}

\begin{sourcecode}[H]
	\begin{lstlisting}[language=Java, escapechar=|]
public static MDDrawingView newDrawingView(
	DrawingEditor editor, int width, int height) {
	final MDStandardDrawingView drawingView = drawingViewSchemaFactory.DrawingView();
	MyJPanel jPanel = new MyJPanel();
	jPanel.setAutoscrolls(true);
 	jPanel.setSize(width, height);
	jPanel.setBackground(Color.lightGray);
	drawingView.panel(jPanel);
	jPanel.setDrawingView(drawingView);

	drawingView.editor(editor);
	drawingView.size(new Dimension(width, height));
	jPanel.setSize(width, height);
	drawingView.lastClick(new Point(0, 0));
	drawingView.constrainer(null);
	drawingView.setDisplayUpdate(new SimpleUpdateStrategy());
	drawingView.setBackground(Color.lightGray);
	drawingView.drawing(new StandardDrawing());

	jPanel.addMouseListener(...);
	jPanel.addMouseMotionListener(...);
	jPanel.addKeyListener(...);
	return drawingView;
}
	\end{lstlisting}
	\label{lst:MDStandardDrawingView Instances Factory}
	\caption{MDStandardDrawingView Instances Factory}
\end{sourcecode}

\section{MDDrawingView Integration}
Finally, in order to integrate the \texttt{MDDrawingView} managed objects in the existing system, first we had to replace every instance of \texttt{DrawingView} with \texttt{MDDrawingView}, every \texttt{StandardDrawingView} with \texttt{MDStandardDrawingView} and every \texttt{NullDrawingView} with \texttt{MDNullDrawingView} accordingly.
In addition, everywhere a new instance of these is created, we replaced it with our \textit{instances factory}.

For instance Listings \ref{lst:original_createDrawingView} and \ref{lst:refactored_createDrawingView} show how the code has been changed in the \texttt{DrawApplication} class.

\lstdefinestyle{smallJava}{
  basicstyle={\scriptsize\ttfamily},
  language=Java
}

\noindent\begin{minipage}{.45\textwidth}
\begin{lstlisting}[
	style=smallJava,
	caption=Original createDrawingView,
	frame=tlrb,
	label=lst:original_createDrawingView
]
Dimension d = getDrawingViewSize();

DrawingView newDrawingView = 
	new StandardDrawingView(this, d.width, d.height);

newDrawingView.setDrawing(newDrawing);
\end{lstlisting}
\end{minipage}\hfill
\begin{minipage}{.47\textwidth}
\begin{lstlisting}[
	style=smallJava,
	caption=ManagedData createDrawingView,
	frame=tlrb,
	label=lst:refactored_createDrawingView
]
Dimension d = getDrawingViewSize();

MDDrawingView newDrawingView = 
	MDDrawingViewFactory
		.newSubjectRoleDrawingView(this, d.width, d.height);

newDrawingView.setDrawing(newDrawing);
	\end{lstlisting}
\label{lst:createDrawingView}
\end{minipage}

The factory's code of the \texttt{MDDrawingView} with the observable data manager is shown in Listing \ref{lst:ManagedDataJHotDraw_MDDrawingView_Factory}.

\begin{sourcecode} [H]
	\begin{lstlisting}[language=Java]
public static MDDrawingView newSubjectRoleDrawingView(
	DrawingEditor editor, int width, int height) {

	Schema drawingViewSchema = SchemaLoader.load(
			schemaFactory,
			JHotDrawPrimitives.class, MDStandardDrawingView.class);

	FigureSelectionListenerSubjectRoleDataManager subjectRoleFactory =
		new FigureSelectionListenerSubjectRoleDataManager();

	DrawingViewSchemaFactory drawingViewSchemaFactory = subjectRoleFactory.factory(
			DrawingViewSchemaFactory.class, drawingViewSchema);

	MDStandardDrawingView drawingView = drawingViewSchemaFactory.DrawingView();

	MyJPanel jPanel = new MyJPanel();
	jPanel.setAutoscrolls(true);
	....
	drawingView.editor(editor);

	drawingView.size(new Dimension(width, height));
	jPanel.setSize(width, height);

	...

	// Panel events
	jPanel.addMouseListener(new MouseAdapter() {...});
	jPanel.addMouseMotionListener(new MouseMotionListener() {...});
	jPanel.addKeyListener(new DrawingViewKeyListener(drawingView));

	return drawingView;
}
	\end{lstlisting}
	\caption{ManagedDataJHotDraw: MDDrawingView Factory}
	\label{lst:ManagedDataJHotDraw_MDDrawingView_Factory}
\end{sourcecode}
\include{./appendices/ImplementingDataManager}
% !TEX root = ../thesis.tex

%%%%%%%%%%%%%%%%%%%%%%%%%%%%%%%%%%%%%%%%%%%%%%%%%%%%%%%%%%%%%%%%%%%%%%%%%%%%%%%
% Chapter: Metrics Results
%%%%%%%%%%%%%%%%%%%%%%%%%%%%%%%%%%%%%%%%%%%%%%%%%%%%%%%%%%%%%%%%%%%%%%%%%%%%%%%
\chapter{Metrics Results}\label{Metrics Results}

\section{JHotDraw Results}

\subsection{FigureSelectionListener}

\begin{table}[H]
\centering
\resizebox{\textwidth}{!}{%
	\begin{tabular}{@{}lccccccccc@{}}
	\toprule
	\textbf{} & \multicolumn{2}{c}{\textbf{Coupling}} & \textbf{Cohesion} & \multicolumn{3}{c}{\textbf{Size}} & \multicolumn{3}{c}{\textbf{Separation of Concerns}} \\ \midrule
	\textbf{Class} & \textbf{CBC} & \textbf{DIT} & \textbf{LCOO} & \textbf{WOC} & \textbf{LOC} & \textbf{NOA} & \textbf{CDC} & \textbf{CDO} & \textbf{CDLOC} \\
	DrawingView (Subject) & - & - & - & - & 54 & - & 1 & 2 & - \\
	StandardDrawingView (concrete Subject) & 35 & 5 & 17 & 136 & 629 & 21 & 1 & 9 & 20 \\
	NullDrawingView (concrete Subject) & 18 & 5 & 46 & 54 & 155 & 5 & 1 & 2 & 2 \\
	FigureSelectionListener  (Observer) & - & - & - & - & 4 & - & 1 & 1 & - \\
	DrawingEditor  (Interface) & - & - & - & - & 14 & - & 1 & 1 & - \\
	DrawApplication (concrete Observer) & 71 & 6 & 2 & 144 & 726 & 20 & 1 & 1 & 4 \\
	AbstractCommand (concrete Observer) & 33 & 1 & 6 & 28 & 133 & 4 & 1 & 3 & 8 \\
	UndoableCommand (concrete Observer) & 13 & 1 & 4 & 21 & 78 & 3 & 1 & 2 & 12 \\ \bottomrule
	\end{tabular}}
\caption{JHotDraw FigureSelectionListener Metrics}
\label{tbl:JHotDraw FigureSelectionListener Metrics}
\end{table}

\begin{table}[H]
\centering
\resizebox{\textwidth}{!}{%
	\begin{tabular}{@{}lcccccccccc@{}}
	\toprule
	 & \multicolumn{2}{c}{\textbf{Coupling}} & \textbf{Cohesion} & \multicolumn{4}{c}{\textbf{Size}} & \multicolumn{3}{c}{\textbf{Separation of Concerns}} \\ \midrule
	 & \textbf{CBC} & \textbf{DIT} & \textbf{LCOO} & \textbf{WOC} & \textbf{LOC} & \textbf{NOA} & \textbf{VS} & \textbf{CDC} & \textbf{CDO} & \textbf{CDLOC} \\
	\textbf{Total} & 170 & 18 & 75 & 383 & 1793 & 53 & 8 & 8 & 21 & 46 \\
	\textbf{Max} & 71 & 6 & 46 & 144 & 726 & 21 & - & - & 9 & 20 \\
	\textbf{Min} & 13 & 1 & 2 & 21 & 4 & 3 & - & - & 1 & 2 \\
	\textbf{Average} & 34 & 3.6 & 15 & 76.6 & 224.125 & 10.6 & - & - & 2.625 & 9.2 \\ \bottomrule
	\end{tabular}}
\caption{JHotDraw FigureSelectionListener Totals}
\label{tbl:JHotDraw FigureSelectionListener Totals}
\end{table}

\subsection{ChangeAttributeCommand}

\begin{table}[H]
\centering
\resizebox{\textwidth}{!}{%
	\begin{tabular}{@{}lcccccccccc@{}}
	\toprule
	\textbf{} & \multicolumn{2}{c}{\textbf{Coupling}} & \textbf{Cohesion} & \multicolumn{3}{c}{\textbf{Size}} & \multicolumn{3}{c}{\textbf{Separation of Concerns}} \\ \midrule
	\textbf{Class} & \textbf{CBC} & \textbf{DIT} & \textbf{LCOO} & \textbf{WOC} & \textbf{LOC} & \textbf{NOA} & \textbf{CDC} & \textbf{CDO} & \textbf{CDLOC} \\
	Command & - & - & - & - & 12 & - & 1 & 2 & - \\
	AbstractCommand & 33 & 1 & 6 & 28 & 133 & 4 & 1 & 4 & 12 \\
	UndoableCommand & 13 & 1 & 4 & 21 & 78 & 3 & 1 & 2 & 4 \\
	ChangeAttributeCommand & 11 & 2 & 2 & 5 & 103 & 2 & 1 & 3 & 6 \\
	DrawApplication & 71 & 6 & 2 & 144 & 726 & 20 & 1 & 5 & 10 \\ \bottomrule
	\end{tabular}}
\caption{JHotDraw ChangeAttributeCommand Metrics}
\label{tbl:JHotDraw ChangeAttributeCommand Metrics}
\end{table}

\begin{table}[H]
\centering
\resizebox{\textwidth}{!}{%
	\begin{tabular}{@{}lcccccccccc@{}}
	\toprule
	 & \multicolumn{2}{c}{\textbf{Coupling}} & \textbf{Cohesion} & \multicolumn{4}{c}{\textbf{Size}} & \multicolumn{3}{c}{\textbf{Separation of Concerns}} \\ \midrule
	 & \textbf{CBC} & \textbf{DIT} & \textbf{LCOO} & \textbf{WOC} & \textbf{LOC} & \textbf{NOA} & \textbf{VS} & \textbf{CDC} & \textbf{CDO} & \textbf{CDLOC} \\
	\textbf{Total} & 128 & 10 & 14 & 198 & 1052 & 29 & 5 & 5 & 16 & 32 \\
	\textbf{Max} & 71 & 6 & 6 & 144 & 726 & 20 & - & - & 5 & 12 \\
	\textbf{Min} & 11 & 1 & 2 & 5 & 12 & 2 & - & - & 2 & 4 \\
	\textbf{Average} & 32 & 2.5 & 3.5 & 49.5 & 210.4 & 7.25 & - & - & 3.2 & 8 \\ \bottomrule
	\end{tabular}}
\caption{JHotDraw ChangeAttributeCommand Totals}
\label{tbl:JHotDraw ChangeAttributeCommand Totals}
\end{table}

% \section{AJHotDraw Results}

\section{ManagedDataJHotDraw Results}

\subsection{FigureSelectionListener}

\begin{table}[H]
\centering
\resizebox{\textwidth}{!}{%
	\begin{tabular}{@{}lccccccccc@{}}
	\toprule
	\textbf{} & \multicolumn{2}{c}{\textbf{Coupling}} & \textbf{Cohesion} & \textbf{Size} & \textbf{} & \textbf{} & \multicolumn{3}{c}{\textbf{Separation of Concerns}} \\ \midrule
	\textbf{Class / Data manager} & \textbf{CBC} & \textbf{DIT} & \textbf{LCOO} & \textbf{WOC} & \textbf{LOC} & \textbf{NOA} & \textbf{CDC} & \textbf{CDO} & \textbf{CDLOC} \\
	MDDrawingView (Interface) & - & - & - & - & 111 & - & 0 & 0 & 0 \\
	MDStandardDrawingView (Managed Data) & 26 & 2 & 16 & 120 & 471 & 18 & 0 & 0 & 0 \\
	MDNullDrawingView (Managed Data) & 15 & 2 & 31 & 38 & 120 & 8 & 0 & 0 & 0 \\
	MDDrawingViewFactory (Helper) & 17 & 1 & 1 & 8 & 177 & - & 1 & 1 & 0 \\
	DrawingViewSchemaFactory (SchemaFactory) & - & - & - & - & 4 & 0 & 0 & 0 & 0 \\
	FigureSelectionListenerMObject (Data Manager) & 5 & 3 & 4 & 6 & 28 & 0 & 1 & 4 & 0 \\
	DrawingEditor  (Interface) & - & - & - & - & 12 & - & 0 & 0 & - \\
	DrawApplication (concrete Observer) & 72 & 6 & 2 & 145 & 732 & 20 & 1 & 1 & 2 \\
	AbstractCommand (concrete Observer) & 33 & 1 & 6 & 28 & 133 & 4 & 0 & 0 & 0 \\
	UndoableCommand (concrete Observer) & 16 & 1 & 4 & 23 & 90 & 3 & 1 & 2 & 4 \\ \bottomrule
	\end{tabular}}
\caption{MDJHotDraw FigureSelectionListener Metrics}
\label{tbl:MDJHotDraw FigureSelectionListener Metrics}
\end{table}

\begin{table}[H]
\centering
\resizebox{\textwidth}{!}{%
	\begin{tabular}{@{}lccccccccc@{}}
	\toprule
	 & \multicolumn{2}{c}{\textbf{Coupling}} & \textbf{Cohesion} & \multicolumn{3}{c}{\textbf{Size}} & \multicolumn{3}{c}{\textbf{Separation of Concerns}} \\ \midrule
	 & \textbf{CBC} & \textbf{DIT} & \textbf{LCOO} & \textbf{WOC} & \textbf{LOC} & \textbf{NOA} & \textbf{CDC} & \textbf{CDO} & \textbf{CDLOC} \\
	\textbf{Total} & 184 & 16 & 64 & 368 & 1878 & 53 & 4 & 8 & 6 \\
	\textbf{Max} & 72 & 6 & 31 & 145 & 732 & 20 & - & 4 & 4 \\
	\textbf{Min} & 5 & 1 & 1 & 6 & 4 & 0 & - & 0 & 0 \\
	\textbf{Average} & 26.285 & 2.285 & 9.142 & 52.571 & 187.8 & 7.571 & - & 0.8 & 0.66 \\ \bottomrule
	\end{tabular}}
\caption{MDJHotDraw FigureSelectionListener Totals}
\label{tbl:MDJHotDraw FigureSelectionListener Totals}
\end{table}

\subsection{ChangeAttributeCommand}

\begin{table}[H]
\centering
\resizebox{\textwidth}{!}{%
	\begin{tabular}{@{}lcccccccccc@{}}
	\toprule
	 & \multicolumn{2}{c}{\textbf{Coupling}} & \textbf{Cohesion} & \multicolumn{4}{c}{\textbf{Size}} & \multicolumn{3}{c}{\textbf{Separation of Concerns}} \\ \midrule
	\textbf{Class / Data manager} & \textbf{CBC} & \textbf{DIT} & \textbf{LCOO} & \textbf{WOC} & \textbf{LOC} & \textbf{NOA} & \textbf{VS} & \textbf{CDC} & \textbf{CDO} & \textbf{CDLOC} \\
	Command & - & - & - & - & 10 & - & 1 & 0 & - & 0 \\
	AbstractCommand & 33 & 1 & 6 & 28 & 133 & 4 & 1 & 0 & 0 & 0 \\
	UndoableCommand & 16 & 1 & 3 & 23 & 90 & 3 & 1 & 1 & 1 & 2 \\
	UndoableChangeAttrCmdMObject & 7 & 3 & 1 & 3 & 98 & 0 & 1 & 1 & 2 & 0 \\
	DrawApplication & 72 & 6 & 2 & 145 & 732 & 20 & 1 & 1 & 5 & 10 \\ \bottomrule
	\end{tabular}}
\caption{MDJHotDraw ChangeAttributeCommand Metrics}
\label{tbl:MDJHotDraw ChangeAttributeCommand Metrics}
\end{table}

\begin{table}[H]
\centering
\resizebox{\textwidth}{!}{%
	\begin{tabular}{@{}lcccccccccc@{}}
	\toprule
	\multicolumn{1}{c}{\textbf{}} & \multicolumn{2}{c}{\textbf{Coupling}} & \textbf{Cohesion} & \multicolumn{4}{c}{\textbf{Size}} & \multicolumn{3}{c}{\textbf{Separation of Concerns}} \\ \midrule
	\multicolumn{1}{c}{\textbf{}} & \textbf{CBC} & \textbf{DIT} & \textbf{LCOO} & \textbf{WOC} & \textbf{LOC} & \textbf{NOA} & \textbf{VS} & \textbf{CDC} & \textbf{CDO} & \textbf{CDLOC} \\
	\textbf{Total} & 128 & 11 & 12 & 199 & 1063 & 27 & 5 & 3 & 8 & 12 \\
	\textbf{Max} & 72 & 6 & 6 & 145 & 732 & 20 & - & - & 5 & 10 \\
	\textbf{Min} & 7 & 1 & 1 & 3 & 10 & 0 & - & - & 0 & 0 \\
	\textbf{Average} & 32 & 2.75 & 3 & 49.75 & 212.6 & 6.75 & - & - & 2 & 2.4 \\ \bottomrule
	\end{tabular}}
\caption{MDJHotDraw ChangeAttributeCommand Totals}
\label{tbl:MDJHotDraw ChangeAttributeCommand Totals}
\end{table}

\section{Metrics Comparison Graphs}\label{Metrics Comparison Graphs}

\subsection{FigureSelectionListener}

\begin{figure}[H]
	\centering
	\includegraphics[scale=0.82]{figures/metrics/Metric_Observer_Coupling.png}
	\caption{FigureSelectionListener Coupling}
  	\label{fig:FigureSelectionListener Coupling}
\end{figure}

\begin{figure}[H]
	\centering
	\includegraphics[scale=0.82]{figures/metrics/Metric_Observer_Cohesion.png}
	\caption{FigureSelectionListener Cohesion}
  	\label{fig:FigureSelectionListener Cohesion}
\end{figure}

\begin{figure}[H]
	\centering
	\includegraphics[scale=0.81]{figures/metrics/Metric_Observer_Size.png}
	\caption{FigureSelectionListener Size}
  	\label{fig:FigureSelectionListener Size}
\end{figure}

\begin{figure}[H]
	\centering
	\includegraphics[scale=0.81]{figures/metrics/Metric_Observer_SoC.png}
	\caption{FigureSelectionListener Separation of Concerns}
  	\label{fig:FigureSelectionListener SoC}
\end{figure}

\subsection{ChangeAttributeCommand}

\begin{figure}[H]
	\centering
	\includegraphics[scale=0.82]{figures/metrics/Metric_Undo_Coupling.png}
	\caption{ChangeAttributeCommand Coupling}
  	\label{fig:ChangeAttributeCommand Coupling}
\end{figure}

\begin{figure}[H]
	\centering
	\includegraphics[scale=0.82]{figures/metrics/Metric_Observer_Cohesion.png}
	\caption{ChangeAttributeCommand Cohesion}
  	\label{fig:ChangeAttributeCommand Cohesion}
\end{figure}

\begin{figure}[H]
	\centering
	\includegraphics[scale=0.81]{figures/metrics/Metric_Undo_Size.png}
	\caption{ChangeAttributeCommand Size}
  	\label{fig:ChangeAttributeCommand Size}
\end{figure}

\begin{figure}[H]
	\centering
	\includegraphics[scale=0.81]{figures/metrics/Metric_Undo_SoC.png}
	\caption{ChangeAttributeCommand Separation of Concerns}
  	\label{fig:ChangeAttributeCommand SoC}
\end{figure}


{%\tiny
  \bibliographystyle{alphaurl}
  \bibliography{thesis}
}

\end{document}
